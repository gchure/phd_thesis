%%%%%%%%%%%%%%%%%%%%%%%%%%%%%%%%%%%%%%%%%%%%%%%%%%%%%%%%%%%%%%%%%%%%%%%%%%%
\section*{Results}

\subsection*{Characterizing Transcription Factor Induction using the Monod-Wyman-Changeux (MWC) Model}

We begin by considering a simple repression genetic
architecture in which the binding of an allosteric repressor occludes the
binding of RNA polymerase (RNAP) to the DNA \citep{Ackers1982,Buchler2003}. When
an effector (hereafter referred to as an ``inducer" for the case of induction)
binds to the repressor, it shifts the repressor's allosteric equilibrium towards
the inactive state as specified by the MWC model \citep{MONOD1965}. This causes
the repressor to bind more weakly to the operator, which increases gene
expression. Simple repression motifs in the absence of inducer have been
previously characterized by an equilibrium model where the probability of each
state of repressor and RNAP promoter occupancy is dictated by the Boltzmann
distribution \citep{Ackers1982, Buchler2003, Vilar2003, Bintu2005a, Garcia2011,
Brewster2014, } (we note that non-equilibrium models of simple repression have
been shown to have the same functional form that we derive below
\citep{Phillips2015a}). We extend these models to consider allostery
by accounting for the equilibrium state of the repressor through the MWC model.

Thermodynamic models of gene expression begin by enumerating all possible states
of the promoter and their corresponding statistical weights. As shown in
\fref[fig_polymerase_repressor_states]\letter{A}, the promoter can either be
empty, occupied by RNAP, or occupied by either an active or inactive repressor. The probability of binding to the promoter will be affected by the protein copy number, which we denote as $P$ for RNAP, $R_{A}$ for active repressor, and $R_{I}$ for inactive repressor. We note that repressors fluctuate between the active and inactive conformation in thermodynamic equilibrium, such that $R_{A}$ and $R_{I}$ will remain constant for a given inducer concentration \citep{MONOD1965}.
We assign the repressor a different DNA binding affinity in the active and
inactive state. In addition to the specific binding sites at the promoter, we
assume that there are $N_{NS}$ non-specific binding sites elsewhere (i.e. on
parts of the genome outside the simple repression architecture) where the RNAP
or the repressor can bind. All specific binding energies are measured relative
to the average non-specific binding energy. Thus, \(\Delta\varepsilon_{P}\) represents the energy difference
between the specific and non-specific binding for RNAP to the DNA. Likewise,
\(\Delta\varepsilon_{RA}\) and \(\Delta\varepsilon_{RI}\) represent the
difference in specific and non-specific binding energies for repressor in the
active or inactive state, respectively.

\begin{figure}
	\centering \includegraphics{main_figs/fig2.pdf}

	\caption{\textbf{States and weights for the simple repression motif.}
	\letterParen{A} RNAP (light blue) and a repressor compete for binding to a
	promoter of interest. There are $R_A$ repressors in the active state (red) and
	$R_I$ repressors in the inactive state (purple). The difference in energy
	between a repressor bound to the promoter of interest versus another
	non-specific site elsewhere on the DNA equals $\Delta\varepsilon_{RA}$ in the
	active state and $\Delta\varepsilon_{RI}$ in the inactive state; the $P$ RNAP
	have a corresponding energy difference $\Delta\varepsilon_{P}$ relative to
	non-specific binding on the DNA. $N_{NS}$ represents the number of non-specific
	binding sites for both RNAP and repressor. \letterParen{B} A repressor has an
	active conformation (red, left column) and an inactive conformation (purple, right
	column), with the energy difference between these two states given by $\Delta
	\varepsilon_{AI}$. The inducer (blue circle) at concentration $c$ is capable of
	binding to the repressor with dissociation constants $K_A$ in the active state
	and $K_I$ in the inactive state. The eight states for a dimer with $n=2$
	inducer binding sites are shown along with the sums of the active and inactive states.} \label{fig_polymerase_repressor_states}
\end{figure}

Thermodynamic models of transcription \citep{Ackers1982, Buchler2003, Vilar2003,
Bintu2005, Bintu2005a, Kuhlman2007, Daber2011a, Garcia2011, Brewster2014,
Weinert2014} posit that gene expression is proportional to the probability that
the RNAP is bound to the promoter $p_{\text{bound}}$, which is given by
\begin{equation}\label{eq_p_bound_definition}
p_\text{bound}=\frac{\frac{P}{N_{NS}}e^{-\beta \Delta\varepsilon_{P}}}{1+\frac{R_A}{N_{NS}}e^{-\beta \Delta\varepsilon_{RA}}+\frac{R_I}{N_{NS}}e^{-\beta \Delta\varepsilon_{RI}}+\frac{P}{N_{NS}}e^{-\beta\Delta\varepsilon_{P}}},
\end{equation}
with $\beta = \frac{1}{k_BT}$ where $k_B$ is the Boltzmann constant and $T$ is
the temperature of the system. As $k_BT$ is the natural unit of energy at the
molecular length scale, we treat the products $\beta \Delta\varepsilon_{j}$ as
single parameters within our model. Measuring $p_{\text{bound}}$ directly is
fraught with experimental difficulties, as determining the exact proportionality between expression and $p_{\text{bound}}$ is not straightforward. Instead, we measure the
fold-change in gene expression due to the presence of the repressor. We define
fold-change as the ratio of gene expression in the presence of repressor
relative to expression in the absence of repressor (i.e. constitutive expression), namely,
\begin{equation}\label{eq_fold_change_definition}
\foldchange \equiv \frac{p_\text{bound}(R > 0)}{p_\text{bound}(R = 0)}.
\end{equation}
We can simplify this expression using two well-justified approximations: (1)
$\frac{P}{N_{NS}}e^{-\beta\Delta\varepsilon_{P}}\ll 1$ implying that the RNAP
binds weakly to the promoter ($N_{NS} = 4.6 \times 10^6$, $P \approx 10^3$
\citep{Klumpp2008}, $\Delta\varepsilon_{P} \approx -2 \,\, \text{to} \, -5~k_B
T$ \citep{Brewster2012}, so that $\frac{P}{N_{NS}}e^{-\beta\Delta\varepsilon_{P}}
\approx 0.01$) and (2) $\frac{R_I}{N_{NS}}e^{-\beta \Delta\varepsilon_{RI}} \ll
1 + \frac{R_A}{N_{NS}} e^{-\beta\Delta\varepsilon_{RA}}$ which reflects our
assumption that the inactive repressor binds weakly to the promoter of interest.
Using these approximations, the fold-change reduces to the form
\begin{equation}\label{eq_fold_change_approx}
\foldchange \approx \left(1+\frac{R_A}{N_{NS}}e^{-\beta \Delta\varepsilon_{RA}}\right)^{-1} \equiv \left( 1+p_A(c) \frac{R}{N_{NS}}e^{-\beta
	\Delta\varepsilon_{RA}} \right)^{-1},
\end{equation}
where in the last step we have introduced the fraction $p_A(c)$ of repressors in
the active state given a concentration $c$ of inducer, such that
$R_A(c)=p_A(c) R$. Since inducer binding shifts the repressors from the active
to the inactive state, $p_A(c)$ grows smaller as $c$ increases
\citep{Marzen2013}.

We use the MWC model to compute the probability $p_A(c)$ that a repressor with $n$ inducer binding
sites will be active. The value of $p_A(c)$ is given by the sum
of the weights of the active repressor states divided by the sum of the weights of all
possible repressor states (see
\fref[fig_polymerase_repressor_states]\letter{B}), namely,
\begin{equation}\label{eq_p_active}
p_A(c)=\frac{\left(1+\frac{c}{K_A}\right)^n}{\left(1+\frac{c}{K_A}\right)^n+e^{-\beta \Delta \varepsilon_{AI} }\left(1+\frac{c}{K_I}\right)^n},
\end{equation}
where $K_A$ and $K_I$ represent the dissociation constant between the inducer
and repressor in the active and inactive states, respectively, and $\Delta
\varepsilon_{AI} = \varepsilon_{I} - \varepsilon_{A}$ is the free energy
difference between a repressor in the inactive and active state (the quantity
$e^{-\Delta \varepsilon_{AI}}$ is sometimes denoted by $L$ \citep{MONOD1965,
Marzen2013} or $K_{\text{RR}*}$ \citep{Daber2011a}). In this equation, $\frac{c}{K_A}$ and $\frac{c}{K_I}$ represent the change in free energy when an inducer binds to a repressor in the active or inactive state, respectively, while $e^{-\beta \Delta \varepsilon_{AI} }$ represents the change in free energy when the repressor changes from the active to inactive state in the absence of inducer. Thus, a repressor which favors
the active state in the absence of inducer ($\Delta \varepsilon_{AI} > 0$) will
be driven towards the inactive state upon inducer binding when $K_I < K_A$. The
specific case of a repressor dimer with $n=2$ inducer binding sites is shown in
\fref[fig_polymerase_repressor_states]\letter{B}.

Substituting $p_A(c)$ from \eref[eq_p_active] into \eref[eq_fold_change_approx]
yields the general formula for induction of a simple repression regulatory
architecture \citep{Phillips2015a}, namely,
\begin{equation}\label{eq_fold_change_full}
\foldchange = \left(
1+\frac{\left(1+\frac{c}{K_A}\right)^n}{\left(1+\frac{c}{K_A}\right)^n+e^{-\beta \Delta \varepsilon_{AI} }\left(1+\frac{c}{K_I}\right)^n}\frac{R}{N_{NS}}e^{-\beta \Delta\varepsilon_{RA}} \right)^{-1}.
\end{equation}
While we have used the specific case of simple repression with induction to
craft this model, the same mathematics describe the case
of corepression in which binding of an allosteric effector stabilizes the active
state of the repressor and decreases gene expression (see
\fref[figInductionCorepressionPhenotypicProperties]\letter{B}). Interestingly,
we shift from induction (governed by $K_I < K_A$) to corepression ($K_I > K_A$)
as the ligand transitions from preferentially binding to the inactive repressor
state to stabilizing the active state. Furthermore, this general approach can be
used to describe a variety of other motifs such as activation, multiple
repressor binding sites, and combinations of activator and repressor binding
sites \citep{Bintu2005, Brewster2014, Weinert2014}.

The formula presented in \eref[eq_fold_change_full] enables us to make
precise quantitative statements about induction profiles. Motivated by the broad
range of predictions implied by \eref[eq_fold_change_full], we designed a series of experiments using the
\textit{lac} system in \textit{E. coli} to tune the control parameters for a
simple repression genetic circuit. As discussed in \fref[figInductionCorepressionPhenotypicProperties]\letterParen{C},
previous studies from our lab have provided well-characterized values
for many of the parameters in our experimental system, leaving only the values
of the the MWC parameters ($K_A$, $K_I$, and $\Delta \varepsilon_{AI}$) to be
determined. We note that while previous studies have obtained values for $K_A$,
$K_I$, and $L=e^{-\beta \Delta \varepsilon_{AI}}$ \citep{OGorman1980,
Daber2011a}, they were either based upon biochemical experiments or
\textit{in vivo} conditions involving poorly characterized transcription factor
copy numbers and gene copy numbers. These differences relative to our
experimental conditions and fitting techniques led us to believe that it was
important to perform our own analysis of these parameters. After
inferring these three MWC parameters (see \nameref{star_methods}, Section
``\nameref{AppendixModel}'' for details regarding the inference of $\Delta
\varepsilon_{AI}$, which was fitted separately from $K_A$ and $K_I$), we were
able to predict the input/output response of the system under a broad range of
experimental conditions. For example, this framework can predict the response of
the system at different repressor copy numbers $R$, repressor-operator
affinities $\Delta\varepsilon_{RA}$, inducer concentrations $c$, and gene copy
numbers (see Appendix \ref{AppendixFugacity}).
