%%%%%%%%%%%%%%%%%%%%%%%%%%%%%%%%%%%%%%%%%%%%%%%%%%%%%%%%%%%%%%%%%%%%%%%%%%%
\section*{Methods} \label{section_methods}

\subsection*{Bacterial Strains and DNA Constructs}

All strains used in these experiments were derived from \textit{E. coli} K12
MG1655 with the \textit{lac} operon removed, adapted from those created and
described in \textcite{Garcia2011, Garcia2011B}. Briefly, the operator variants
and YFP reporter gene were cloned into a pZS25 background which contains a
\textit{lacUV5} promoter that drives expression as is shown schematically in
\fref[fig_polymerase_repressor_states]. These constructs carried a kanamycin
resistance gene and were integrated into the \textit{galK} locus of the
chromosome using $\lambda$ Red recombineering \citep{Sharan2009}. The
\textit{lacI} gene was constitutively expressed via a
P$_\mathrm{LtetO\hbox{-}1}$ promoter \citep{Lutz1997}, with ribosomal binding
site mutations made to vary the LacI copy number as described in
\textcite{Salis2009} using site-directed mutagenesis (Quickchange II;
Stratagene), with further details in \textcite{Garcia2011}. These {\it lacI}
constructs carried a chloramphenicol resistance gene and were integrated into
the \textit{ybcN} locus of the chromosome. Final strain construction was
achieved by performing repeated P1 transduction \citep{Thomason2007} of the
different operator and \textit{lacI} constructs to generate each combination
used in this work. Integration was confirmed by PCR amplification of the
replaced chromosomal region and by sequencing. Primers and final strain
genotypes are listed in Appendix \ref{AppendixStrainlist}.

It is important to note that the rest of the \textit{lac} operon
(\textit{lacZYA}) was never expressed. The LacY protein is a transmembrane
protein which actively transports lactose as well as IPTG into the cell. As LacY
was never produced in our strains, we assume that the extracellular and
intracellular IPTG concentration was approximately equal due to diffusion across
the membrane into the cell as is suggested by previous work
\citep{FernandezCastane2012}.

To make this theory applicable to transcription factors with any number of DNA
binding domains, we used a different definition for repressor copy number than
has been used previously. We define the LacI copy number as the average number
of repressor dimers per cell whereas in \textcite{Garcia2011}, the copy number is
defined as the average number of repressor tetramers in each cell. To motivate
this decision, we consider the fact that the LacI repressor molecule exists as a
tetramer in \textit{E. coli} \citep{Lewis1996} in which a single DNA binding
domain is formed from dimerization of LacI proteins, so that wild-type LacI
might be described as dimer of dimers. Since each dimer is allosterically
independent (i.e. either dimer can be allosterically active or inactive,
independent of the configuration of the other dimer) \citep{Daber2009}, a single
LacI tetramer can be treated as two functional repressors. Therefore, we have
simply multiplied the number of repressors reported in \textcite{Garcia2011} by a
factor of two. This factor is included as a keyword argument in the numerous
Python functions used to perform this analysis, as discussed in the code
documentation.

A subset of strains in these experiments were measured using fluorescence
microscopy for validation of the flow cytometry data and results. To aid in the
high-fidelity segmentation of individual cells, the strains were modified to
constitutively express an mCherry fluorophore. This reporter was cloned into a
pZS4*1 backbone \citep{Lutz1997} in which mCherry is driven by the
\textit{lacUV5} promoter. All microscopy and flow cytometry experiments were
performed using these strains.

%%%%%%%%%%%%%%%%%%%%%%%%%%%%%%%%%%%%%%%%%%%%%%%%%%%%%%%%%%%%%%%%%%%%%%%%%%%%%%%
\subsection*{Growth Conditions for Flow Cytometry Measurements}

All measurements were performed with \textit{E. coli} cells grown to
mid-exponential phase in standard M9 minimal media (M9 5X Salts, Sigma-Aldrich
M6030; $2\,\text{mM}$ magnesium sulfate, Mallinckrodt Chemicals 6066-04; $100\,\mu\text{M}$
calcium chloride, Fisher Chemicals C79-500) supplemented with 0.5\% (w/v)
glucose. Briefly, $500\,\mu\text{L}$ cultures of \textit{E. coli} were inoculated into
Lysogeny Broth (LB Miller Powder, BD Medical) from a 50\% glycerol frozen stock
(-80$^\circ$C) and were grown overnight in a $2\,\text{mL}$ 96-deep-well plate sealed with
a breathable nylon cover (Lab Pak - Nitex Nylon, Sefar America Inc. Cat. No.
241205)~with rapid agitation for proper aeration. After approximately $12$ to
$15$ hours, the cultures had reached saturation and were diluted 1000-fold into
a second $2\,\text{mL}$ 96-deep-well plate where each well contained $500\,\mu\text{L}$ of M9
minimal media supplemented with 0.5\% w/v glucose (anhydrous D-Glucose, Macron
Chemicals) and the appropriate concentration of IPTG (Isopropyl $\beta$-D-1
thiogalactopyranoside Dioxane Free, Research Products International). These were
sealed with a breathable cover and were allowed to grow for approximately eight
hours. Cells were then diluted ten-fold into a round-bottom 96-well plate
(Corning Cat. No. 3365) containing $90\,\mu\text{L}$ of M9 minimal media
supplemented with 0.5\% w/v glucose along with the corresponding IPTG
concentrations. For each IPTG concentration, a stock of 100-fold concentrated
IPTG in double distilled water was prepared and partitioned into
$100\,\mu\text{L}$ aliquots. The same parent stock was used for all experiments
described in this work.

%%%%%%%%%%%%%%%%%%%%%%%%%%%%%%%%%%%%%%%%%%%%%%%%%%%%%%%%%%%%%%%%%%%%%%%%%%%%%%%
\subsection*{Flow Cytometry}

Unless explicitly mentioned, all fold-change measurements were collected on a
Miltenyi Biotec MACSquant Analyzer 10 Flow Cytometer graciously provided by the
Pamela Bj\"{o}rkman lab at Caltech. Detailed information regarding the voltage
settings of the photo-multiplier detectors can be found in Appendix Table
\ref{table_instrument_param}. Prior to each day's experiments, the analyzer was
calibrated using MACSQuant Calibration Beads (Cat. No. 130-093-607) such that
day-to-day experiments would be comparable. All YFP fluorescence measurements
were collected via $488\,\text{nm}$ laser excitation coupled with a
525/$50\,\text{nm}$ emission filter. Unless otherwise specified, all
measurements were taken over the course of two to three hours using automated
sampling from a 96-well plate kept at approximately $4^\circ \, \hbox{-} \,
10^\circ$C on a MACS Chill 96 Rack (Cat. No. 130-094-459). Cells were diluted to
a final concentration of approximately $4\times 10^{4}$ cells per $\mu\text{L}$
which corresponded to a flow rate of 2,000-6,000 measurements per second, and
acquisition for each well was halted after 100,000 events were detected. Once
completed, the data were extracted and immediately processed using the following
methods.

%%%%%%%%%%%%%%%%%%%%%%%%%%%%%%%%%%%%%%%%%%%%%%%%%%%%%%%%%%%%%%%%%%%%%%%%%%%%%%%
\subsection*{Unsupervised Gating of Flow Cytometry Data}

Flow cytometry data will frequently include a number of spurious events or other
undesirable data points such as cell doublets and debris. The process of
restricting the collected data set to those data determined to be ``real'' is
commonly referred to as gating. These gates are typically drawn manually
\citep{Maecker2005} and restrict the data set to those points which display a
high degree of linear correlation between their forward-scatter (FSC) and
side-scatter (SSC). The development of unbiased and unsupervised methods of
drawing these gates is an active area of research \citep{Lo2008,
Aghaeepour2013}. For our purposes, we assume that the fluorescence level of the
population should be log-normally distributed about some mean value. With this
assumption in place, we developed a method that allows us to restrict the data
used to compute the mean fluorescence intensity of the population to the
smallest two-dimensional region of the $\log(\mathrm{FSC})$ vs.
$\log(\mathrm{SSC})$ space in which 40\% of the data is found. This was
performed by fitting a bivariate Gaussian distribution and restricting the data
used for calculation to those that reside within the 40th percentile. This
procedure is described in more detail in the supplementary information as well
as in a Jupyter notebook located in this paper's
\href{https://rpgroup-pboc.github.io/mwc_induction/code/notebooks/unsupervised_gating.html}{Github
repository}.

%%%%%%%%%%%%%%%%%%%%%%%%%%%%%%%%%%%%%%%%%%%%%%%%%%%%%%%%%%%%%%%%%%%%%%%%%%%%%%%
\subsection*{Experimental Determination of Fold-Change}

For each strain and IPTG concentration, the fold-change in gene expression was
calculated by taking the ratio of the population mean YFP expression in the
presence of LacI repressor to that of the population mean in the absence of LacI
repressor. However, the measured fluorescence intensity of each cell also
includes the autofluorescence contributed by the weak excitation of the myriad
protein and small molecules within the cell. To correct for this background, we
computed the fold change as
\begin{equation}
 \text{fold-change} = \frac{\langle I_{R > 0} \rangle - \langle I_\text{auto}\rangle}{\langle I_{R = 0} \rangle - \langle I_\text{auto}\rangle},
\end{equation}
where $\langle I_{R > 0}\rangle$ is the average cell YFP intensity in the
presence of repressor, $\langle I_{R = 0}\rangle$ is the average cell YFP
intensity in the absence of repressor, and $\langle I_\text{auto} \rangle$ is
the average cell autofluorescence intensity, as measured from cells that lack
the \textit{lac}-YFP construct.

%%%%%%%%%%%%%%%%%%%%%%%%%%%%%%%%%%%%%%%%%%%%%%%%%%%%%%%%%%%%%%%%%%%%%%%%%%%%%%%
\subsection*{Bayesian Parameter Estimation}

In this work, we determine the the most likely parameter values for the inducer
dissociation constants $K_A$ and $K_I$ of the active and inactive state,
respectively, using Bayesian methods. We compute the probability distribution of
the value of each parameter given the data $D$, which by Bayes' theorem is given
by
\begin{equation}\label{bayes_theorem}
	P(K_A, K_I \mid D) = \frac{P(D \mid K_A, K_I)P(K_A, K_I)}{P(D)},
\end{equation}
where $D$ is all the data composed of independent variables (repressor copy
number $R$, repressor-DNA binding energy $\Delta\varepsilon_{RA}$, and inducer
concentration $c$) and one dependent variable (experimental fold-change). $P(D
\mid K_A, K_I)$ is the likelihood of having observed the data given the
parameter values for the dissociation constants, $P(K_A, K_I)$ contains all the
prior information on these parameters, and $P(D)$ serves as a normalization
constant, which we can ignore in our parameter estimation.
\eref[eq_fold_change_full] assumes a deterministic relationship between the
parameters and the data, so in order to construct a probabilistic relationship
as required by \eref[bayes_theorem], we assume that the experimental fold-change
for the $i^\text{th}$ datum given the parameters is of the form
\begin{equation}
\foldchange _{\exp}^{(i)} = \left( 1 + \frac{\left(1 +
\frac{c^{(i)}}{K_A}\right)^2}{\left( 1 + \frac{c^{(i)}}{K_A}\right)^2 +
e^{-\beta \Delta \varepsilon_{AI}} \left(1 + \frac{c^{(i)}}{K_I} \right)^2} \frac{R^{(i)}}{N_{NS}} e^{-\beta
\Delta \varepsilon_{RA}^{(i)}}\right)^{-1} + \epsilon^{(i)},
\label{eq_fold_change_exp}
\end{equation}
where $\epsilon^{(i)}$ represents the departure from the deterministic
theoretical prediction for the $i^\text{th}$ data point. If we assume that these
$\epsilon^{(i)}$ errors are normally distributed with mean zero and standard
deviation $\sigma$, the likelihood of the data given the parameters is of the
form
\begin{equation} \label{eq_likelihood}
P(D \vert K_A, K_I, \sigma) =
\frac{1}{(2\pi\sigma^2)^{\frac{n}{2}}}\prod\limits_{i=1}^n \exp
\left[-\frac{(\foldchange^{(i)}_{\exp} - \foldchange(K_A, K_I, R^{(i)},
	\Delta\varepsilon_{RA}^{(i)}, c^{(i)}))^2}{2\sigma^2}\right],
\end{equation}
where $\foldchange^{(i)}_{\text{exp}}$ is the experimental fold-change and
$\foldchange(\,\cdots)$ is the theoretical prediction. The product
$\prod_{i=1}^n$ captures the assumption that the $n$ data points are
independent. Note that the likelihood and prior terms now include the extra
unknown parameter $\sigma$. In applying \eref[eq_likelihood], a choice of $K_A$
and $K_I$ that provides better agreement between theoretical fold-change
predictions and experimental measurements will result in a more probable
likelihood.

Both mathematically and numerically, it is convenient to define $\tilde{k}_A =
-\log \frac{K_A}{1\,\text{M}}$ and $\tilde{k}_I = -\log \frac{K_I}{1\,\text{M}}$
and fit for these parameters on a log scale. Dissociation constants are scale
invariant, so that a change from $10\,\mu\text{M}$ to $1\,\mu\text{M}$ leads to
an equivalent increase in affinity as a change from $1\,\mu\text{M}$ to
$0.1\,\mu\text{M}$. With these definitions we assume for the prior
$P(\tilde{k}_A, \tilde{k}_I, \sigma)$ that all three parameters are independent.
In addition, we assume a uniform distribution for $\tilde{k}_A$ and
$\tilde{k}_I$ and a Jeffreys prior \citep{Sivia2006} for the scale parameter
$\sigma$. This yields the complete prior
\begin{equation}
P(\tilde{k}_A, \tilde{k}_I, \sigma) \equiv \frac{1}{(\tilde{k}_A^{\max} -
\tilde{k}_A^{\min})} \frac{1}{(\tilde{k}_I^{\max} -
\tilde{k}_I^{\min})}\frac{1}{\sigma}.
\end{equation}
These priors are maximally uninformative meaning that they imply no prior
knowledge of the parameter values. We defined the $\tilde{k}_A$ and
$\tilde{k}_A$ ranges uniform on the range of $-7$ to $7$, although we note that
this particular choice does not affect the outcome provided the chosen range is
sufficiently wide.

Putting all these terms together we can now sample from $P(\tilde{k}_A,
\tilde{k}_I, \sigma \mid D)$ using Markov chain Monte Carlo (see
\href{https://rpgroup-pboc.github.io/mwc_induction/code/notebooks/bayesian_parameter_estimation}{GitHub repository}) to compute the most likely parameter as well as the error bars (given by the 95\% credible region) for $K_A$ and $K_I$.

%%%%%%%%%%%%%%%%%%%%%%%%%%%%%%%%%%%%%%%%%%%%%%%%%%%%%%%%%%%%%%%%%%%%%%%%%%%%%%%
\subsection*{Data Curation}

All of the data used in this work as well as all relevant code can be found at
this \href{http://rpgroup-pboc.github.io/mwc_induction}{dedicated website}. Data
were collected, stored, and preserved using the Git version control software in
combination with off-site storage and hosting website GitHub. Code used to
generate all figures and complete all processing step as and analyses are
available on the GitHub repository. Many analysis files are stored as
instructive Jupyter Notebooks. The scientific community is invited to fork our
repositories and open constructive issues on the
\href{https://www.github.com/rpgroup-pboc/mwc_induction}{GitHub repository}.
