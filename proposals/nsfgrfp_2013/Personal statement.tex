\documentclass[12pt]{report}
\usepackage[margin=1in]{geometry}
\usepackage{color}
\usepackage{microtype}
\begin{document}
\begin{center}
{\bf Personal Statement, Relevant Background, and Future Goals}
\end{center}
\hspace{5mm} For most of my undergraduate career I had the opportunity to work in a biochemistry research lab that focuses on the genetic regulation, bioenergetics, and biochemical assembly of the bacterial flagellar motor in {\it E. coli} and {\it Salmonella}. I am indebted to Dr. David F. Blair, University of Utah Professor of Biology, for providing that lab experience as well as encouraging and stimulating my interest in and dedication to scientific research. In my three and a half years of work in the Blair Lab, I not only learned the laboratory skills needed to pursue a professional career in science, but developed a deep interest in the physical and chemical principles governing life at the molecular and cellular levels. My undergraduate research experience allowed me to apply what I learned through classwork to real research questions and gave me first-hand experience analyzing a problem and developing an approach to answer the research question.
\\
\indent My research experiences have also made me think strategically about how to pursue a career in this field. Realizing that research often benefits from a multidisciplinary approach, I chose a dual major in Biology and Chemistry with a minor in Physics. This broader background has helped me see how different approaches can be used to solve a research problem. On a more specific level, I have learned a number of skills useful in molecular biological research including targeted mutagenesis, construction of genetic deletions, protein purification, cross-linking, immunoblotting, physiological assays of cell motility, and the use of basic bioinformatic tools. 
\\
\indent The goal of my research was to characterize FlhE, a flagellar protein of previously unknown function. Through my research, it was found that FlhE plays an important role in controlling bacterial motility and deletion of chromosomal {\it flhE} results in aberrant morphologies and poor motility. Coprecipitation assays performed in the lab have shown an interaction between FlhE and a core component of the flagellar basal body. This is the first known interacting partner of FlhE. We developed a model in which FlhE acts as a molecular tether between the flagellar motor and the peptidoglycan of the cell wall during flagellar biosynthesis. In the absence of FlhE, the assembly of the flagellum sometimes goes awry and results in the extracellular components being secreted into the periplasmic space, distorting cellular morphology and hindering motility. The final details of this model are currently being tested by others in the lab. I have had the opportunity to present both oral and poster presentations of my research results at research symposia attended by undergraduate and graduate students, as well as professors. I enjoyed those opportunities to discuss my research with fellow students and with faculty who had been my instructors. In addition to research and presentation experience, I have served as Teaching Assistant for Introductory Biology (Dr. Tanya Vickers), Biochemistry and Molecular Biology Laboratory Techniques (Dr. Rosemary Gray), Genetics (twice with Dr. Sandy Parkinson), and Biochemistry Lab (Dr. David Goldenberg) all at the University of Utah and am slated to TA for the Cell Biology Laboratory at Caltech during the third term. Students felt that my sessions were successful and I really enjoyed helping fellow students learn about things I felt passionately about. In addition to assisting fellow researchers in the Blair Lab and conducting my own honor's thesis research, I have trained several undergraduate students when they first started working in the lab.
\newpage

\indent During the 2010-2011 academic year, I was selected as one of twelve Crocker Science Scholars and lived in the Crocker Science House, an honors dormitory designated for twelve science majors of all disciplines. Living in an environment plentiful in scientific discussion and intellectual stimulation allowed me to approach both my academics and research from many different viewpoints. During my residence, we were privileged to have dinners with distinguished guests, such as the Dean of the College of Science and guest lecturers for university sponsored colloquia.  

\indent Within the field of biochemistry, I am specifically interested in both the structural elucidation of macromolecules and the physical principles which underly their behavior. I am currently serving a ten week rotation in the lab of Dr. William Clemons working on structural biology of the protein BacA, a membrane bound enzyme in the peptidoglycan biosynthetic pathway of gram negative bacteria. This research experience has been invaluable as the practice of protein purification and crystallization requires me to learn new skills as well as draw upon my previous research experience to chip away at the research problem. I plan on attaining my PhD working in a lab that utilizes the principles of physics, biology, and chemistry to understand the processes that govern life at the molecular level and am looking forward to learning more in the next laboratory rotation. I am interested in conducting basic scientific research, however, I appreciate that the application of such basic research drives advances in medicine, energy, and technology. I intend to focus on basic research that can help fuel those developments. 

\indent I realize that graduate school is the beginning of my scientific career and now is the time to acquire the skills I need for the career I desire. I can attribute my desire to become a research scientist to the professors I had as an undergraduate. The best professors inspired me because they were involved in both research and education. Being able to draw examples in class from their own research helped me understand the material at a deeper level. In my career, I wish to be both a researcher and an educator, teaching others the beauty of chemistry and biology and how that can be used to explain and understand the world around us. During my graduate career, I will make it a priority to focus on becoming an effective educator as well as a researcher. 
\\
\indent In addition to biological research, I wish to become involved in enhancing the impact science has on public policy. Quality scientific education is paramount to societal success and progress. I grew up in a rural ranching community in eastern Utah. While the surrounding areas were filled with National Parks, National Forests, spectacular canyons, and world-famous rivers, growing up in a small town of 300 people did not come with the perk of quality public education. For example, a majority of the science teachers in the school system contradicted both accepted science and state educational standards by pushing concepts such as ``there is no evidence for evolution" and that ``vaccinations are poisonous." Luckily, my parents served as a constant source of inspiration and motivation to pursue my academic goals. Both of my parent's attained BS Degrees in Geology and my father obtained a PhD in paleontology at Colombia University. While working as Park Paleontologist at Dinosaur National Monument, my father would take me along on dinosaur excavations, geological surveys, and explorations of the national park. Having parents immersed in the scientific world allowed me to attend international conferences and professional meetings, such as the Society of Vertebrate Paleontology yearly conference. As a research scientist, I wish to use my resources and experience to increase the role science plays in the classroom, especially in rural areas where science education is at its weakest.  Increasing the quality and availability of science to those in rural areas of the world is vitally important. 
\\
\indent The experiences afforded to me as an undergraduate have been transformative. Being involved in research as an undergraduate made me passionate about science both as a profession and a lifestyle. As I begin my first year of graduate study, I look forward to the time I will have to dedicate myself not only to research, but to cultivating an ability to teach. Even though I have spent the past four years immersed in science, I realize that I am now just at the beginning of my career. Becoming a researcher, educator, and scientific activist is how I wish to spend the rest of my life and I can think of nothing I would enjoy more. 
\end{document}