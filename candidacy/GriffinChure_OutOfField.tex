%%%%%%%%%%%%%%%%%%%%%%%%%%%%%%%%%%%%%%%%%%%%%%%%%%%%%%%%%%%%%%%%%%%%%%%%%%%%%%%
%%%%%%%%%%%%Preamble%%%%%%%%%%%%%%%%%%%%%%%%%%%%%%%%%%%%%%%%
\documentclass[letterpaper, 12pt]{article}                 %
\usepackage[margin=1in]{geometry} 	                   %
\usepackage{microtype}                                     %
\usepackage{amsmath}                                       %
\usepackage{graphicx}                                      %
\usepackage{xcolor}                                        %
\usepackage{verbatim}                                      %
\usepackage{fancyhdr}                                      %
\pagestyle{fancy} 				           %
	\lhead{Griffin Chure} 				   %
	\rhead{Physical Basis of Genomic Promiscuity}      %
%%%%%%%%%%%%%%%%%%%%%%%%%%%%%%%%%%%%%%%%%%%%%%%%%%%%%%%%%%%%

%%%%%%%%%%%%%Bibliography Bullshit%%%%%%%%%%%%%%%%%%%%%%%%%%%%%%%%%%%%%%%%%%%%%%%%%%%%
\usepackage[super, comma]{natbib} %Sets bib dependency using natbib/bibtex
 \setlength{\bibsep}{0.0pt} %Reduces spacing between bibliography
 \renewcommand*{\bibsection}{} %Removes bibliography title.
 \renewcommand*{\bibfont}{\footnotesize} %Makes bib fontsize smaller.
%%%%%%%%%%%%%%%%%%%%%%%%%%%%%%%%%%%%%%%%%%%%%%%%%%%%%%%%%%%%%%%%%%%%%%%%%%%%%%%%%%%%%%

\rhead{Original Research Proposal}
%%%%%%%%%%%%%%%%%%%%%%%%%%%%%%%%%%%%%%%%%%%%%%%%%%%%%%%%%%%%%%%%%%%%%%%%%%%%%%%

%%%%%%%%%%%%%TITLE INFORMATION%%%%%%%%%%%%%%%%%%%%%%%%%%%%%%%%%%%%%%%%%%%%%%%%%
\title{From Free-Living to Endosymbiont: Transcriptional Analysis of
	\textit{Rhizobium} Differentiation 
in Legume Root Nodule Development}
\author{Griffin Chure -- Original Research Proposal -- Rob Phillips Group\\
	\small \textit{California Institute of Technology - Biochemistry and
	Molecular Biophysics}}
\date{\today}
%%%%%%%%%%%%%%%%DOCUMENT%%%%%%%%%%%%%%%%%%%%%%%%%%%%%%%%%%%%%%%%%%%%%%%%%%%%%%%
\begin{document}

\maketitle

\abstract{All plants rely on the metabolic action of soil microbes
to convert atmospheric dinitrogen into fixed ammonia. In return, some species of
plants (primarily legumes) house these microbes in complicated structures
known as root nodules. Within the root nodules, specific species of bacteria are
internalized into the legume cytoplasm and form stable pseudo-organelles known
as symbiosomes. The development of root nodules relies on extensive
communication between the nitrogen fixing microbe and the plant root cells to
safely and efficiently form these dynamic metabolic structures. This
communication between domains of life results in a large amount of
transcriptional rearrangements in the root tissues of the plant and has been
well studied through genetic and sequencing methods.  The extent of
transcriptional reprogramming in the infecting microbe, however, is unknown.
Through RNA-sequencing, transcriptomics, and genetic assays, this research will
reveal how the transcription profile of the symbiotic nitrogen-fixing bacterium
\textit{Rhizobium} changes over the course of infection and symbiosome
differentiation. This will be performed in determinate and indeterminate nodule
forming legumes (\textit{Lotus japonicus} and \textit{Medicago truncatula}
respectively), providing insight on microbial reactions to differentiation
signals from the host plant cell.} 

\section*{Introduction} 
Second only to photosynthesis, nitrogen fixation is the most important
biological process on the planet. All forms of life require nitrogen to survive.
However, the machinery necessary for the energetically expensive conversion of
molecular nitrogen into a biologically active ammonia, is limited to a subset of
soil and marine dwelling microbes (diazotrophs). Although unable to fix nitrogen
directly, legumes (such as the pea, lentil, and clover) have evolved a symbiotic
relationship with the surrounding soil diazotrophs to form
nitrogen-fixing root nodules. The importance of legumes (and crop rotation in
general) to soil fertility has been known for most of human history and its role
in the nitrogen cycle has been the subject of extensive study for the past 150
years (reviewed in \citet{Hirsch:2009to}). The identity  of the
"bumps" growing along the roots of leguminous plants were described in the late
17th century and were believed to be due to parasitic insect activity, serving
no important function to the plant. Although it took over two centuries, 
Hellriegel and Wilfarth \cite{Hellriegel:1888ug} proved that the legume root nodules were
directly responsible for fixing dinitrogen into a biologically accessible form.
Furthermore, they were able to show that this process was not due to the plant
cells directly, but was a product of some other organism living within the
nodules. A little over a decade later, Martinus Beijerinck was able to
isolate and classify the nitrogen-fixing nodule dwellers as bacterial members of
the genus \textit{Rhizobium} \cite{Beijerinck:1901wz}.  

In the century and half after its discovery, we have learned much about the
evolution, ecology, development, and biochemistry of this fascinating
symbiosis between different domains of life (described below). While found among
many species of legumes, nodules typically take one of two forms. Determinate
nodules halt meristematic activity, meaning that lateral growth is halted. Root
cells of the cortex begin to grow and expand in size, resulting in a spherical
nodule.  Indeterminate nodules retain meristematic activity and can continue to
grow laterally into the soil sometimes forming large, branched structures. The
continued growth results in non-homogeneous nitrogen fixation along the length
of the developing root nodule \cite{Crespi:2000dw} caused by continual infection
by new nitrogen-fixing microbes and senescence of old symbiosomes. While the
large-scale morphological differences between the two nodule modalities are
striking, the basic architecture of the symbiotic interface is relatively
similar, although there are differences in the regulation of bacterial growth
within the nodule between determinate and indeterminate types. 

The functional symbiotic relationship forms upon entry of the nitrogen-fixing
bacterium within a plant-derived plasma membrane vesicle into the cytosol of the
plant root cell. The formation of this structure, known as the symbiosome,
involves drastic changes in morphology and cell differentiation for both the
plant and bacterial cells requiring mechanisms of inter-domain signaling. The
molecular events within the nodulating root cells have been extensively studied
and are fairly well understood. However, our knowledge of their relationship to
differentiation of the free-living bacterium to a nitrogen-fixing endosymbiont
is extremely limited. Through RNA sequencing of the infecting bacteria throughout nodule
development coupled with targeted  mutagenesis, I aim to examine the extent and
timing of development the bacterium undergoes on its journey to an endosymbiont.
The following sections provide an overview of the infection process of
free-living bacteria and development into a functional symbiosome as well as a
suite of experiments which may reveal previously
unidentified plant-microbe signaling events.

\subsection*{Development of the symbiosome} 
\indent The formation of nitrogen-fixing root nodules is a two-step developmental
process requiring organogenesis in the legume root and infection by a
nitrogen-fixing bacterium (an overview is given in Fig. \ref{fig:overview}). While each 
process can be induced genetically,  the two must occur at the same
position and time in order for a functional root nodule to develop
\cite{Oldroyd:2008hd}.  These processes are highly coordinated between the
nitrogen-fixing bacteria and the root epidermal cells, requiring extensive
communication and transcriptional rewiring. In general, plants exude an
impressive amount of small molecules and proteins into the soil which support
the surrounding microbial communities,
known as the rhizosphere \cite{Perret:2000ue}. By providing radio-labeled
$^{14}$CO$_2$, \citet{Helal:1988tq} were able to show that an astounding 19\% of
the photosynthetically fixed carbon was released back into the soil. Chemical
analysis of this secreted material (exudate) revealed the presence of 
polysaccharides, antimicrobial agents, and a diverse array of
flavanoids. From a symbiotic perspective, flavanoids are the most
important constituents of the exudate. They are recognized by nearby bacteria
and serve as a chemoattractant \cite{Zuanazzi:1998vk, Hirsch:1992vo} driving
navigation of the bacteria to the nitrogen-starved roots. In
addition, their reception  
stimulates the synthesis and secretion of complex
polysaccharide-decorated lipids known as nodulation factors (NF) from
the microbes which are critical to initiate symbiosis. The type of flavanoid
and decorations of the NF play important roles in species specificity of the
symbiosis \cite{Perret:2000ue, Oldroyd:2011ej, Denarie:1996ts} and are suspected
to have evolved from an early symbiotic relationship between plants and
mycorrhizal fungi \cite{Parniske:2008jx}.

Recognition of the NF by transmembrane receptors on the surface of
the root epidermis sparks a calcium-dependent signaling cascade which penetrates
into the center of the root called the cortex (reviewed in
\citet{Oldroyd:2004dv, Oldroyd:2008hd}). The identity of this diffusible signal
remains unknown, however many other components of this signaling cascade have
been identified. The NF is sufficient for
nodule formation as application of the NF in the absence of microbes stimulates
organogenesis, although the resulting nodule will be incapable of nitrogen
fixation \cite{Truchet:1991tu}. 

For a fully-functional nodule to develop, the nitrogen-fixing bacteria 
must penetrate the root epidermis and ultimately invade the cells of the cortex.
As plant cells are protected by hydrophobic plasma membranes and a rigid
cellulose cell wall, there must be a large mechanical disruption of the root
epidermal cells. Nitrogen-fixing bacteria have been shown to enter the root
cortex through cracks in the epidermis, by slipping into the interstitial space
between adjacent epidermal cells, or (most commonly) through invaginations
within the root hairs -- tubular extensions of root epidermal cells that cover the
root
epidermis \cite{Oldroyd:2011ej, Gage:2004ee, Grierson:2002jf}. Upon the
recognition of NF via receptors that decorate root hair surface, a
large cytoskeletal rearrangement occurs causing the root hair to curl and fold upon
itself. The surrounding bacteria become trapped within this fold
forming what is known as an infection pocket \cite{Geurts:2005jx}. The trapped
microbes continue to divide in this pocket, forming small colonies known as
infection foci. However, continued division is not sufficient for
progressive infection into the root cortex. It is likely that a release of
cell-wall degrading enzymes from both the microbes of the infection foci and the
root epidermal cells themselves weaken the cell wall surrounding the infection
foci and
promote invagination of the plasma membrane and cell wall into a structure known as the
infection tube \cite{Ridge:1985wo}. Delivery of membrane vesicles to the tip of
the infection tube allows continued invagination through the epidermal cells
towards the root cortex. The
infection tube grows with an impressively high degree of directionality,
navigating its way to the next
layer of cells ultimately penetrating into the root cortex. This directionality
is achieved by a direct connection of the nucleus to the tip of the infection tube
via cytoskeletal fibers, most likely actin filaments
\cite{MonahanGiovanelli:2006ki}. Movement of the nucleus along actin
cables pulls the growing infection tube towards the adjacent root cortex cells.
Once the tip of the infection tube approaches the medial wall of the root hair
cell, the nucleus of the adjacent cell aligns itself with the incoming infection
tube and continues the direction to
the next cell \cite{Gage:2004ee}. This process is repeated with the infection
tube periodically branching, infecting a large number of root cortical cells.
The amount of inter-cellular signaling to accommodate the passage of the
infection tube into the root cortex is astounding and many of the signaling
partners are unknown \cite{Oldroyd:2011ej}.

\begin{figure}
	\centerline{\includegraphics[width=\textwidth]{figs/nodule_overview.eps}}
	\caption{Microbial infection and nodule formation. The top panel
	overviews the major events in which bacteria infect root epithelium and
form symbiosome within nodules. Plant cells have been omitted for clarity. Blue
and red dots represent plant-secreted flavanoids and bacterium-secreted
nodulation factors (NF) respectively. The bottom panel shows representative
images of \textit{L. japonicus} determinate nodules (image courtesy of
\texttt{seedquest.com}), root hair and infection thread labeled with red and green
respectively (image adapted from \citet{Meckfessel:2012df}), and \textit{L. japonicus}
symbiosomes containing bacteroids (image adapted from
\citet{VandeVelde:2010cn}).}
\label{fig:overview}
\end{figure}

As the infection tube continues invagination, the bacteria of the infection foci
continue to divide, forcing themselves into the cavity of the infection
tube forming an infection thread. While these microbes are still topologically
outside of the plant cells, they must combat a large array of metabolic stresses.
It has been shown that  within the infection thread, 
\textit{Sinorhizobium meliloti} induces a suite of catalase genes typically
activated by oxidative stress \cite{Jamet:2003us}. Mutation of these genes
resulted in poor nodulation and aberrant infection physiology, suggesting that
oxidative stress is experienced during the entire infection process. Many plants
synthesize and secrete a variety of antimicrobial agents into the soil, such as dicholorethane and
phytoalexins. This process is important in preventing parasitism
of the root, but continues even during infection thread progression
\cite{Karr:1992uk} serving as another hurdle which the invading bacteria must overcome.
Some symbiotic species express various resistance cassettes specific to these
anitimicrobial agents in the free-living state and likely in the infection
thread \cite{Ramachandran:2011iq}. While these serve as specific examples, the
infecting bacteria face other adverse conditions within in the infection thread
microenvironment, allowing the legume to select for the most competitive
symbiotes. 


Once the infection thread reaches the inner cortical cells of the legume root,
the progression of the infection tube ceases. However, the cells within the
infection thread continue to proliferate until they are released into the
cytosol of the root cortical cell within an infection tube vesicle. This
plant-derived vesicle containing one to several bacteria forms the
symbiosome pseudo-organelle \cite{Oldroyd:2011ej}. At this stage, numerous plant
proteins are targeted to the symbiosomal membrane (SM) initiating
differentiation of the bacterium into a nitrogen-fixing form named the bacteroid.  
Symbiosomes of determinate nodule forming legumes, such as \textit{Lotus japonicus}, 
often contain two or more bacteroids similar in size to their
free-living counter parts. In general, the determinate-nodule bacteroid cell
cycle is unperturbed, and division continues uninhibited. This means that the
cortical cell can be packed with thousands of symbiosomes each containing a
handful of bacteroids \cite{Ott:2009vx}. In stark contrast, 
indeterminate nodule forming legumes, such as \textit{Medicago truncatula},
explicitly control the bacteroid life cycle resulting in the symbiosome 
behaving more similarly to a cellular organelle. Bacteroids in these legumes
appear to be larger and swollen when compared to their free-living state 
\cite{Oono:2010hg} and have compromised membranes \cite{Mergaert:2006wi}. \textit{M. truncatula} produces a large abundance of
cysteine-rich small peptides which control terminal
differentiation of bacteroids and halt cell division \cite{VandeVelde:2010cn,
Farkas:2014dv}. Furthermore, these bacteroids are forced into endoreduplication
where the chromosome is replicated to 24 copies, more than double the mean copy
number in the free-living state \cite{Mergaert:2006wi}.

This gives an overview of invasion and differentiation of free-living microbes to
differentiated, nitrogen-fixing bacteroids within plant symbiosome. Many studies
regarding the plant hormone involvement for organogenesis of nodule tissue
(reviewed in \citet{Oldroyd:2004dv} and \citet{Ryu:2012dt}) have focused
primarily on the interactions between the root cells and the bacterial
symbionts. The plant hormone signaling pathways and consequences directing
nodule organogenesis have been studied extensively and are relatively well
understood. However, there is still much to be learned about the interactions
between the bacteria and the root cells at all stages of development.  

\subsection*{Symbiosome maintenance}
Once differentiated, the bacteroids rely entirely on the host
cell for all nutrients necessary for nitrogen fixation and survival. To
complicate matters, the semi-permeable SM prevents the free diffusion of
material to the bacteroid making transport machinery within the SM a necessity
\cite{Smith:2014jg}. Upon release from the infection thread, proliferation of
the bacteroids requires rapid expansion of the SM, requiring the
synthesis and targeting of lipids and proteins to the SM \cite{Smith:2014jg}. In
the completely differentiated state, the surface area of the SM can reach 100
fold that of the plasma membrane \cite{Roth:1989vz}. Proteomic analysis of
proteins associated with the SM revealed a common N-terminal signal sequence
\cite{Meckfessel:2012df, Hohnjec:2009uk} suggesting the existence of a
symbiosome-specific trafficking pathway. The molecular details of this transport
pathway remain unknown, however it is likely that other transport pathways exist as
proteins lacking the N-terminal signal sequence are still successfully targeted
to the symbiosome \cite{Catalano:2004du}.  For protein targeting and transport
at the SM to be completely understood, more work is needed to identify and
characterize the symbiosome-specific secretion pathway(s). The protein content
of the SM varies over the symbiosome life cycle to accommodate the changing
metabolic requirements of the bacteroid \cite{Whitehead:1997wu}. This strongly
suggests a level of communication between the bacteroid and the plant cell
directly. What these signals are or how they are received by the plant nucleus
is still entirely unknown and warrants further investigation.  

The reduction of N$_2$ is an expensive endeavor
requiring large amounts of ATP \cite{Halbleib:2000tx}. It is unsurprising that
the primary nutrient transported across the SM is fixed carbon produced via
photosynthesis in the leaves and transported to roots through the phloem.
Biochemical assays with purified and metabolically active symbiosomes revealed
a putative transporter of succinate and malate \cite{Udvardi:1988wm} however,
even 26 years after its discovery, the identity of this transporter has not been elucidated.
The other major protein constituents of the SM are ammonium
transporters. The product of bacteroid nitrogen fixation is NH$_3$, which is
secreted out of the cell into the symbiosomal space (SS) where it is protonated
to NH$_4^+$ and transported into the cortical cytoplasm. To ensure that the
fixed nitrogen is not squandered and used by the bacteroid, the host plant cell
represses the ammonium uptake machinery during differentiation 
into a bacteroid \cite{Howitt:1986wq}. As in the case of the malate/succinate
transporter, the precise identity of the ammonium transporter is still unknown,
although some biochemical and mutational studies suggest that NOD26, an
aquaporin, is the likely culprit \cite{Hwang:2010cs}.

While there has been much work characterizing the role of the SM, there has been
little interest the function of the SS. Proteins and small molecules produced by
both the plant cell and the bacteroid populate this volume creating a unique
opportunity for cooperation between domains of life at the protein level
\cite{Simonsen:1999hj}. Proteomic analysis of the SS revealed that the
population of non-metabolic/maintenance proteins is large relative to metabolic
enzymes suggesting that the SS may serve a role beyond carbon/nitrogen
transport. Many of the same proteins have been found in numerous proteomic studies in a
variety of legume-\textit{Rhizobium} symbioses (reviewed in \citet{Emerich:2014fo}).
Unfortunately, these studies do not reveal the connections between these
components. A more detailed biochemical characterization is needed to reveal any
hidden functions of the SS.

The transition from root cells and free-living bacteria to a fully-developed
nodule requires drastic genetic and morphological reorganization for both
parties. Even when differentiation is complete, cooperation between the plant
cell and the bacteroid is needed to ensure that the symbiosome successfully
converts dinitrogen into a physiologically relevant form. As described previously,
many of the genetic details of development in the plant cells has been
determined, characterized, and manipulated, but much remains unknown regarding
the regulatory rewiring of the bacteria in transition to the bacteroid symbiont.


\section*{Experimental approach} 
\subsection*{Profiling the \textit{Rhizobium} transcriptome throughout nodulation}
Many studies have used high-throughput methods to
analyze transcriptional reprogramming in nodulating plant tissues and between
free-living and differentiated bacteroids \cite{Cabeza:2014ht, Peng:2014dz,
Moreau:2011en, Breakspear:2015ek, Becker:2004ts}. However, there have been no
studies examining the transcriptome of the infecting microbes over the
development of the root nodule, leaving the dynamics or intermediate states of
development uncharacterized. The initiation of infection via release of
cell-wall degrading enzymes, the progression of the infection thread deep
into the root tissue, and the release of plasma-membrane wrapped bacteria into
root cortical cells suggests that signaling does not occur only within the plant
cells. Understanding how bacterial gene expression changes over
time will shed light on the complete process of bacteroid differentiation.

Most transcriptomics are performed either by microarray analysis
\cite{Rensink:2005da, Guttikonda:2010hq} or RNA sequencing (RNA-seq) \cite{Martin:2013hg,
Wang:2009di}. Briefly, in a microarray analysis the RNA of interest is isolated
reverse-transcribed into cDNA fragments. The cDNA fragments are subsequently
labeled with specific fluorophores and are washed over a microfluidic chip on
which oligonucleotides corresponding to the organism of interest are conjugated
in a known pattern. The chip with bound cDNA is then imaged and analyzed as regions of
fluorescence correspond to conjugated oligonucleotides hybridized with the
fluorescently labeled cDNA, indicating gene expression. By comparing different
expression samples, one can make measurements of transcript abundance relative
to a standard. In contrast, RNA-seq operates via shotgun-sequencing of the
reverse transcription-generated cDNA rather than hybridization. RNA-seq has many
advantages over traditional microarray analysis as it allows absolute determination of
gene transcription, is not limited to known transcripts, and reduces
signal-to-noise as the resulting cDNA is sequenced and cross-hybridization to
unrelated probes is not a concern. RNA-seq has been
popular in analyzing eukaryotic samples as all eukaryotic RNA contains a polyA
tail, making the design of sequencing probes trivial. Prokaryotic RNAs, however,
lack such a barcode, making the sequencing library generation more difficult.
The recent development of differential RNA-seq solves this issue by ligating
polyA RNA barcodes onto partially fragmented purified RNA, allowing the use of
standard sequencing probes \cite{Creecy:2015bk, Sharma:2014ed}.

Using RNA-seq, I propose to monitor the transcriptional reprogramming of
\textit{Rhizobium} sp. NGR234 (hereinafter referred to as \textit{Rhizobium}) from
the soil-dwelling state, through the early stages of nodule development, and
differentiation into functional bacteroids in both determinate and indeterminate
legumes. \textit{Rhizobium} is a fast-growing nitrogen-fixing bacterium capable
of infecting and forming functional symbiosomes in more than 110 genera of
legumes, including determinate and indeterminate nodulating species
\cite{Hussain:2005vu}, making it an attractive bacterium for studying the
developmental differences between them. In addition, the full genome sequence is known
\cite{Schmeisser:2009cp} allowing for a detailed analysis of expression
profiles.  

The experimental procedure is outlined in Fig. \ref{fig:experimental_procedure}. 
Methods for the cultivation and observation of nodule-forming legumes are widely
available and well described in the literature \cite{Journet:2006ws,
Sasaki:2013wj, Diaz:2005ud}. Briefly, the determinate and indeterminate model
legumes \textit{Lotus japonicus} and \textit{Medicago truncatula} respectively
will be grown in soil lacking \textit{Rhizobium} but rich in nitrogen for 1 -
3 weeks. Well-developed seedlings will then be moved to nitrogen depleted soil
for several days leaving the plants starved of nitrogen and primed for
nodulation. The seedling will then be inoculated with freshly grown
\textit{Rhizobium} liquid culture and placed into a paper/plastic growth pouch
(described in \citet{Journet:2006ws}) where root nodule development can be
monitored directly through a plastic viewing screen. The nodulation process from
liquid culture inoculation is rapid with infection threads forming within the
first 24 - 36 hours post inoculation and immature nodules visible within 48
hours post inoculation. Previous \textit{Rhizobium} transcriptional analyses
have only compared the transcriptome of the free-living rhizobium with that of
fully developed nodules (10 - 20 days post inoculation). In this study, I propose to
monitor the bacterial development of nodules over many stages of development
sampling numerous nodule from many individual plants every 24 - 36 hours.
Many seedlings will be inoculated at the same time allowing for the formation of
many nodules on similar time scales.

Several protocols exist for the isolation of symbiosomes and bacteroids from
developing and functional nodules \cite{Wienkoop:2005wu, Kazandjian:2008vu,
Panter:2000wj} along with multiple strategies for the isolation of bacterial RNA
from mixed environmental samples \cite{Wang:2012gj, Stark:2014ea} . Briefly,
after sampling 2 - 10 grams of nodules at each time point, the nodule exterior is
sterilized and flash-frozen with liquid nitrogen. Mechanical disruption via
mortar and pestle breaks apart the plant cells while releasing intact bacteria
and bacteroids which are then purified using differential centrifugation. The
bacteroid is then lysed and the RNA can be isolated using proprietary commercial kits or more
carefully via chemical methods \cite{Sessitsch:2002vq} where it is then
subjected to differential RNA-seq (described above). The returned sequences can
then be mapped to the reference \textit{Rhizobium} genome and a full
transcriptome can be produced.

\begin{figure}
	\centerline{\includegraphics{figs/experimental_procedure.eps}}
	\caption{Experimental outline for nodule transcriptomics. Seedlings
		grown in nitrogen-rich soil are isolated and transferred to
		nitrogen depleted soil priming the roots for nodulation. Liquid
		culture of \textit{Rhizobium} is added to the roots and soil and
		is monitored nodule development. For several days post
		inoculation (dpi), nodule growth is monitored and periodically
		sampled (red circles) from multiple plants (not shown). Root nodules are then
		frozen in liquid nitrogen and mechanically disrupted using a
		mortar and pestle (described in \cite{Vercruysse:2010ji}).
		Bacterial RNA is then purified and reverse transcribed to cDNA
		and barcoded with known sequences. cDNA fragments are then
		sequenced and mapped to the known genomic sequence for
		transcriptome analysis.} 
\label{fig:experimental_procedure}
\end{figure}

\subsection*{Revealing unknown regulatory behavior }
\indent The total number of \textit{Rhizobium} cells occupying the nodule increases as
the infection process progresses. As nodules sampled later in development will
contain more bacteria, it is important to quantify the RNA abundance with
respect to a standard which does not change over the course of development.
Genes imperative to survival, such as 16S rRNA, serve as attractive candidates
for an internal reference. After sequencing the extracted bacteroid
RNA, the relative abundance of each gene can be calculated and compared to the
expression of their free-living counterparts (Fig. \ref{fig:fake_data}). This
will allow identification of large-scale changes in RNA abundance. However,
abundance differences in low-copy number RNAs are more difficult to identify as the impact
of stochastic variation in expression is greater than in high-copy number cases.
As the \textit{Rhizobium} sequence is known and well-annotated, the specific
identity of each over- or underexpressed gene can be identified, providing
insight into how the bacterial symbiont progresses from free-living
\textit{Rhizobium} to  nitrogen-fixing bacteroid.

\begin{figure}
	\centerline{\includegraphics{figs/copy_number_comparison.eps}}
	\caption{Hypothetical data for \textit{Rhizobium} RNA abundance differences over the course of
		nodule development. RNA abundance is computed relative to an RNA
		standard that does not vary over bacteroid development (such as
		16S rRNA) and plotted as a function of RNA abundance of the
		free-living \textit{Rhizobium}.  Arrows indicate examples of
		upregulated and downregulated genes over the course of
		development. Black line represents perfect agreement with the
		free-living expression pattern.
		Stochastic variation in abundance of genes with low expression makes
		determination of statistically significant changes more difficult.}
 \label{fig:fake_data}
\end{figure}
As described in the introduction, the signaling pathways
and transcriptional networks of nodule organogenesis within the plant cells has
been well studied over the past two decades \cite{Oldroyd:2004dv,
Soyano:2014ic}. With this developmental road map, I will be able to compare
changes in bacterial transcription with the molecular events occurring in the
surrounding plant cells to identify putative signaling relationships.
\textit{L. japonicus}, \textit{M. truncatula}, and \textit{Rhizobium} are all
model organisms whose genomes have been sequenced and are easily manipulated
with standard genetic tools. With the transcriptome information in hand, it will
be possible to compare coordinated changes in expression between the bacterium and
root cells to reveal previously undiscovered signaling events.


With this picture of the transcriptional
dynamics of the whole genome, it will be possible to monitor the
activation and/or repression of specific genes known to be important for the
formation of functional nodules.  For example, \textit{Rhizobium} mutants
deficient in nitrogen fixation can still form morphologically normal bacteroids,
although the nodule is not functional \cite{Hirsch:1987ux, Lang:2015wt}. This is
surprising as one would expect the nodulating plant to enter symbiosis only with
bacterial candidates capable of fixation. On the other hand, there are some
small mutations which appear to have great power in directing nodulation.
\textit{BacA}, a non-essential integral membrane protein involved in
peptidoglycan biosynthesis, was shown to be necessary for \textit{Rhizobium} to
infect some, but not all legumes \cite{Karunakaran:2010gr}. Those which the
$\Delta$\textit{bacA} mutants successfully infected developed 
functional nodules, suggesting that its presence is important for the infection
initiation in some legume species. By monitoring the expression of genes with
known importance over time, their role in nodule development may become more
obvious.  Of particular interest is the expression profile of the nitrogen
fixing (\textit{nif}) genes as it is still unknown at what point transcription
begins.  Nitrogenase is irreversibly inactivated by the presence of molecular
oxygen \cite{Gallon:1981uq} and its synthesis is repressed in free-living
\textit{Rhizobium} \cite{Merrick:1995tf}.  At what stage in symbiosome formation
is nitrogenase active? Furthermore, are there any plant-derived signals that
govern its regulation? 

\subsection*{Targeted mutagenesis to characterize developmental stages}
\indent Random mutagenesis screens and transposon insertions have served as an
incredibly fruitful strategy to dissect the information transduction pathways
which dictate organismal development \cite{Sikora:2011fk, Flibotte:2010fc}. Such
strategies have been applied to a number of model legumes and, in conjunction
with transcriptomics, have helped reveal the large number of players that
coordinate to generate nitrogen-fixing nodules as well as many of their
interactions. There has been work on the random mutagenesis of the bacterial
symbionts, but the conclusions regarding bacteroid development have been limited
and less complete. 

Using the expression profiles generated from the above experiments, I propose to
use site-directed mutagenesis  (coupled with modern microscopy techniques and
acetylene reduction assays to quantify nitrogen fixation \cite{Vessey:1994vc}) on a
myriad of up and downregulated \textit{Rhizobium} genes at various stages of
development that may trap the developmental state and provide insight into the
involvement of the bacterium on overall nodule morphology and function (Fig.
\ref{fig:microscopy}). There has been evidence that occasionally, seemingly
innocuous mutations in the bacterial symbiont can dramatically alter nodule
morphology. \textit{Rhizobium} mutants unable to synthesize the tryptophan
precursor molecule anthranallic acid demonstrate extended infection threads and
elongated, defective nodules \cite{Barsomian:1991uc}, a phenotype that could not
be predicted \textit{a priori}. There are many other examples of surprising
nodule phenotypes caused by mutations in the bacterial symbiont although many of
the specific function of the mutated genes in terms of symbiosis remain unknown.
Having the knowledge of when specific genes are
turned on and off will allow for a much more careful analysis of the interplay
between changes in the bacterium and maturation of the root nodule.

\begin{figure}[!h]
	\centerline{\includegraphics{figs/microscopy.eps}}
	\caption{Trapping developmental states through targeted mutagenesis of
		\textit{Rhizobium}. The top panel provides an overview of the
		experimental procedure. The gene expression profile of
		\textit{Rhizobium} at a specific time point shows the relative
		up/downregulation of a suite of genes. Mutations altering these
		genes (either in regulation or in function) are generated and cloned
		into \textit{Rhizobium} which are then used to inoculate
		nitrogen-starved seedlings. The morphological changes in nodule
		development (if development occurs) is assessed through
		microscopy. Bottom panel: morphological aberrations in root-hair
		infection of a \textit{dmi2-1} \textit{M.  truncatula} mutant.
		This panel displays the use of mutagenesis and microscopy to
		characterize a nodulation-deficient phenotype.  Images D and E
		show WT infection by \textit{Rhizobium}-GFP while F shows failed
		infection of the \textit{dmi2-1} mutant. Scale bars represent 15
		\textmu m.  Images adapted from \citet{Esseling:2004ja}.} 
	\label{fig:microscopy}	

\end{figure}
\newpage
\section*{Concluding Remarks} 
As no eukaryotic organism encodes the ability to fix nitrogen in its genome,
the entire biosphere relies on the perpetuation of the handful of microbial species
who do. Approximately 200 $\times 10^6$ tons of nitrogen is fixed per year through
biological processes, 20\% of which is due to legume root nodules alone
\cite{Brady:2008vg}. While the organogenesis and function of the legume root
nodule has been the focus of extensive study for nearly 200 years, there is
still much to be understood regarding the extent and mechanisms of communication
between the nitrogen-fixing bacterium and the plant host. The research described
above will help shed light on the developmental changes within bacterial
symbiont that allows the transition from free-living to nitrogen-fixing and
completely dependent. Furthermore, these findings will advance our understanding
on the formation and stabilization of bacterial-eukaryotic symbiotic
relationships in general.  

%%%%%%%%%%%%%%%%%%%%%%%%%%%%%%%%%%%%%%%%%%%%%%%%%%%%%%%%%%%%%%%%%%%%%%%%%%%%%%%%%%%%%%%%%%%
\section*{References}
\bibliographystyle{unsrtnat}
\bibliography{../../bib_files/library.bib}
%%%%%%%%%%%%%%%%%%%%%%%%%%%%%%%%%%%%%%%%%%%%%%%%%%%%%%%%%%%%%%%%%%%%%%%%%%%%%%%%%%%%%%%%%%%

\end{document}
