%%%%%%%%%%%%%EXTRA PREAMBLE SHIT%%%%%%%%%%%%%%%%%%%%%%%%%%%%%%%%%%%%%%%%%%%%%%%
%%%%%%%%%%%%Preamble%%%%%%%%%%%%%%%%%%%%%%%%%%%%%%%%%%%%%%%%
\documentclass[letterpaper, 12pt]{article}                 %
\usepackage[margin=1in]{geometry} 	                   %
\usepackage{microtype}                                     %
\usepackage{amsmath}                                       %
\usepackage{graphicx}                                      %
\usepackage{xcolor}                                        %
\usepackage{verbatim}                                      %
\usepackage{fancyhdr}                                      %
\pagestyle{fancy} 				           %
	\lhead{Griffin Chure} 				   %
	\rhead{Physical Basis of Genomic Promiscuity}      %
%%%%%%%%%%%%%%%%%%%%%%%%%%%%%%%%%%%%%%%%%%%%%%%%%%%%%%%%%%%%

%%%%%%%%%%%%%Bibliography Bullshit%%%%%%%%%%%%%%%%%%%%%%%%%%%%%%%%%%%%%%%%%%%%%%%%%%%%
\usepackage[super, comma]{natbib} %Sets bib dependency using natbib/bibtex
 \setlength{\bibsep}{0.0pt} %Reduces spacing between bibliography
 \renewcommand*{\bibsection}{} %Removes bibliography title.
 \renewcommand*{\bibfont}{\footnotesize} %Makes bib fontsize smaller.
%%%%%%%%%%%%%%%%%%%%%%%%%%%%%%%%%%%%%%%%%%%%%%%%%%%%%%%%%%%%%%%%%%%%%%%%%%%%%%%%%%%%%%

\rhead{Candidacy Research Proposal}
%%%%%%%%%%%%%%%%%%%%%%%%%%%%%%%%%%%%%%%%%%%%%%%%%%%%%%%%%%%%%%%%%%%%%%%%%%%%%%%


%%%%%%%%%%%%%%%%%%%%%%TITLE STUFF%%%%%%%%%%%%%%%%%%%%%%%%%%%%%%%%%%%%%%%%%%%%%%
\title{The Mobile Genome: Probing Horizontal Gene Transfer at Single-Cell
Resolution}
\author{Griffin Chure -- Candidacy Research Proposal -- Rob Phillips Group\\
	\textit{\small California Institute of Technology - Biochemistry and Molecular
	Biophysics}}
\date{\today}
%%%%%%%%%%%%%%%%%%%%%%%%%%%%%%%%%%%%%%%%%%%%%%%%%%%%%%%%%%%%%%%%%%%%%%%%%%%%%%%

\begin{document}
\maketitle

%%%%%%%%%%%%%%%%%%%%%%%%%%%%ABSTRACT%%%%%%%%%%%%%%%%%%%%%%%%%%%%%%%%%%%%%%%%%%%%%
\begin{abstract}
	Bacterial populations live in a fertile cocktail of foreign
	genetic material. Whether by viral infection, conjugative transfer, or
	direct uptake of DNA through transformation, bacteria frequently share
	their genetic information with their neighbors (including those of
	different species) strongly influencing bacterial evolution. Prior work
	on horizontal gene transfer has primarily been performed in bulk requiring the
	growth of saturated and well-mixed bacterial cultures. Such conditions,
	however, are far removed from natural environments where cells often grow
	in non-homogeneous multi-species biofilms. Precision measurements of
	horizontal gene transfer in single cells will allow us to paint a more
	detailed picture of the evolutionary implications in natural
	environments and thus provide a better  understanding of the
	evolutionary implications of horizontal gene transfer. In this research,
	I will characterize and quantify the frequency and dynamics of
	horizontal gene transfer during transformation and transduction at
	single-cell resolution to understand how mobile genetic information
	affects the formation and evolution of microbial communities. I have
	employed a genetically encoded system in \textit{Escherichia coli} and
	\textit{Bacillus subtilis} that allows for the observation of DNA
	molecules with sequence specificity as they are transferred between
	individual cells. The physiologically non-perturbative nature of the
	visualization system allows for \textit{in vivo} studies of the dynamics
	of conjugation, transformation, and transduction. In addition, this
	research will advance our understanding of how recombination of foreign
	genetic material into the chromosome can influence genomic architecture.
\end{abstract}
%%%%%%%%%%%%%%%%%%%%%%%%%%%%%%%%%%%%%%%%%%%%%%%%%%%%%%%%%%%%%%%%%%%%%%%%%%%%%%%%%


\section*{Introduction}
\indent Understanding the ebb and flow of genetic information through lineages
and communities has long been a focal point of biological inquiry.  Genetic
information, however, is not limited only to travel through a family tree.
Horizontal gene transfer (HGT) is the transmission of genetic material from one
organism to another through means other than sexual or asexual reproduction. In
essence, it is the assimilation of foreign genetic information into another
organisms genotype, for better or for worse. HGT allows for an immediate
variability in genotype not possible only through mutation and descent with
modification, allowing cells to tap into the local gene pool and adapt to
changing conditions.  HGT is the primary mechanism of antibiotic resistance
propagation \cite{Gyles:2014je,Koonin:2001vz, Avrain:2004hb}. In 2014, the World
Health Organization declared the rise in antibiotic resistance as the most
pressing issue in global health declaring a post-antibiotic era in which common
microbial infection are untreatable as "far from an apocalyptic fantasy, [but] a
very real possibility for the 21st century"
\cite{WorldHealthOrganization:2014ww}.  The dwindling effectiveness of our
standard antimicrobial arsenal has prompted research and development of
alternative antibiotic resistance therapies. While phage therapy (using lytic
bacteriophage to clear infections -- reviewed in \citet{Knoll:2014gq}) has been
the subject of extensive and recent research, more study is needed to understand
how genetic information in general can traverse through microbial communities.
Bulk experiments and bioinformatic methods have shown that the range of genetic
information that can (and has) been transferred is vast and its role in
evolution is still not well understood.

HGT was first discovered by Frederick Griffith in 1928 as he studied the
transformation of non-pathogenic \textit{Pneumococcus} into its nefarious
counterpart \cite{Griffith:1928vg}. However, the extent of HGT was not realized
until whole genome sequencing evolved from fantasy to reality. The emergence of
bioinformatics revealed a surprising abundance of "genomic islands" -- regions
of the genome with unusual G+C content or aberrant codon usage -- which  often
encoded functional gene products such as secretion systems and pathogenicity
cassettes (so frequent they garnered their own terminology of pathogenicity
islands \cite{Hacker:1990}). While HGT occurs at a higher frequency and has
received extensive study in prokaryotes, the eukaryotes are
not immune to this powerful force of evolution. The most obvious example of HGT
within the eukarya is the genome reduction and gene shuttling from endosymbiont
to host which occurs at surprisingly high frequency (reviewed in
\citet{Timmis:2004ks}).  Prokaryote-eukaryote HGT events (apart from bacterial
endosymbionts) are common and the number of genes acquired in this manner can
range from zero to hundreds depending on the eukaryotic species
\cite{Keeling:2008ch}. Likewise, HGT from eukaryote to prokaryote has also been
reported \cite{Doolittle:1990uo} the most astonishing example of which is the
encoding of alpha- and beta-tubulin along with the kinesin light chain in the
bacterium \textit{Prosthecobacter} \cite{Jenkins:2002ty}. There have been
several reports of HGT between eukaryotes (reviewed in \citet{Keeling:2008ch})
although most events appear to  replace genes with existing homologs rather than
confer new functions. 

%Our current understanding of eukaryote-eukaryote HGT is
%almost certainly an underestimate but will continue to improve as more genomes
%are sequenced and published.

Prokaryotes have been the classical choice for probing HGT for good reason.
The short generation times, physical size, and modern molecular-biological tool
set make bacteria the model organisms for studying not only the molecular
details of how DNA is transported into the cell but for a quantitative
dissection of the fitness effects such transfers can incite. Bioinformatic
approaches have provided great insight into the extent and limitations of
transfer. However, it is often difficult to infer how the transfer
occurred. For example, a recent comparison of 61 \textit{E. coli} genomes showed
that only 6\% of genes were common to all strains \cite{Lukjancenko:2010gd}.
What drove the evolution of this incredible genetic diversity? Many of the variable
genes appeared within genomic islands, suggesting HGT, although clues as to how
or when it was transferred are difficult to infer. More importantly, the
information produced by these studies reveals only the horizontal transfer events
which did not confer a deleterious change in fitness to the host. It is likely
that HGT is a far more common occurrence than the genomic sequence would suggest
as bioinformatic methods only report the events which "stuck". Bacterial genomes
are far from static. The genomic repertoire is nearly constantly in flux
and given a large enough culture, any section of DNA has the potential to be
transferred. However, the fraction of horizontally transferred elements being
maintained over numerous generations is low. Absorbing foreign
DNA into your genome is a risky endeavor as a mis-step in recombination or the
uptake of incompatible or fatal alleles is more likely than acquisition of
something beneficial. Restriction-modification and CRISPR systems appear
to have evolved to seek and destroy foreign DNA (reviewed in
\citet{Samson:2013hu}). While often presented in the context of phage immunity,
evolution of these systems to target HGT elements (such as conjugative
plasmids \cite{Marraffini:2008vm}) is not unheard of. Thus, it is likely that our
current measurements underestimate the frequency of HGT as our bulk and
bioinformatic methods are biased against any transfer events that convey a
fitness cost.

The breadth of HGT across all major divisions of life shows
that mutation and descent with modification isn't the whole story behind the
incredible biodiversity of nature. Unfortunately, we still lack a thorough
understanding of the role that HGT currently plays in the natural
world as the vast majority of work probing the events that facilitate HGT has been
performed in conditions that are far from those in the real world. In order to understand the
influence HGT holds over natural microbial populations, one must have a clear
picture of the frequency, dynamics, and important environmental parameters on
size and timescales that are relevant to emerging and established communities.
In the research described below, I aim to generate a quantitative understanding
of the frequency and dynamics of HGT through transformation and bacteriophage
transduction at single-cell and single-molecule resolution. By varying
properties of transferred DNA (such as concentration, physical characteristics,
and homology with the chromosome), I will be able to unravel what parameters
drive successful transformation in \textit{Bacillus subtilis} as well as the
relationship between DNA uptake and duration of the competence state. Transduction frequency and the positional dependence of transferred genes will
be investigated in both bacteriophage P1 and $\lambda$ which are known to
package host DNA through different mechanisms. Finally, I will probe how HGT may
influence genomic architecture through recombination near genes of varying
transcriptional activity.  With this
information, we will be better equipped to explore how HGT can drive the
evolution of bacterial communities who thrive in a mixed communal gene pool.
The following sections will describe the primary mechanisms of HGT in the
microbial world (Fig.  \ref{fig:mechanisms}), special cases which have only
recently been discovered, and the evolutionary implications HGT which tangle the
branches of the microbial tree-of-life.


\subsection*{Conjugation}
\indent Through the early 1940s, understanding of bacterial phylogeny and
prokaryotic inheritance as a whole had been limited to the accumulation of
mutations through vertical transmission of genetic information from parent to
progeny. The first evidence of genetic recombination in bacterial specimens came
from the work of \citet{Tatum:1947va} in 1947 which showed that bacteria can
transfer information with their neighbors by entering a "sexual stage." Several
years later, the fertility factor (F element -- a specific conjugative plasmid)
was discovered and was shown to permit transfer of genetic information from the
donor to the recipient \cite{Lederberg:1952wq}. In the 70 years since discovery,
many other naturally occurring conjugative plasmids have been identified
and the molecular details of conjugation have been studied extensively.

Conjugative elements encode all genes necessary for a cell to initiate conjugation.
Upon the assembly of a specialized membrane-bound sex pilus, the donor cell
binds to the surface of an acceptor cell and pulls the two cells toward
one another. Once sufficiently close, a fragile membrane connection is made
between the two cells through which the DNA will be transferred. While passage
through the membrane connection is the most common route for conjugation,
transfer has been observed at several \textmu m distances, suggesting that DNA
is able to transfer through the pilus directly \cite{Babic:2008bl}. Unlike
transduction and transformation (described in the following sections) the
transferred DNA is replicated as it is pushed into the acceptor cell, ensuring that
the donor retains the conjugative element. The safe passage of DNA into the acceptor cell
requires the orchestration of many conjugation-specific enzymes (known as the
relaxosome) which are the subject of active study \cite{Guglielmini:2014bc}.
Once successfully transferred, the acceptor cell becomes a donor and is capable
of infecting nearby cells, meaning that conjugative elements can pass
through concentrated microbial communities in an epidemic manner.

To ensure that acceptor cells will become capable of conjugation themselves,
conjugative elements begin  and end transfer at a specific site recognized by
the relaxosome. If a conjugative plasmid is integrated into the chromosome, the
cell becomes capable of transferring genomic material through conjugation, known
as a high-frequency recombination cell (HFR). Once conjugation from an HFR
begins and as long as the cell-cell connection is maintained, chromosomal DNA
will be transferred into the acceptor cell, meaning that entire genome transfers are
possible.  The cell-cell connection is very fragile, however, and is sensitive
to both chemical and mechanical disruption such that transfer may be  halted
before the entire chromosome (or plasmid) is completely transferred. This
phenomenon proved to be incredibly useful to create the first gene maps of
bacterial chromosomes through intermittent disruption of conjugation
\cite{Bresler:1978uv}. As cells need to be in direct contact, conjugation has a
very short range of transfer. The fragility of the connection confounds the
transfer efficiency so that substantive conjugation is a relatively minor avenue
for HGT in planktonic environments. However, conjugation is a frequent
occurrence (one transfer per 10$^{3}$ cells \cite{Molin:2003cqa, Li:2001km,
Hausner:1999ua}) in biofilms where cells are in very close quarters
\cite{Sorensen:2005jw}. 

Conjugative plasmids, HFRs, and transposons are classified as integrative conjugative
elements (ICEs) so long as they encode the proper machinery for their
maintenance, dissemination, and regulation \cite{Burrus:2004ca}. Unlike
plasmids, integration into the chromosome of its host ensures vertical
transmission into the progeny upon division. The probability of loss is much
lower than in the case of plasmids which are autonomous elements requiring their
own partitioning systems \cite{Funnell:2004wi}. Most ICEs contain a recombinase
which integrates into a specific locus on the host genome  and a locus on the
ICE \cite{Burrus:2004ca}. As transfer during conjugation is always initiated at
the same location, the host genes transferred are dependent on the location of
integration. Related ICEs often share the same integration sites, resulting in
increased variability at these sites within bacterial species and genera. A
prime example of such variability is within the \textit{fda} locus of
\textit{Streptococcus thermophilus} in which various genomic islands and ICEs
have been integrated, resulting in a high degree of polymorphism
\cite{Pavlovic:2004hl}.
 
While common in the microbial world, conjugation has also been shown to permit the
transfer of genetic material across domains of life. The parasitic
\textit{Agrobacterium} is capable of infecting neighboring plant roots with
tumor inducing (Ti)  and root inducing (Ri) plasmids. The infected cells develop
into specialized structures known as crown gall (for Ti) or root tumors (for Ri)
which synthesize and secrete small molecules used by
\textit{Agrobacterium} as a nitrogen source \cite{Pan:1995tf}.

Although limited, there have been \textit{in situ} and \textit{in vivo} studies
which have helped shed light on how the conjugal elements traverse bacterial
communities at single-cell resolution. \citet{Hausner:1999ua} tracked the
passage of conjugative elements through developing \textit{Alcaligenes
eutrophus} biofilms via fluorescence \textit{in situ} hybridization (FISH)
coupled with automated confocal imaging.  It was found that plasmids were
transferred at a frequency 1000 fold the estimate measured via traditional bulk
plating assays and was independent of nutrient concentration of the medium. This
method served as a measure of total transfer events, but was unable to measure
the stability of the transferred plasmid over the course of subsequent cell
divisions. Nearly a decade later, \citet{Babic:2008bl} used real-time
fluorescence microscopy to measure the
frequency of conjugative transfer within a mixed population of donors and
acceptors. To visualize the horizontally transferred DNA, the authors created a
SeqA-YFP fusion which specifically binds hemimethylated DNA, forming bright foci
wherever transferred DNA was found.  This allowed for observation of transfer of
any DNA sequence from a methylation deficient host. By watching the flow of DNA
through a mix of donor and acceptor cells, the authors were able to measure
several phenomenon not possible using bulk methods. Under experimental
conditions, conjugation allowed the transfer of HFR DNA to nearly all nearby
recipients over the course of a single cell division. Furthermore, they were
able to show that  recombination into the chromosome occurred in 97\% of the
observed cells, a rate much higher than the 10\%-30\% estimated through bulk
studies \cite{Rayssiguier:1991to}.  Transfer was observed within five minutes of
incubation and occurred over great distances (with a maximum of 12\textmu m between
donor and acceptor), proving that DNA can be transferred through the  pilus
directly. These studies demonstrate the power of using
single-cell/single-molecule methods to reveal both the molecular mechanisms as
well as the dynamics of HGT not visible (or at least underestimated) through
bulk methods.  Similar approaches studying transformation and transduction
(described below) have not been pursued, leaving a single-cell
description of such processes incomplete. 

\subsection*{Transduction}
\indent After the discovery of recombination through conjugation in \textit{E. coli},
Joshua Lederberg, along with Norton Zinder , searched for a similar mechanism in
\textit{Salmonella typhimurium} \cite{Zinder:1952ug}. Recombination between
auxotrophs to generate wild-type cells was observed at rate of 1 in 10$^5$.
However, recombination still occurred even when the two mutants were
separated by a filter. After varying the pore size of this filter, the vector of
recombination was found to be approximately the size of bacteriophage P22, a
temperate phage of \textit{Salmonella}.  Instead of finding conjugation, they
had discovered yet another avenue of gene transfer -- viral transduction.

With a global population of 10$^{31}$ particles, bacteriophage (phage) are the
most numerous biological entities \cite{Comeau:2008jo}. This population is estimated
to perform $10^{16}$ transduction events per second
\cite{ChibaniChennoufi:2004gv}, demonstrating their power in altering microbial
genomes. The range of genes that can be transferred is diverse, but is dependent
on the phage species. Transduction can be either generalized or specialized,
depending on the phage life cycle. A variety of phage --
such as T4 and P22 -- fill their capsid using a "headful packaging"
mechanism with continues the packaging until full even if the entire viral
genome has already been packaged \cite{Coren:1995up}. This leads to promiscuous
packaging and allows the occasional loading of bacterial chromosomal fragments into the
phage, forming a transducing particle. Upon lysis, these transducing phage can
infect other bacterial cells facilitating HGT. Bacteriophage P1 and Mu, both generalized
transducers, are frequently used in molecular biology to transfer chromosomal DNA
between bacterial strains of interest \cite{Thomason:2007gz, Wang:1987ul}.

Many phage can enter a stage of dormancy called lysogeny in which the phage
genome will be integrated into the chromosome of the host. In this state, the genes
needed for viral assembly and lysis are silenced, allowing the cell to
proliferate and spread the silenced phage (prophage) into its progeny. Once the
phage exits lysogeny and becomes lytic, the prophage is excised from the
host chromosome, sealing the cells fate. This excision from the chromosome is
not always perfect, however, as bacterial genes adjacent to
the site of integration will occasionally be excised as well as packaged into
phage particles \cite{Sato:1970tz}. This phenomenon, called specialized
transduction, allows for only specific regions of the host chromosome to form
transducing particles. The set of genes that can be transferred via specialized
transduction is dependent on the location of the phage integration locus. Unless
the phage also packages DNA via headful packaging, random segments of the host
DNA will not be packaged. 

Despite being discovered 70 years ago, the ecological implications of
transduction have only been the subject of interest for the past two decades.
Initially disregarded as negligible due to the lytic consequences of phage
infection, multiple studies have shown that transduced bacterial cells can
survive for many weeks after infection in natural conditions such as soil
and marine environments \cite{Zeph:1988up, Jiang:1998ul}. While initially
considered an oddity and attributed to contamination of free bacterial DNA,
metagenomic and genotyping studies showed that a myriad of cellular genes have
been incorporated into bacteriophage genomes \cite{Comeau:2008jo} such as the
photosystem II core reaction center and electron transfer components
\cite{lindell:2004vg}, phosphate-sensing genes \cite{Miller:2003hf}, and vitamin
B-12 synthesis enzymes \cite{Williamson:2008fc}. The abundance of cellular
genes buried within phage genomes suggests that phage can strongly 
influence the metabolic repertoire of their host. Unlike the conjugation pilus, phage
are surprisingly robust in the face of physical and chemical environmental
challenges (reviewed in \cite{Jonczyk:2011db}). Although their stability is
dependent on type, phage particles can withstand wide ranges of temperatures
(0$^\circ$ - 105$^\circ$ C), pH (3.5 - 9.0), and abrupt changes in salinity.
This means that transducing phage can survive through multiple generations of
microbial species, potentially serving as a genomic fossil of previous
environments. The specificity of phage-host interactions continues to be a
subject of intense research \cite{ChibaniChennoufi:2004gv} and is variable
between phage, limiting the range of hosts to which phage can transfer bacterial
DNA. Their ubiquity and stability make them powerful vectors for HGT, however
their role in shaping bacterial genomes and communities remains enigmatic. 

While HGT in biofilms is dominated by transformation (described below) and
conjugation, it has only recently been shown that 
transducing bacteriophage are capable of propagating bacterial genes through
biofilms. An unfortunate example is the deadly 2011 outbreak of pathogenic
\textit{E. coli} in which acquisition of Shiga-toxin production genes was
acquired horizontally from related pathogens. This was first discovered via
sequencing methods \cite{Rasko:2011bf} but was later shown experimentally to be
transferred via temperate bacteriophage $\phi$731 through intestinal biofilms
\cite{Solheim:2013bi}. As the biofilm state is believed to be \textit{modus
operandi} for most natural bacterial species, bulk-scale transduction of
planktonic bacteria is not an adequate representation of microbial life. To
understand the extent at which generalized and specialized transducing phage can
influence the establishment of bacterial communities, a single-cell description
is necessary.


%%%%%%%%%%%%%%%%%%%%%%%%%%%%%%%%FIGURE 1%%%%%%%%%%%%%%%%%%%%%%%%%%%%%%%%%%%%%%%%%%%%%%%%%%
\begin{figure}
\centerline{\includegraphics[width=\textwidth]{figs/mechanisms.eps}}

\caption{The primary mechanisms of HGT. \textbf{Conjugation},
\textbf{transduction}, and \textbf{transformation} are illustrated in the above
panels from left to right respectively. While the molecular machinery for each
mechanism of transfer is vastly different, each is capable of drastically
changing the genomic landscape. The host specificity, range at which transfer
from donor to acceptor can occur, and stability (resistance to chemical and physical
degradation) is given for each transfer mechanism. A symbol legend is present at the bottom of each panel.}

\label{fig:mechanisms}
\end{figure}
%%%%%%%%%%%%%%%%%%%%%%%%%%%%%%%%%%%%%%%%%%%%%%%%%%%%%%%%%%%%%%%%%%%%%%%%%%%%%%%%%%%%%%%%%%


\subsection*{Transformation}
Being one of the largest and most hydrophilic biomolecules in nature, the
transfer of DNA from the extracellular environment into the crowded environment of the
cytoplasm is no easy task. After traversing the rigid cell wall and the
hydrophobic membrane barrier, the DNA must evade the myriad of
restriction systems evolved to degrade the potentially hazardous foreign
material and either circularize as a plasmid or integrate into the host
chromosome. In spite of these challenges, the uptake of free DNA from the
environment into the cytosol (known as genetic transformation) is a common
occurrence in natural environments. Transformation was first observed by
Frederick Griffith in 1928 \cite{Griffith:1928vg} upon noticing that incubation
of heat-killed pathogenic \textit{Pneumococcus} with a living yet non-pathogenic
variant yielded pathogenic organisms. While Griffith himself was not able to
identify the chemical identity of the "transforming principle", Oswald Avery,
Colin MacLeod, and Maclyn McCarty were able to prove that this substance was DNA
(not RNA or protein) \cite{Avery:2014wx} leading to the conclusion that DNA was
the hereditary material of life. Unfortunately, their results were initially
disregarded as protein was still considered the most likely candidate for
hereditary material. In the nearly 100 years since its discovery, the process of
DNA transformation remains a topic of extensive research at the biochemical,
systematic, and evolutionary level.

Our current understanding suggests that transformation can occur via two
fundamentally different processes -- artificially via chemical and physical
shock or naturally through entry into a genetically regulated physiological
state known as competence. While artificial transformation is used ubiquitously
in a molecular biology context, chemical and electrotransformation has been
shown to be a viable route for HGT in natural environments. Chemical
transformation requires both a chemical and temperature shock to cells,
resulting in permeabilization of the plasma membrane, allowing the diffusion of
plasmids and linear DNA fragments into the cytoplasm. Despite its ubiquity, the
precise mechanism of chemical transformation is still not well understood, yet this
phenomenon has proven to be incredibly valuable in  biological research.  A
standard protocol requires the somewhat harsh treatment of cells by washing and
incubating cells in non-physiological CaCl$_2$ concentrations along with large
temperature shocks (often $>$ 40$^\circ$ C). This protocol has been optimized
for laboratory purposes and one would be hard pressed to find these conditions
in natural environments.  However, chemical transformation can occur at lower
concentrations of CaCl$_2$ and smaller temperature shocks, both of which are
present in natural environments. \citet{Baur:2006ba} showed that \textit{E.
coli} cells can readily be transformed in natural freshwater with low
($\sim$2mM) CaCl$_2$ concentrations and temperature fluctuations (0$^\circ$ -
10$^\circ$ C) that are typically observed on a daily basis. This work was
expanded upon by \citet{Woegerbauer:2002ev} who showed that clinical isolates of
\textit{E. coli} could be transformed in mineral water, but not in medically
relevant media such as urine. The extent of chemical transformation in nature is
undoubtedly sensitive to the geochemistry of the ecosystem, but has the
potential to serve as another avenue of HGT.

Electrotransformation is another method of artificial transformation in which
an  electric field (often $\sim$ 1800 V/cm) is applied to a DNA
and cell mixture suspended in a low ionic strength medium. The application of
pulsed electric fields results in the transient permeabilization of the membrane,
allowing the diffusion of DNA and other extracellular components into the
cell \cite{Wegner:2015fx}. Much as in chemical transformation, the preparation of
electrocompetent cells requires non-physiological treatment including successive
washes with deioninzed water to remove ions that would lead to electrical
discharge. While it may be hard to imagine an environment that would be
conducive to electrotransformation, it has been shown that such conditions can
be found in soil microcosms through lighting-mediated current
injection \cite{Demaneche:2001kb}. Given that lighting strikes occur 
approximately 44 times per second \cite{NOAAlighting}, it is not unreasonable to
believe that electrotransformation is a naturally occurring phenomenon.
So-called "lightning-competent" \textit{Pseudomonas} strains have been isolated
from natural soil environments \cite{Ceremonie:2004gr} suggesting that natural
electrotransformation is more than conjecture.

Unlike artificial transformation, natural transformation through genetic
competence is very well understood in several model organisms and its relevancy
in natural environments is indisputable. Competence is found across many phyla
of bacteria with gram-positive and gram-negative members equally represented. As
of this writing, there are approximately 80 species known to exhibit
competence \cite{Johnston:2014dc}, although the evidence in some cases consists
of a single report.  The uptake machinery is best understood in \textit{Bacillus
subtilis} and \textit{Neisseria gonorrhea} (which is constitutively competent
\cite{Biswas:1982tl, Biswas:1989ui}) and serve as the archetypal systems
for gram-positive and gram-negatives respectively. Regardless of the profound
differences in the cell envelope composition, both gram-negatives and
gram-positives share similar mechanisms of DNA transport and processing. Due to
the presence of a double membrane, transformation occurs in two steps in
gram-negative bacteria -- DNA uptake (passage of dsDNA across the outer
membrane) and DNA translocation (passage of ssDNA across the inner membrane). In
both gram-positive and gram-negative bacteria, free DNA in the environment is
bound to the exterior of the cell through interaction with surface fibers of the
Type IV pilus family (T4P) \cite{Burton:2010ee}. T4P complexes are used by many
bacteria in "twitching motility" -- a flagella-independent method of movement
across moist surfaces \cite{Mattick:2002il}. A competence-specific pseudopilus
found in both gram-negative and gram-positive bacteria has been reported,
although its role in free DNA recognition remains
unclear \cite{Johnston:2014dc}. After being bound to the cell surface, the
dsDNA is pulled into the periplasmic space in gram-negatives (or to the inner
membrane in gram-positives) where it is bound by a channel within the inner
membrane. One strand of the dsDNA is hydrolyzed as the other strand is brought
into the cytoplasm where it is either degraded by restriction systems,
circularized and replicated into a plasmid, or integrated into the chromosome
via homologous or non-homologous recombination.


In general, DNA uptake is non-specific allowing for the horizontal transfer
between different species, kingdoms, and even domains of life. However, two natural competence systems
appear to preferentially take up DNA from related organisms by recognizing a DNA
uptake sequence (DUS) (reviewed in \citet{Mell:2014dj}). This phenomenon is
found in the \textit{Neisseria} and \textit{Pasteurellacae} families (both of
which are gram-negative) who recognize short 9 - 10 bp motifs in exogenous DNA.
Upon sequencing of these genomes, it was found that these same motifs were found
scattered throughout the genome with a frequency of about one occurrence per
kilobase pair.  How this sequence is recognized still remains unknown. A single
sequence is sufficient to allow the transport of both short and long DNA
fragments, suggesting that DUS recognition is important only for initiation of
DNA transport. Of the DUSs found within the genome, one-third are within open
reading frames and are transcribed and translated into specific tripeptides. The
proteomes of \textit{Neisseria} and \textit{Pasteurellacae} show a distinct
enrichment of these peptides compared to other organisms which lack a DUS (such
as \textit{E. coli}), illustrating that uptake specificity can exert a powerful
force on genome evolution.  It is important to note that specificity in uptake
is found in only these two organisms who are distantly related. Their DUS differ significantly in
composition, suggesting that specificity has evolved independently.


Entry into the competence state is a metabolically expensive endeavor, so what
prompted its evolution? It is possible that competence and
natural transformation evolved to support genetic diversity \cite{Barton:1998uq,
Otto:2006vm}, to allow the use of DNA as a carbon source \cite{Dubnau:1999vq,
Redfield:2001vx}, or to use free DNA as a template for repair
\cite{Claverys:2006do, Dorer:2010tf}. 


\subsection*{Special Cases}
While conjugation, transformation, and transduction are the primary ways which
DNA is horizontally transferred, there are other mechanisms which appear to
facilitate HGT, although the precise mechanism and the extent at which they
occur is unknown.

Gene Transfer Agents (GTAs) are  phage-like particles that package random dsDNA
fragments from the chromosome into a capsid-like structure which are released
into the environment through cell lysis (reviewed in \citet{Lang:2012df}). While
reminiscent of generalized phage transduction, the production of these
particles is not the product of a viral infection as all components needed for
their assembly are encoded within the producing cell's genome. Additionally, the
amount of DNA that can be packaged is less than the length of the GTA encoding
genes, meaning that the particles cannot be self perpetuating (although the GTA
operon maybe reconstituted upon multiple infection). Four
naturally occurring GTAs have been discovered and all appear to be unrelated
evolutionarily. Their structural and functional relationship with known
bacteriophages is indisputable and suggests an evolutionary connection. It is
unknown whether GTAs were derived from pre-existing phage, whether the GTAs
have a completely bacterial origin and gave rise to tailed phage, or if GTAs
and phage evolved from undiscovered or extinct virus-like particle.

Direct cytoplasmic connection between neighboring cells is a common form of
communication in multicellular organisms, such as plasmodesmata in
plants \cite{Heinlein:2004fa} and gap junctions in mammals \cite{Kumar:1996tm}.
When growing upon a solid surface, bacterial cells can form networks of
tunneling nanotubes -- cytoplasmic connections between adjacent
cells \cite{Dubey:2011dp}. These connections have been shown to transport
cytoplasmic contents between cells including small-molecules, proteins, and even
plasmids. While nanotubes are short (up to 1\textmu m in length), they may play
an important role in HGT within biofilms.  This is of particular concern as they
are capable of passing antibiotic resistance genes within single and
multi-species biofilms. It remains unknown how cargo is transported through the
nanotubes (active vs. passive diffusion), their molecular composition, or what
(if anything) regulates their formation.

It has been known for nearly 50 years that many species of gram-negative
bacteria readily produce outer membrane vesicles (MVs), although the function
and composition has only been the subject of inquiry for the past two decades.
Ranging from 50 - 250 nm in diameter and composed entirely of outer membrane,
MVs contain a diverse array of cargo such as toxins, signaling molecules,
various other proteins, and DNA (reviewed in \citet{MashburnWarren:2006jc}). MVs
may serve numerous biological roles ranging from predation and defense,
long-range communication, and as an avenue of HGT. What drove their evolution is
still not understood and their role in HGT may be an unintended
consequence. Regardless, their ability to transfer DNA not only between species
but between domains warrants further investigation into their role of genome evolution.


\subsection*{Evolutionary implications of HGT}

The ubiquity and evolutionary importance of HGT is becoming increasingly
apparent as more complete genomes are sequenced and made publicly available. As
the microbial world is not limited only to the set of genes present in its own
lineage, the "species tree" concept becomes difficult to interpret. From a
mathematical framework, trees are defined as a framework in which there are no
reticulations between nodes in a given branch. This type of model is not
applicable to the microbial world where genes can flow in multiple dimensions.

For nearly the past 60 years, the uniformity of the
genetic code amongst all life was described by one of two models. The
stereochemical model suggested a deterministic relationship between the chemical
properties of the codon and its amino acid \cite{Grafstein:1983wv} whereas the
"frozen accident" model suggested that the genetic code could be arbitrary and
was universally adopted as all life evolved from the last universal common
ancestor \cite{Crick:1968wg}. There is a third possibility, however, that HGT
exerted a selective pressure to allow the exchange of genetic information
between species \cite{Syvanen:1985vv, Syvanen:2012jn}, forcing all life to adopt
and optimize the same (or at least very similar) genetic
code. \citet{Vestigian:2006va} demonstrated a unity between the stereochemical model
and HGT. They showed computationally that a pool of individuals connected by
avenues for HGT lead to the unification and optimization of a standard genetic
code. 

While HGT has received much experimental and theoretical attention over the past
century, there are still gaping holes in our understanding of its deeper
implications and influence on evolution as whole. Generating a complete picture
of how genetic information is passed between organisms (both through vertical
and horizontal transmission) is imperative not only for our understanding of the
origin and diversification of microbes, but of the entire biosphere.

HGT is particularly studied by looking for the transfer of operational or
protein-coding sequences. However, the transfer of short non-coding regions is
more difficult to measure experimentally and has not been examined in depth
through bioinformatics. It has only recently come to light that non-coding
regulatory regions of a significant minority of core genes (13\%) in \textit{E.
coli} have been acquired horizontally \cite{Oren:2014ee}. This horizontal
regulatory transfer (HRT) frequently resulted in new transcription factor
binding sites, allowing for physiologically relevant changes in expression to
occur. The breadth of HRT in microbial evolution is largely unknown and warrants
further research into its frequency and ability to significantly alter
physiology.


\section*{Experimental Approach} 
\subsection*{Watching DNA flow in real-time}
A vast majority of studies focusing on the frequency and efficiency of HGT in
microbes are performed in bulk -- often growing bacterial cultures to saturation
and measuring success of HGT by counting "colony forming units" (CFU). This leaves
many aspects of what happens at the cellular level unanswered, however, as CFU
serves as merely an approximation of cell number. In addition, microbial
populations are often far from saturation in natural
environments \cite{Vieira:2005jw, Whitman:1998tj} and thrive as structured,
non-homogeneous mixtures known as biofilms \cite{HallStoodley:2004cv}. The
disparity in physiological relevance between the lab bench and the real-world
calls for single-cell and single-molecule techniques to study the dynamics and
evolutionary consequences of HGT.

A common approach to monitoring HGT is by selecting for some acquired gene, such
as an antibiotic resistance element or a fluorescent reporter. While these
methods are useful for bulk measurements, the time from acquisition to
observable phenotype is often longer than the typical bacterial division time as
the transferred element must be stabilized (not degraded), transcribed,
translated, and properly folded before they can combat the selection. While this
is particularly true for antibiotic resistance genes, the maturation time of
many fluorescent proteins range from ten minutes to two
hours \cite{Iizuka:2011ia}. It is therefore difficult to identify when the DNA
was transferred. Additionally, measuring HGT with selectable markers grossly
underestimates the frequency as one only measures the "winners" -- those events
which avoided degradation, were properly expressed, and did not kill the cell
along the way.  It is likely that HGT occurs much more frequently but is missed
by classical measurements. To properly quantify the extent of HGT, one can not
ignore the "losers".

Although not focused on horizontal transfer, many molecular systems are in place
to  probe the sub-cellular localization of DNA elements using fluorescence
microscopy. FISH has been used extensively
over the past three decades to make such measurements \cite{Levsky:2003bz},
although the cell fixation required makes dynamical measurements impossible.
Over the past ten years, techniques for the sequence-specific labeling of DNA
\textit{in vivo} have been developed with varying degrees of applicability.
Repressor binding site arrays, such as \textit{lacO} and \textit{tetR}, have
been used to observe dynamic phenomena plasmid trafficking \cite{Ho:1911wf},
chromosomal rearrangements \cite{Lau:2004bp}, and telomere
mobility \cite{Jegou:2009kb} in living cells by fusing the repressor protein with
a fluorophore and observing localization. To be observable, however, there must
be many repeats of the binding sequence (often around 250) which not only
results in a large insertion of DNA (8 - 10 kbp) but can be physiologically
disruptive to the cell by interfering with DNA replication \cite{Dubarry:2011bx,
Payne:2006fc}. As HGT of such massive fragments is likely inefficient and the
physiological consequences are too severe, an alternative methodology is needed.

%%%%%%%%%%%%%%%%%%%%%%%%%%%%%%%%%%%FIGURE%%%%%%%%%%%%%%%%%%%%%%%%%%%%%%%%%%%%%
\begin{figure}

	\centerline{\includegraphics[width=\textwidth]{figs/par_function_two.eps}}

	\caption{Fluorescent labeling using the \textit{parABS} system. The {\bf
		\textit{parABS} segregation mechanism} under native conditions
		is described in the top left panel. Upon binding of a ParB protein 
		to a \textit{parS} sequence, many other copies of ParB are
		recruited and oligomerize on the plasmid. This complex then
		binds to polymerized ParA bound to the nucleoid, triggering
		depolymerization. The plasmids chase this depolymerization to
		opposite poles of the dividing cell resulting in segregation.
		{\bf Plasmid visualization in \textit{E. coli}} is shown in the
		right and bottom panel. The bottom image shows \textit{E. coli}
		cells expressing \textit{YGFP-parB} system with a \textit{parS}
	containing plasmid growing on a solid surface. The top right panel
illustrates the source of fluorescent puncta.} 
		
		\label{fig:par_system}
\end{figure}
%%%%%%%%%%%%%%%%%%%%%%%%%%%%%%%%%%%%%%%%%%%%%%%%%%%%%%%%%%%%%%%%%%%%%%%%%%%%%%%

Only recently has such a system become available. Low-copy number autonomous
genetic elements, such as plasmids and chromosomes, often encode partitioning
systems which ensure faithful segregation into the daughter cells upon division.
Partitioning systems are diverse in strategy ranging from site-specific
recombination to post segregational killing mechanisms to highly-specific and
highly-regulated DNA-protein complex formation. The latter is exceptionally
common and is often referred to as a Par system. This autoregulated mechanism of
segregation consists of three components (reviewed in \citet{Funnell:2004wi} and
\citet{Bignell:2001ti}, Fig. \ref{fig:par_system} top panel) -- a $\sim$ 100bp
DNA binding site (\textit{parS}), a DNA binding protein (ParB), and an ATPase
(ParA). During the first steps of partitioning, diffusive ParB molecules
recognize and bind to an available \textit{parS} sequence. Once bound, ParB
experiences a conformational change which promotes the cooperative association
of other ParB dimers, resulting in the recruitment of several hundred copies to
the DNA. This complex is very large and leads to non-specific binding of DNA
outside of the \textit{parS} sequence (a phenomenon known as "spreading"). In
its ATP bound state, ParA non-specifically binds to the nucleoid in a
cooperative manner forming long polymers. Upon binding of the
ParB-\textit{parS} complex to ATP-bound ParA, hydrolysis occurs forcing
depolymerization of the ParA-nucleoid complex. ParB-bound plasmids can "chase"
this receding wave of depolymerizing ParA to the poles of the cell, resulting in
faithful segregation (Fig. \ref{fig:par_system} right panel).  This mechanism was only recently observed although
hypothesized through plasmid
labeling via \textit{tetO}-arrays \cite{Ringgaard:2009uh}.

By removing ParA from the system, the ParB-\textit{parS} complex remains
assembled but is doomed to drift aimlessly through cytoplasmic milieu. Attaching
fluorophores to the ParB results in the formation of fluorescent puncta (Fig.
\ref{fig:par_system} bottom panel). This system has been used to monitor
chromosomal rearrangements in numerous bacteria, but has never to my knowledge
been used to observe the transmission of mobile genetic
elements \cite{Nielsen:2006hmb, Nielsen:2007ct, Shebelut:2010ic,
Broedersz:2014jz}. Beginning with two \textit{parABS} homologs, four
fluorophores, and two organisms, I have been able to employ this system in
\textit{E. coli} and am fine-tuning it in \textit{Bacillus subtilis}, allowing
for the visualization and quantitation of HGT through transformation,
transduction, and conjugation in multiple species at single-cell resolution. 

Aside from studying HGT, this system will allow further investigations into
events and parameters such as gene copy number and plasmid competition.
Plasmids often encode regulatory components which titrate the copy number to
ensure faithful inheritance without imposing a metabolic burden on the host. The
copy number of plasmids can vary greatly from single copies to many hundreds per
cell.  Determination of copy number has typically been performed through
quantitative electrophoresis \cite{Schmidt:1996wk}, qPCR and digital PCR
methods, and through single-cell dilution studies \cite{BrewsterDilution}.
Preliminary data shows that there is a correlation between the diffusive motion
of the fluorescent puncta and their brightness. This may be representative of
plasmid clustering either through binding of ParB or through other unknown
mechanisms. By examining the correlation between motion and intensity, it may be
possible to elucidate plasmid copy number as well as the dynamics of
replication. By integrating the \textit{parS} sequence into various loci in the
chromosome, it  will be possible to observe the dynamics of chromosomal
replication throughout the cell cycle. While prokaryotes are  almost always
haploid organisms, the chromosome is nearly continually replicated as the cells
are growing in the exponential phase, allowing for 2 - 4 copies of the
chromosome to exist in the dividing cell at a single point in time. Gene dosage
effects have been shown to be important in fast growing bacteria
\cite{Couturier:2006hm}, suggesting that changes of gene copy number can alter
the physiology of the organism. In addition to plasmid counting, this system can
be used to track the copy number of numerous loci over the course of the cell
division. By integrating physiologically relevant genes (such as repressor
molecules) alongside the \textit{parS} site, one will be able to quantify the
physiological effects variations in copy number may play in cellular decision
making.

\subsection*{Measuring the limits and frequency of natural transformation}
%%%%%%%%%%%%%%%%%%%%%%%%%%%%%%%%FIGURE 3%%%%%%%%%%%%%%%%%%%%%%%%%%%%%%%%%%%%%%%%
\begin{figure}
	\centerline{\includegraphics[width=\textwidth]{figs/bacillus_experiments.eps}}

	\caption{
		Measuring the transformation frequency and the relationship
		between uptake and competence. \textbf{Transformation frequency}
		can be measured by forcing \textit{B. subtilis} into competence
		and supplying exogenous DNA containing \textit{parS} binding
		sites. The three plots represent hypothetical curves and
		distributions from transformation measurements of varying DNA
		concentration, type, and homology. \textbf{Dynamics of
		competennce} in relation to DNA uptake can be measured in a
		similar manner. By using the natural competence of \textit{B.
		subtilis}, we can measure the time to DNA uptake and time to
		exit from competence by tracking the production of a CFP
		reporter. The two histograms represent hypothetical
		distributions of these measurements.}
	\label{fig:transformation_exp} 

\end{figure}
%%%%%%%%%%%%%%%%%%%%%%%%%%%%%%%%%%%%%%%%%%%%%%%%%%%%%%%%%%%%%%%%%%%%%%%%%%%%%%%%%%%

While DNA uptake may occur through other currently unidentified processes, our
current understanding dictates that entry into the competence state is a
requirement for uptake of free DNA from the environment through natural
transformation. Luckily, the regulatory networks controlling entry and exit from
the competence state have been studied extensively in many model organisms such
as \textit{Streptococcus pyogoenes}\cite{Woodbury:2006dg}, \textit{Pseudomonas
stutzeri} \cite{Graupner:2001ii}, and \textit{Bacillus subtilis}
\cite{Suel:2006ea,Suel:2007dm}, among others.  Those mentioned are model
organisms, affording us the opportunity to study the dynamics of DNA uptake and
persistence or degradation in developing colonies and multi-species communities.

When faced with nutritional stress, a minority of \textit{B. subtilis} cells
will enter the competence state while most will differentiate into endospores --
durable yet non-reproductive structures capable of being reanimated when
environmental conditions improve. Entry into the competence state is driven by
the activity of the necessary and sufficient transcription factor ComK
\cite{Berka:2002uf}. ComK is positively autoregulated and is degraded by the
ClpC-MecA complex \cite{Smits:2005jw}. This degradation is inhibited by ComS
which in turn is repressed by ComK \cite{Ogura:1999wb}. While relatively simple,
this regulatory architecture results in a highly excitable circuit in which
minor changes in expression are amplified into a developmental response. The
dynamics of competence entry and exit has been studied in depth
\cite{Suel:2007dm, Suel:2006ea, Cagatay:2009ie} producing many strains in which
the components of the regulatory circuit are under genetic control. In
collaboration with Adam Rosenthal from the Michael Elowitz group, I have been
able to translate the \textit{parBS} DNA visualization system into both
wild-type and genetically modified \textit{B. subtilis} cells whose entry into
competence can be chemically induced (strains described in
\citet{Suel:2007dm}). In these strains, the expression of ComK is under control
of an inducible promoter. \citet{Suel:2007dm} were able to titrate the extent of
competence through careful induction with IPTG. The sensitivity of this circuit
will allow me to force cells to enter and exit competence while varying the
parameters of the transforming DNA. While \textit{B. subtilis} halts division
upon entrance to competence, a more sensitive metric is needed to track entry
and exit from competence. A CFP fluorescent reporter gene is driven by the
\textit{comG} promoter which is activated by ComK. Through time lapse
microscopy, one can track the entry, duration, and exit from the competence state
by monitoring the change in CFP production. This serves as measurement of
competence without the physiological consequences of fusing fluorophores to
specific competence proteins.  By controlling the entry into the competence
state, I seek to quantify the frequency of DNA uptake into competence cells as
well as tuning the chemical and physical characteristics of the transferred DNA
to probe the limits of HGT via transformation (Fig.
\ref{fig:transformation_exp}). 


The concentration, type, and sequence diversity of free DNA can vary greatly
depending on the environment \cite{DellAnno:2002uw, DeFlaun:1987vw,
Matsui:2003hd}. For example, extracellular chromosomal DNA has been shown to be
an integral part of biofilms and can reach concentrations of several \textmu g
$\cdot$ mL$^{-1}$ \cite{Tang:2013kj}, grazing of protists on marine bacteria can
facilitate the release of plasmid DNA into solution 
\cite{Matsui:2003hd}, and multi-species biofilms contain a diverse pool of
extracellular DNA distinct from cellular DNA \cite{Steinberger:2005bg}. It is
likely that concentration, type, and homology of transformable DNA are important parameters to a
competent cell governing successful transformation. To
probe this dependence, I will use the \textit{parBS} visualization system to
quantify the extent of uptake with varying concentrations of DNA, the type
(plasmid, chromosomal, and linear), and the homology with the chromosome (Fig.
\ref{fig:transformation_exp} left panel). As described previously, I have
established the \textit{parBS} visualization system in \textit{B. subtilis}
strains where the competence regulator ComK is under control of an inducible
promoter. As a small proportion of the population enters the competence state
naturally, the throughput of these measurements can be increased by forcing all
cells into the competence state before adding the transforming DNA to
environment. This will allow me to image many more competent cells than relying
on the natural entry to competence. Once in the competent state
\textit{B. subtilis} halts its growth via action of ComGA \cite{Haijema:2001ui}.
Ensuring that all cells are in the competence state will prevent the formation
of microcolonies which can prove difficult to segment and process
computationally, further increasing the throughput of these measurements. In
addition to tuning the parameters of the transforming DNA, the extent of uptake
in the competence state is unknown. Are all cells who enter the competence state
capable of transformation? Traditional bulk assays provide insight
only on the total number of individuals in the population who successfully
transferred DNA, but provides no information on the proportion of the competence
cells that were successful. It is possible that although the cells develop the
ability to pick up DNA from the environment, not all may be lucky enough to
import DNA and successfully integrate it into their chromosome. Using single
cell microscopy, I will be able make such measurements by forcing the cells into
competence through genetic manipulation and providing \textit{parS} containing
DNA in the extracellular environment. By imaging many cells over a period of
many hours, I will be able to visualize how many of those that have entered
competence undergo HGT.

The sensitivity of this visualization system also allows us to make dynamic
measurements regarding the physiology of the competence state (Fig.
\ref{fig:transformation_exp} right panel). In addition to the inducible ComK
\textit{B. subtilis} strain, I have also employed the \textit{parBS}
visualization system in a strain with wild-type ComK regulatory architecture
along with the CFP competence reporter. With these strains, I will be able to
probe any relationship between successful transformation and duration of the
competence state in the wild-type context.  Upon entry into the competence
state, what is the average time until the first uptake occurs? Entry into the
competence state triggers a cascade of molecular events triggering the
expression and assembly of the massive T4P as well as the repression of
cytoplasmic restriction systems and the expression of the recombination
machinery. The amount of time necessary to be functionally competent is far
longer than it takes for the circuity to flip on. Through microscopy, I will
measure the time distribution from entry into the competence state to the first
appearance of puncta. The duration of the competence state is another parameter
of interest to those who study the systems biology.  \citet{Cagatay:2009ie} were
able to show that slight modifications of the competence regulatory circuit
drastically altered competence duration. While the behavior of the modified
circuit was highly predictable, the natural circuit architecture showed a much
wider distribution of duration times. This extended duration time appeared to
allow for a increase in transformation frequency when compared to the synthetic
circuit. While this difference was attributed to stochastic fluctuations in the
operation of the native circuit, it is possible that there may be some other
signal to exit the competence state that has escaped discovery. The presence of
DNA has never been reported to stimulate competence in any organism
\cite{Mell:2014dj}.  As DNA is ubiquitous in the environment
\cite{DellAnno:2002uw} and are an integral component in biofilms
\cite{Tang:2013kj}, this is not a surprising result. However it is not known
whether the uptake of DNA serves as a signal to exit from the competence state.
While difficult to answer using bulk methods or by using conventional
fluorescent reporters in single-cell studies, the answer is well within reach
using the \textit{parBS} visualization system (Fig.\ref{fig:transformation_exp}
right panel).Through time-lapse microscopy, I will be able to track the dwell time
from first successful uptake to exit from the competence state. Normal duration
of the competence state can be many tens of hours in the absence of DNA,
allowing many positions to be tracked using automated microscopy. Persistence in
the competence state is a metabolically expensive task, it may be wise to sense
when DNA has been successfully taken up and exit the competence state. However,
it is possible that the exact opposite phenomenon may occur. So long as DNA is
being taken up, there may be signals to remain in the competence state.  


The experiments described above will allow us to paint a more vivid picture of
natural transformation at the single cell level. I will be able to examine how
the properties of the transforming DNA modulates HGT efficiency. Additionally, I
will be able to gain insight into the physiological consequences of DNA uptake
in a well-characterized competent bacterium. Single cell microscopy is typically
low-throughput, however, complicating the detection and quantification of rare
events. Should the frequency of transformation be too low to measure in this
fashion, I will be able to increase throughput by using microfluidic flow-cells
as are described in \cite{Lambert:2015bk, Lambert:2014gc} and the
\citet{Wang:2010jz} "mother machine". Both of these devices allow for the perpetuating of
bacterial cultures in exponential phase for extended periods of time, allowing
screening of a larger number of cells than I would be able to using traditional
slide-mounting microscopy.


%Not well developed, may cut this out.
%Despite being possibly the worlds best-studied organism, the natural competence
%of \textit{Eschericha coli} is still enigmatic. Several \textit{E. coli} genes
%have been identified as homologs to competence-specific genes in other
%gram-negative bacteria \cite{Averhoff:2003ex, Chen:2004iy}. A "natural
%competence" state can be forced in \textit{E. coli} upon expression of a handful
%of these homologs, allowing the use of DNA as a carbon
%source\cite{Palchevskiy:2006kqb, Finkel:2001ge}. While expressing the competence
%genes alone is sufficient to allow uptake of DNA, transformants never form,
%suggesting the lack of appropriate DNA processing machinery to ensure
%stabilization of the transferred fragment. Expressing the $\lambda$-red
%recombinase along with the competence homologs readily produced
%transformants\cite{Sinha:2012eha}. These studies suggest that \textit{E. coli}
%posses the ability to acquire exogenous DNA through natural transformation,
%although the processing machinery (if present) and the physiological signals
%that stimulate expression of the competence genes is unknown. Using the
%\textit{parBS} visualization system, coupled with what is learned from studying
%transformation in \textit{Bacillus subilis}, I would like to examine the
%frequency of DNA uptake in \textit{E. coli}. \textcolor{red}{I'm unsure about
%including this. It would be nice to see uptake occur in E. coli but it won't
%tell us anything about what triggers the competence state.}



\subsection*{Testing the limits of generalized and specialized transduction}
\indent Both conjugation and natural transformation rely on close proximity between the
DNA and accepting cell. To make matters worse, pilus attachment and the conjugal
bridges are highly unstable and subject to mechanical fracture. Free DNA in the
environment is highly susceptible to chemical decay meaning that mobility (both
spatial and temporal) is limited. Transduction, on the contrary, is a long-range
mechanism of HGT as DNA within the phage is well protected from harsh
environmental conditions. Transduction frequencies and efficiencies are
typically measured in bulk by infecting bacterial strains of interest with
transducing phage carrying a selectable marker and counting those who survive.
This methodology is inadequate to measure the frequency at which transductions
occur as cells who fall victim to superinfection or improper integration of the
selectable marker are not visible in plating. 

As described previously, phage P1 is a generalized-transducer and a commonly used
tool of molecular biology. Its ease-of-use and lytic nature makes it an
attractive target to test the limits of generalized transduction. The small size
of the \textit{parS} makes integration into various regions of the
\textit{E. coli} chromosome relatively trivial, allowing me to integrate this sequence
into five separate loci frequently used in molecular biology for recombinant expression. By
raising P1 bacteriophage from this host, I am able to generate transducing particles
occasionally containing the \textit{parS} binding site
(Fig. \ref{fig:transduction_exp} left panel).  By using these phage to
infect \textit{E. coli} expressing YGFP-ParB, I will be able to instantaneously
identify cells which have been transduced and track their probability of
survival.  In a similar manner to P1, the biology of bacteriophage $\lambda$ has
been the subject of intense study for decades. The molecular biology driving the
decision between lytic and lysogenic development as well as viral assembly is
well understood. As $\lambda$ integrates into the host genome, it functions as a
specialized transducer. In \textit{E. coli}, $\lambda$ integrates into the
\textit{att} locus adjacent to the \textit{gal} and \textit{bio} loci. By
integrating \textit{parS} into either or both of these sites, I will be able to 
track the rate at which $\lambda$ performs transduction (Fig.
\ref{fig:transduction_exp} right panel). 
%In cooperation with the
%$\lambda$ helper phage, bulk-measurement transduction frequencies can reach
%levels as high as 50\%. 
In addition to probing the wild-type ability of
transduction in $\lambda$, I can also test its ability to transduce a variety
of loci by moving the \textit{att} site to other locations around the genome.
One can imagine how movement of this site can influence architecture of the
surrounding genome. 

While phage $\lambda$ integrates into a specific site,
bacterial genes transferred through P1 generalized transduction have the ability
to be integrated into its native position in the chromosome, although it is
likely that not all transduced DNA safely finds a home and is eventually degraded. By
monitoring the persistence or disappearance of puncta, I will be able to measure
the frequency of integration versus degradation. In the context of conjugation,
\citet{Babic:2008bl} made a similar measurement where the stability of
conjugative transposons was found to be very high (97\% integration
probability). This parameter is unknown for the case of generalized transducing
phage. Integration probability is likely very high for specialized transducers
as the bacterial genes are associated with the phage genome and are recombined
alongside the prophage. Transduction events which are successfully blocked by
the bacterial cell would not appear in bulk measurements, suggesting a higher
frequency of recombination than it is in reality. 

%As the DNA is
%injected in a double stranded form, ParB molecules will readily bind and
%polymerize along the transduced DNA, allowing us to track it's movement from
%infection to recombination by tracking its diffusive motion. While amount of
%packaged DNA is large (90 kbp for P1 and 45 kbp for $\lambda$), it is still much
%smaller than the bacterial chromosome and is therefore more diffusive.
% A marked
%difference in the rate of diffusion of the puncta will indicate successful
%recombination events (Fig.  \ref{fig:transduction_exp} middle panel).

As is the case with quantifying transformation efficiencies, the low-throughput
nature of single cell microscopy will likely be an issue for studying
transduction. The transduction efficiency determined via bulk assays varies from
phage to phage and likely by integration site. The microfluidic device described
in \citet{Lambert:2015bk, Lambert:2014gc} will be especially useful for these
experiments as many cells can be imaged over a long period of time. However,
these methods will not allow time-lapse measurements of survival and colony
development, making assessment of survival probability difficult. In addition,
high-throughput methods will likely not allow us to measure the rate of
recombination as high temporal resolution is needed for such measurements.




%%%%%%%%%%%%%%%%%%%%%%%%%%%%%%%%%%%%%%%%%%%%%%%%%%%%%%%%%%%%%%%%%%%%%%%%%%%%%%%%%
\begin{figure}
	\centerline{\includegraphics[width=\textwidth]{figs/transduction.eps}}
	\caption{Probing the dynamics of transduction in phage P1 and $\lambda$. P1
	phage, a generalized transducer, is capable of packaging random portions
	of the host genome. To monitor generalized transduction (shown in left
	panel), \textit{parS} can be placed anywhere within the host genome. P1 phage
	can then be raised against these cells and used to infect YGFP-ParB
	expressing cells where puncta appearance indicates a transduction event.  Phage $\lambda$
	performs specialized transduction (left panel) in which the \textit{gal}
	and \textit{bio} loci are adjacent to the prophage. By integrating parS
	into these adjacent operons, I will be able to measure the frequency and
	positional dependence of transduction via $\lambda$.  The middle panel
	illustrates parameters and processes to be measured in both P1 and
	$\lambda$. The frequency of stable and unstable transfer events, the
	probability of chromosomal integration, and the positional dependence of
	transferred loci likely vary between generalized and specialized transduction.}
	\label{fig:transduction_exp}
\end{figure}
%%%%%%%%%%%%%%%%%%%%%%%%%%%%%%%%%%%%%%%%%%%%%%%%%%%%%%%%%%%%%%%%%%%%%%%%%%%%%%%%%


\subsection*{Probing the positional and functional dependence of recombination}

%%%%%%%%%%%%%%%%%%%%%%%%%%%%%%%%FIGURE%%%%%%%%%%%%%%%%%%%%%%%%%%%%%%%%%%%%%%%%%%%%
\begin{figure}
	\centerline{
		\includegraphics[width=\textwidth]{figs/position_function_Two.eps}}

	\caption{Probing the positional and functional dependence on HGT. Genes
		are often clustered along the bacterial genome with respect to
		their physiological function (left column). Genes at the
		interface of clusters may experience different levels of
		recombination with incoming genetic material. Through
		fluorescence microscopy and bioinformatic methods, I seek to
		probe the frequency of HGT events with respect to physiological
		function and position along the chromosome (top middle column).
		This will be performed by integrating a fluorescent reporter
		into various regions of the \textit{B. subtilis} chromosome and
		quantifying successful recombinants with  flow activated cell
		sorting. Bioinformatic methods (bottom middle column) can also
		be investigated to reveal the most frequently transferred genes
		and their next-door neighbors. Gene occupancy as a regulator of
		recombination can be tested experimentally (right column). By
		integrating a fluorescent reporter next to a gene where the
		expression is controlled, any relationship
		between transcriptional activity and recombination can be measured.}

	\label{ref:postion_function}

\end{figure}
%%%%%%%%%%%%%%%%%%%%%%%%%%%%%%%%%%%%%%%%%%%%%%%%%%%%%%%%%%%%%%%%%%%%%%%%%%%%%%%%%%%

\indent Through single-cell/single-molecule experimental study, I hope to build a
quantitative understanding of the frequency and dynamics of HGT in the context
of transformation and transduction. While this will provide insight into the
dynamics of different modes of transfer, a more detailed study is needed to
understand how HGT affects genome architecture. 

It is likely that successful integration of horizontally acquired DNA is
dependent on the activity of the surrounding genes. Heterogeneity of integration
efficiency with respect to position in the \textit{B. subtilis} chromosome has
been reported numerous times \cite{Biswas:1992tt, Tomita:2014jn}, suggesting
the presence of unknown cellular signals which guide recombination. As
homologous recombination requires scanning of the DNA to find a region of
satisfactory homology, followed by coordinated excision and insertion of ssDNA,
it is likely that this process is highly sensitive to the protein occupancy of
the homologous region. As highly transcribed genes, such as rRNA, are stacked
nearly end-to-end with RNA polymerase, the recombination machinery is likely
unable to explore that region for homology. One can imagine that this phenomenon
may promote the clustering of important and highly transcribed genes, lowering
the probability of recombination and protecting the cell from disaster.

While useful for making dynamical measurements, using low-throughput methods to
gather statistics of rare events is a hurdle difficult to overcome. To measure
heterogeneity in gene transfer with respect to positioning along the chromosome,
a method capable of screening through large numbers of cells with a high degree
of sensitivity is needed. Fluorescence-activated cell-sorting (FACS) is high
throughput, capable of measuring and sorting 10$^5$ cells per second, and can
make quantitative fluorescence measurements with single-cell resolution.  I
propose to probe the frequency of integration of a traditional fluorescent
reporter (such as YFP) into regions of the chromosome with a wide array of
transcriptional activity and quantifying the number of successful integrations
through FACS. This will require cloning an array of integration vectors which do
not create potentially fatal gene knockouts, but integrate in between various
loci. As a large degree of homology is needed for efficient transformation in
\textit{B. subtilis}, the design of this homology will likely need to be
performed manually. This unfortunately limits the number of integration sites
that can be investigated. However, the expression profiles of many genes in
\textit{B. subtilis} under a variety of conditions is known
\cite{Buescher:2012ih, Michna:2013jj, Salvetti:2011bs, Carpentier:2005ip}. Using
this wealth of information, I will specifically choose to monitor integration
into regions near a range of genes with variable expression in the competence
state. The level of expression of the nearby genes can also be tested through
RNA seq and fluorescence measurements.

This hypothesis gene occupancy driven recombination can also be tested through
synthetic methods. We are able to tune expression strength of bacterial
genes through single base-pair changes within the RNA polymerase binding site
\cite{Brewster:2012jw}, allowing modulation the expression level of
naturally occurring and recombinant genes in various bacteria. Using a range of
promoters with varying strength, we are able to test how changing the expression
level of neighboring genes affects genetic recombination. In such an experiment,
promoters with varying strength can be used to modulate expression of a
fluorescent reporter gene (such as mCherry). I will then try to integrate
another fluorescent reporter (YFP) adjacent to the mCherry reporter. The
insertion of the new YFP reporter would be detectable via FACS, even in the case
of low-frequency recombination. If gene occupancy affects the probability of
homologous recombination, one would expect to see an inverse relationship
between recombination frequency and promoter activity. However, it is also
possible that recombination is independent of occupancy. This would still be a
very surprising result as even highly-transcribed genes would then be
susceptible to homologous recombination. 




\section*{Concluding remarks}

\indent Modern advancements in genome-editing, high-throughput and single-molecule
microscopy techniques, long-read sequencing, and computational approaches to
model living systems have placed us in the auspicious position to query the
molecular processes that drive all of evolution. A multidisciplinary approach
using tools from evolutionary biology and ecology, the mathematical logic from
physics, and modern biological techniques holds the potential to advance our
understanding of the biophysical details of biology's greatest idea --
evolution. HGT is a fundamental and major driver of biological evolutionary
innovation and transformation across a wide temporal and phylogenetic scale.
However, to date, experimental work on HGT at the cellular level has been
limited. As a result, we lack an understanding of how HGT occurs in complex,
natural microbial populations and environments. Because of the disparity in
physiological relevance between the lab bench and the real-world, an
\textit{in vivo}, single-cell study of the frequency and dynamics of HGT via
conjugation, transformation, and transduction in bacteria is necessary in order
to accurately characterize how HGT occurs in naturally occurring populations and
communities. This cellular study of HGT will provide new details and information
about HGT and its role as a driver of evolution and will have application to
practical issues such as the acquisition of antibiotic resistance, the
development of novel industrial processes, and assessing the risk posed by
accidental release of genetically engineered organisms and vectors into the
environment. 


%%%%%%%%%%%%%%%%%%%%%%%BIBLIOGRAPHY BULLSHIT%%%%%%%%%%%%%%%%%%%%%%%%%%%%%%%%%%
\section*{References} \bibliographystyle{unsrtnat}
\bibliography{../../bib_files/library.bib} 
%%%%%%%%%%%%%%%%%%%%%%%%%%%%%%%%%%%%%%%%%%%%%%%%%%%%%%%%%%%%%%%%%%%%%%%%%%%%%%
\end{document}

