%%%%%%%%%%%%Preamble%%%%%%%%%%%%%%%%%%%%%%%%%%%%%%%%%%%%%%%%
\documentclass[letterpaper, 12pt]{article}                 %
\usepackage[margin=1in]{geometry} 	                   %
\usepackage{microtype}                                     %
\usepackage{amsmath}                                       %
\usepackage{graphicx}                                      %
\usepackage{xcolor}                                        %
\usepackage{verbatim}                                      %
\usepackage{fancyhdr}                                      %
\pagestyle{fancy} 				           %
	\lhead{Griffin Chure} 				   %
	\rhead{Physical Basis of Genomic Promiscuity}      %
%%%%%%%%%%%%%%%%%%%%%%%%%%%%%%%%%%%%%%%%%%%%%%%%%%%%%%%%%%%%



%%%%%%%%%%%%%TITLE INFORMATION%%%%%%%%%%%%%%%%%%%%%%%%%%%%%%%%%%%%%%%%%%%%%%%%%
\title{Interkingdom Signaling in the Symbiosome}
\author{Griffin Chure -- Out-of-Field Research Proposal\\
	\small \textit{California Institute of Technology - Biochemistry and
	Molecular Biophysics}}
\date{\small Draft -- \today}

%%%%%%%%%%%%%%%%DOCUMENT%%%%%%%%%%%%%%%%%%%%%%%%%%%%%%%%%%%%%%%%%%%%%%%%%%%%%%%
\begin{document}
\maketitle

\abstract{Many species of plants rely on the metabolic action of soil microbiota
to convert atmospheric dinitrogen into fixed ammonia. In return, some species of
plants, especially the legumes, house these microbes in complicated structures
known as root nodules. Within the root nodules, specific species of bacteria are
internalized into the legume cytoplasm and form stable pseudo-organelles known
as the symbiosome. Both plant and bacterial proteins are found within the
symbiosome membrane suggesting extensive communication between the microbes and
the plant cell. XXXXXXXXX Something about methods XXXXXXX}



\section*{Introduction}
\subsection*{The need for symbiosis}
\subsection*{Development of the symbiosome}
\subsection*{Anatomy of the symbiosome}
\subsection*{Proteins of the symbiosome membrane}
\subsection*{Metabolic role of the symbiosomal space}

\section*{Experimental approach}
\subsection*{The role of Ca$^{2+}$ sensing and signaling in the symbiosomal space}
\subsection*{DNA content of the symbiosomal space}
\subsection*{Marking the destruction of the symbiosome.}

	\section*{Concluding Remarks.}
\section*{References}
\bibliographystyle{abbrv}
\bibliography{../../bib_files/library.bib}
\end{document}
