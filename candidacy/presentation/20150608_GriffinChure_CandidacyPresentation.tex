%Preamble shit------------------------------------------------------------------ 
\input{/Users/gchure/tex_tools/beamer_default.tex}
\usepackage{multimedia}
\usepackage{xltxtra}
\usepackage{polyglossia}


%------------------------TITLE SHIT--------------------------------------------- 
\title{The Mobile Genome: Probing Horizontal Gene Transfer at Single-Cell
Resolution}
\author{Griffin Chure - Rob Phillips Group - Candidacy Exam}
\date{June 8, 2015}

\begin{document}

%------------------------------------------------------------------------------- 
\begin{frame}[fragile]
	\titlepage
\end{frame}
%------------------------------------------------------------------------------- 
%-------------------------SECTION I--------------------------------------------- 
%------------------------------------------------------------------------------- 
\section{How do genomes evolve?}

%------------------------------------------------------------------------------- 
\begin{frame}
	\frametitle{Genes flow in two directions.}
	\centerline{\includegraphics[height=0.7\textheight]{figs/transfer_comparison.eps}}
\end{frame}

%------------------------------------------------------------------------------- 
\begin{frame}[fragile]
\frametitle{Why take the risk?}

\begin{columns}[T]
	\begin{column}{0.48\textwidth}
		\includegraphics[width=\textwidth]{figs/kloesges_et_al.png}\\
		\tiny{Kloesges et al. \textit{Mol. Biol. Evol.} 28(2) 2010}
	\end{column}
\pause
	\begin{column}{0.48\textwidth}
	Foreign DNA may be taken up to \ldots 
		\begin{itemize}	
			\item be used as a carbon source.
				\vspace{-1em}\begin{itemize}
					\item \tiny{DNA-subsisting bacteria.}
				\end{itemize}
			\item acquire new traits.
				\begin{itemize}
					\item \tiny{Antibiotic resistance transfer.}
				\end{itemize}
			\item use as a repair template.
				\begin{itemize}
					\item \tiny{Single-species biofilms.}
				\end{itemize}
		\end{itemize}
	\end{column}
\end{columns}


\end{frame}

%------------------------------------------------------------------------------- 
\begin{frame}
	\frametitle{HGT occurs through three main mechanisms}
	\centerline{\includegraphics[width=0.7\textwidth]{figs/mechanisms_of_transfer.png}}
\end{frame}

%------------------------------------------------------------------------------- 
%---------------------SECTION II------------------------------------------------ 
%------------------------------------------------------------------------------- 

\section{What happens at the cellular level?}

% -------------------------------------------------------------------------------
\begin{frame}
	\frametitle{Bulk conditions are far from reality.}

\begin{columns}[T]
	\begin{column}{0.48\textwidth}
		\centering{\includegraphics[width=0.8\textwidth]{figs/staph_biofilm.png}}\\
		\tiny{Courtesy of CDC/ Rodney M. Donlan}
	\end{column}
	
	\begin{column}{0.48\textwidth}
	\begin{itemize}
		\item Life is not always well-mixed.
		\begin{itemize}
			\item Environmental microbes thrive as biofilms
		\end{itemize}

		\item Plating only reveals the ``winners''.
			\begin{itemize}
				\item ``Losers'' may die
				\item 1 colony $\neq$ 1 transformant!
			\end{itemize} 
		\item Single-cell methods allow the measurement of dynamics.
	\end{itemize}
\end{column}
\end{columns}
\end{frame}

% -------------------------------------------------------------------------------
\begin{frame}
	\frametitle{How can we label DNA \emph{in vivo}?}
	\centerline{\includegraphics[height=0.8\textheight]{figs/par_function_one.png}}
\end{frame}

%------------------------------------------------------------------------------- 
\begin{frame}
	\frametitle{How can we label DNA \emph{in vivo}?}
	\centerline{\includegraphics[height=0.8\textheight]{figs/par_function_two.png}}
\end{frame}

%------------------------------------------------------------------------------- 
\begin{frame}
	\frametitle{How can we label DNA \emph{in vivo}?}
	\centerline{\includegraphics[height=0.7\textheight]{figs/parB_system.png}}
\end{frame}
%------------------------------------------------------------------------------- 
\begin{frame}
	\frametitle{Puncta distinction is sensitive to copy number.}

	\centerline{\includegraphics[height=0.8\textheight]{figs/example_cells.png}}
\end{frame}

%------------------------------------------------------------------------------- 
\begin{frame}
	\frametitle{Plasmid diffusion can be measured in real time.}
\end{frame}	

%------------------------------------------------------------------------------- 
\begin{frame}
	\frametitle{Plasmid diffusion can be measured in real time.}
\end{frame}	

%----------------------------QUESTION I----------------------------------------- 
\section{What are the limits of natural transformation?}
\begin{frame}
	\frametitle{\emph{Bacillus subtilis} is naturally competent}
	\begin{itemize}
		\item Competence machinery and circuitry is well-characterized.\\
		\item High-efficiency of genomic recombination.\\
		\item Entry into competence is \alert{genetically controllable.}\\
		\item Extent of DNA uptake is unknown.	\\
	\end{itemize}

\end{frame}

%------------------------------------------------------------------------------- 
\begin{frame}
	\frametitle{\emph{Bacillus subtilis} is naturally competent.}
	\centerline{\includegraphics[width=0.8\textwidth]{figs/bacillus_comG.png}}
\end{frame}
%------------------------------------------------------------------------------- 
\begin{frame}
	\frametitle{How does the DNA dictate transfer?}
	\centerline{\includegraphics[height=0.9\textheight]{figs/bacillus_transformation.eps}}
\end{frame}

%------------------------------------------------------------------------------- 
\begin{frame}
	\frametitle{Does DNA uptake influence the competence state?}
	\centerline{\includegraphics[height=0.8\textheight]{figs/bacillus_competence.eps}}
\end{frame}

%----------------------------QUESTION II---------------------------------------- 
\section{How frequent is transduction?}



%------------------------------------------------------------------------------- 
\begin{frame}
	\frametitle{Viral infections aren't always bad.}
	\centerline{\includegraphics[width=\textwidth]{figs/bacteriophage.png}}
\end{frame}

%------------------------------------------------------------------------------- 
\begin{frame}
	\frametitle{Transfer can be observed for ``winners'' and ``losers.''}
	\centerline{\includegraphics[width=0.8\textwidth]{figs/transduction_simple.eps}}
\end{frame}

%------------------------------------------------------------------------------- 
\begin{frame}
	\frametitle{Transduction can be either general or specialized.}
	\textbf{Generalized transduction:} Packaging of bacterial genes is
	\alert{random} and transducing phage contain \alert{no viral DNA.}\\
	\vspace*{1em}
	\textbf{Specialized transduction:} Packaging of bacterial genes is
	\alert{restricted} to those adjacent to prophage integration site and
	transducing phage also contain \alert{viral DNA}.
\end{frame}
%------------------------------------------------------------------------------- 
\begin{frame}
	\frametitle{How does transduction differ between P1 and $\lambda$?}
	\centerline{\includegraphics[height=0.8\textheight]{figs/transduction_detailed.eps}}
\end{frame}

%------------------------------------------------------------------------------- 

%----------------------------QUESTION III--------------------------------------- 
\section{Does HGT care about position?}

%------------------------------------------------------------------------------- 
\begin{frame}
	\frametitle{Genes are not evenly distributed.}
	\centerline{\includegraphics[width=0.8\textwidth]{figs/position_function.png}}
\end{frame}

%------------------------------------------------------------------------------- 
\begin{frame}
	\frametitle{Does the genomic neighborhood influence recombination?}
	\centerline{\includegraphics[width=0.9\textwidth]{figs/recombination_figure.eps}}
\end{frame}

%----------------------------SUMMARY-------------------------------------------- 
\section{In conclusion\ldots}

%------------------------------------------------------------------------------- 
\begin{frame}
	\frametitle{How do genomes evolve?}
	\begin{center}\alert{What is the frequency and dynamics of HGT at single-cell resolution and what can
	it tell us about complex microbial populations and
communities?}\end{center}
\begin{itemize}
	\item What are the limits of natural transformation?
		\begin{itemize}
			\item Importance of concentration, size, and homology?
			\item Importance in competence regulation?
		\end{itemize}
	\item How frequent is viral transduction?
		\begin{itemize}
			\item Transduction frequencies of P1 and $\lambda$?
			\item Positional dependence of transferred loci?
		\end{itemize}
	\item Does HGT influence genome architecture?
		\begin{itemize}
			\item Gene occupancy and recombination efficiency?
		\end{itemize}
\end{itemize}

\end{frame}
%--------------------------THANKS----------------------------------------------- 
\begin{frame}
	\frametitle{Thanks to\ldots}
\begin{itemize}
	\item \textbf{Phillips Lab} \\
			  \alert{Rob Brewster}, \alert{Jacob Shenker},
				\alert{Robin Brown}, \alert{Rob Phillips},
				Stephanie Barnes, Nathan Belliveau,
				Tal Einav, Soichi
				Hirokawa, Bill Ireland, Heun Jin Lee, Gita
				Mahmoudabadi, Muir Morrison, Manuel Razo 

	\item \textbf{Others}\\
	\indent	Shimon Weiss (UCLA), Adam Rosenthal (Elowitz Lab)
	\item \textbf{Funding}\\
		NIH Training Grant  (T32GM07616)
\end{itemize}
\end{frame}

%------------------------------------------------------------------------------- 
\plain{Questions?}

%------------------------------------------------------------------------------- 

%-----------------------EXTRA SLIDES-------------------------------------------- 
\begin{frame}
	\frametitle{Single-cell assays reveal hidden behavior.}
	\centerline{\includegraphics[width=1.1\textwidth]{figs/single_cell_example.eps}}
\end{frame}

%------------------------------------------------------------------------------- 
\begin{frame}
	\frametitle{Competence in \emph{B. subtilis} is controlled by
	{\small com}K.}
	\centerline{\includegraphics[width=0.9\textwidth]{figs/bacillus_comp_circuit.eps}}
\end{frame}

%------------------------------------------------------------------------------- 
\begin{frame}
	\frametitle{The Lambert-Kussel flow cell can increase throughput.}
	\begin{columns}[T]
		\begin{column}{0.48\textwidth}
			\centerline{\includegraphics[width=\textwidth]{figs/lambert-kussel}}
			{\tiny Lambert and Kussel. \textit{PLoS Genetics.}
		10(9) 2014}
	\end{column}
	\begin{column}{0.48\textwidth}
		\begin{itemize}
			\item DNA containing media is flowed through chamber.
				\begin{itemize}
					\item Media exchange with wells is rapid
						(~250ms for complete exchange).
				\end{itemize}
			\item Many hundreds of cells can be observed
				over extended time periods.
				\begin{itemize}
					\item Allows for lineage analysis.
				\end{itemize}
			\item Does not sacrifice physiological relevance.
				\begin{itemize}
					\item Cells are closely packed as in
						biofilms and colonies.
				\end{itemize}
		\end{itemize}
	\end{column}
\end{columns}
\end{frame}



%------------------------------------------------------------------------------- 
\begin{frame}
	\frametitle{Some P{\small ar}B homologs mislocalize in \emph{E. coli}.}


\end{frame}



%------------------------------------------------------------------------------- 
\end{document}


