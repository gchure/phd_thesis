%%%%%%%%%%%%Preamble%%%%%%%%%%%%%%%%%%%%%%%%%%%%%%%%%%%%%%%%
\documentclass[letterpaper, 12pt]{article}                 %
\usepackage[margin=1in]{geometry} 	                   %
\usepackage{microtype}                                     %
\usepackage{amsmath}                                       %
\usepackage{graphicx}                                      %
\usepackage{xcolor}                                        %
\usepackage{verbatim}                                      %
\usepackage{fancyhdr}                                      %
\pagestyle{fancy} 				           %
	\lhead{Griffin Chure} 				   %
	\rhead{Physical Basis of Genomic Promiscuity}      %
%%%%%%%%%%%%%%%%%%%%%%%%%%%%%%%%%%%%%%%%%%%%%%%%%%%%%%%%%%%%

\usepackage{textcomp}

\setcitestyle{square}
\title{Probing the Physical Basis of Genomic Promiscuity in Bacterial Evolution}
\author{Griffin Chure - Research Candidacy Proposal\\
	\textit{\small California Institute of Technology - Biochemistry and Molecular
	Biophysics}}
\date{\small Draft -- \today}

\begin{document}
\maketitle
\begin{abstract}
	Bacterial populations live in a fertile soup of foreign genetic
	material. Whether by viral infection, conjugative transfer, or
	direct uptake of DNA through transformation, bacteria frequently share
	their genetic information with their neighbors, including those of
	different species, strongly influencing bacterial evolution. In this
	research, I seek to characterize and quantify the frequency and dynamics
	of the horizontal transfer of genetic material at single-cell
	resolution. I have employed a genetically encoded system in \textit{E.
	coli} and \textit{Bacillus subtilis} that allows for the observation of
	specific DNA molecules as they are transfered between individual cells.
	The orthogonal nature of the visualization system allows for \textit{in
	vivo} studies. In addition, this research will advance our understanding of the
	evolutionary consequences horizontal gene transfer on the formation of
	microbial communities. 
\end{abstract}

\section*{Introduction}
Understanding the ebb and flow of genetic information through lineages and
communities has long been the focal point of biological inquiry.  

The genome, however, is not limited only to travel through a family tree.
Recent advancements in sequencing and molecular biological techniques have shown
that vertical transmission of genetic material from parent to daughter cell is
not the whole story. 


Horizontal Gene Transfer (HGT) is the transmission of genetic material, whether
it be informational or operational, from one organism to another through means
other than sexual or asexual reproduction. In essence, it is the incorporation
of foreign genetic information into your own. 

HGT allows for an immediate variability in genotype not possible only through
mutation and descent with modification. Antibiotic resistance cassettes are the
most cited examples of HGT, although the range of genetic information that can
(and has) been transferred is vast and it's role evolution is still not well
understood.


HGT is not limited to the prokaryotic domain and has been shown to occur in the
eukaryotes from the transfer of XXXX from bacterial hosts to plants (CITE) to
the transfer of photosynthetic genes from cyanobacteria to slug somatic cells.

The role that HGT plays in the evolution of prokaryotes is both more dramatic
and abundant than in the eukaryota. The short generation times, physical size,
and modern molecular-biological tool set make bacteria the model organisms for
studying not only the molecular details of how the DNA is transported into the
cell but for a quantitative dissection of the fitness effects such transfers can
cause.

	Explain that the genetic constructs built in the lab could get
		into the environment and permanently imprint themselves on the
		genome of nature. 

\subsection*{Conjugation}
Through the early 1940's, our understanding of bacterial phylogeny and
prokaryotic inheritance as a whole had been limited to the accumulation of
mutations through vertical transmission of genetic information from cell to
cell. While Frederick Griffith, Avery MacLeod, and others had shown that
bacteria could be ``transformed" into pathogenicity by
DNA\cite{Griffith:1928vg, Avery:2014wx} there had been little to no
investigation into its affect on bacterial evolution. The first evidence of
genetic recombination in bacterial specimens came from the work of Tatum and
Lederberg in 1947 which showed that bacteria can transfer information with their
neighbors by entering a ``sexual stage"\cite{Tatum:1947va}.

In the 70 years since the discovery, the molecular details of bacterial
conjugation have been studied extensively. 
\begin{itemize}
	\item Fertility element (HFR) encodes all genes necessary for transfer.
		This is also the genetic element that gets transfered. This
		means that so long as it successfully transfers in it entirety,
		the receiving cell will become able to pass along the plasmid. 
	\item The plasmid has a specific site \textit{oriT} where the transfer
		and begins and ends. It is at this site that relaxase proteins
		nick the plasmid allowing a single strand to be passed through
		the pilus.
	\item HFR-HFR transfers are rare or do not occur at all (NEED CITATION).
		This indicates that there is some molecular marker which HFR
		cells can identify viable targets.
	\item Primary mechanism for horizontal gene transfer. This is very
		short-range, however, as the cells need to be close enough to
		touch each other. This means that conjugal transfer in biofilms
		is expected and incredibly common. 
	\item Because a F element is needed with a specific start and end site,
		cells are limited in what genes can be transfered. Unlike in
		transformation and transduction, it is very specific in it's transfer. 
	\item ICEs are mobile elements which have been integrated into the
		chromosome and have the ability to be transfered via
		conjugation, but do not necessarily encode the conjugation
		machinery. 
	\item For mobile elements to be self-transmissible, they must contain
		information for maintenance, dissemination, and regulation
		\cite{Burrus:2004ca}. Unlike plasmids, integration into the
		chromosome of it's host ensures vertical transmission into the
		hosts' progeny. The probability of loss is much lower than in
		the case of plasmids which are autonomous elements in most
		cases. Most ICEs contain a recombinase which integrates into a
		specific locus on the host genome (\textit{attB}) and a locus on
		the ICE (\textit{attP})\cite{Burrus:2004ca}.
	\item ICE frequently contain another factor, excisionase (\textit{Xis}) which is
		responsible for the excision of the ICE from the host chromosome
		after integration. WHAT IS THE FUNCTION EXACTLY?
	\item Capable of transferring an entire genome into a recipient. This is
		dependent on the stability of the cell suspension as the pilus
		connection is fragile and susceptible to mechanical stress.

	\item  DNA is most frequently transferred between cells during
		conjugation as a single strand. This requires four components -
		the relaxase for nicking and binding of the ssDNA, a coupling
		protein which binds and transfers the ssDNA between the donor
		membranes, and a type IV seqcrion system (T4SS) which establish
		the contact between the two cells and serves as a conduit for
		the ssDNA to transfer (CHECK) \cite{Guglielmini:2014bc}. As T4SS
		are highly diverse structures and are widely distributed across
		bacterial species, not all mobile elements encode T4SS and use
		them \textit{in trans}.
	\item As cell-cell contact is necessary, this type of horizontal gene
		transfer is important in dense microbial communities, such as in
		biofilms. This transfer is very short range, unlike
		transformation and transduction.
\end{itemize}

Many of the experiments which characterized this system were done either \textit{in
vitro} or in bulk, leaving many questions regarding the cellular and molecular
events unanswered. Such questions, such as the frequency of conjugation in a
mixed population of donors and recipients still remain unanswered.  

There have been some \textit{in vivo} which have helped shed light on how the
conjugal elements traverse bacterial communities. Babi\'{c} et al., 2008 used
real-time fluorescence microscopy to measure the frequency of conjugative
transfer within a mixed population of donors and acceptors. To visualize the
horizontally transferred DNA, the authors created a SeqA-YFP fusion which
specifically and strongly binds hemimethylated DNA. DNA from an F+ Dam+
\textit{E. coli} strain was mixed with an F- Dam- recipient. The
methylated DNA transferred into the recipient was bound by SeqA-YFP upon
formation of the hemimethylated duplex forming a bright foci within the cell.
This allowed for observation of transfer of any DNA sequence. While this proved
useful for measuring the frequency of transfer, it provided little-to-no
information regarding what was transferred. The authors observed transfer
happening over a very short period of time after incubation and up to great
distances apart (with a maximum of 12$\mu$m), proving that DNA can be
transferred through the F pilus directly.

This shows that single-molecule and single-cell approaches to studying
horizontal gene transfer can provide deep insight into both the molecular
mechanisms as well as the frequency of transfer \cite{Babic:2008bl}.


\subsection*{Transduction}
Bacteriophages are the most numerous forms of life on the planet. The global
population is estimated to be on the order of 10$^{31}$ phage
particles \cite{Comeau:2008jo}. This population is estimated to perform 2$\times
10^{16}$ transduction events per second \cite{ChibaniChennoufi:2004gv}.


Sequence based methods are often used to infer phage-mediated gene transfer.
Genomic islands in an array of marine cyanobacteria are flanked by viral-like
sequences, suggesting they were transferred through a transduction event
relatively recently in their evolution\cite{Anonymous:2004vg, Palenik:2003ef}.

The horizontal transfered can also have a profound effect on the evolution of
viral genomes. While initially considered an oddity and attributed to
contamination of free bacterial DNA, metagenomic and genotyping studies showed
that a myriad of cellular genes have been incorporated into bacteriophage
genomes such as the photosystem II core reaction center and electron transfer
genes\cite{Anonymous:2004vg,Comeau:2008jo}.  This abundance of cellular genes
buried within the phage genomes emphasizes the power phage can have in
influencing the metabolic repertoire of their host. 

The benefit is not only felt by the host bacterium, however. A body of evidence 
\subsection*{Transformation}
DNA is one of the largest and most hydrophillic biomolecules in nature and it's
transfer from the environment into the crowded environment of the cytoplasm is
no easy task. After traversing the rigid cell wall and the hydrophobic barrier
of the cell membrane, the DNA must evade the myriad of restriction systems
evolved to degrade the potentially hazardous foreign material. 


\begin{itemize}
	\item Natural transformation allows for gene transfer between organisms
		who are not in the same host range for bacteriophage
		(transduction) or conjugation.
	\item Natural transformation is widespread in both gram negative and
		gram positive bacteria despite the profound structural
		differences in the cell boundary.
	\item Gram positive and negative bacteria have evolved similar but
		mechanistically different ways in which DNA is brought in from
		the extracellular milieu.

	\item \textit{Neisseria} and \textit{Pasteurellacae} preferentially uptake
		DNA from related genera. This is achieved by recognizing a
		specific uptake sequence that is similar to repeats present in
		the respective genomes of each. How this sequence is recognized
		still remains unknown. These species are the special cases
	\item Taking up the wrong sequences can shift the specificity landscape\cite{Mell:2014dj}?

\end{itemize}

\subsection*{Special Cases}
While conjugation, transformation, and transduction are the primary ways which
DNA is horizontally transfered, there are other mechanisms which appear to
facilitate HGT.
\begin{itemize}
	\item Gene Transfer Agents. Phage-derived elements. Capsid is too small
		to accommodate it's own equipment and has evolved to transfer
		other bits of DNA. 
	\item Why is it released up on death of the organism or stress or
		something? I don't entirely remember.
\end{itemize}


\subsection{HGT in Eukaryotes}
\begin{itemize}
	\item Prevalent in the plant world. Conjugation between bacteria and
		plant cells has been observed.
	\item Give example of cancer cells
	\item Entire nuclear genomes can be transferred between plants after
		grafting of roots. This has been experimentally shown to lead
		to the evolution of novel species \cite{Fuentes:2014kr}. 
	
	\item  Give example of slug transfer with photosynthetic genes.
\end{itemize}
\subsection*{Evolutionary implications of HGT}

The ubiquity and evolutionary importance of HGT is becoming increasingly
apparent as more complete genomes are sequenced and made publicly available. As
the microbial world is not limited only to the set of genes present in its own
lineage, the "species tree" concept becomes difficult to interpret. From a
mathematical framework, trees are defined as a framework in which there are no
reticulations between nodes in a given branch. This type of model is not
applicable to the microbial world where genes can flow in multiple dimensions.

For nearly the past 60 years, the uniformity of the
genetic code amongst all life was described by one of two models. The
stereochemical model suggested a deterministic relationship between the chemical
properties of the codon and it's amino acid \cite{Grafstein:1983wv} whereas the
"frozen accident" model suggested that the genetic code could be arbitrary and
was universally adopted as all life evolved from the last universal common
ancestor\cite{Crick:1968wg}. There is a third possibility, however, that HGT
exerted a selective pressure to allow the exchange of genetic information
between species \cite{Syvanen:1985vv, Syvanen:2012jn}, forcing all life to adopt
and optimize the same (or at least very similar) genetic
code.\citet{Vestigian:2006va} demonstrated a unity between the stereochemical model
and HGT. They showed computationally that a pool of individuals connected by
avenues for HGT lead to the unification and optimization of a standard genetic
code. 

While HGT has received much experimental and theoretical attention over the past
century, there are still gaping holes in our understanding of the deeper
implications and influence on evolution as whole. Generating a complete picture
of how genetic information is passed between organisms (both through vertical
and horizontal) transmission is imperative not only to our understanding of the
origin and diversification of microbes, but of the entire biosphere.


\begin{itemize}
	\item Destruction of the 'tree' model of life. A network is more
		appropriate, although computationally intractable.
	\item Promoted the use of the universal genetic code? There is an
		immense selective pressure of translating a library than
		learning the language. 
	\item Exacerbates the "species problem" of microbes. I should give the
		example that an alignment of all of the 70 \textit{E. coli}
		genomes reveals only something like 700 genes in common. Why is
		there so much variation in the genome? If the genomes of the
		entire environment is available how can you define differences?
	\item Generating a complete picture of how genes are passed around (and
		how frequently they do it) is imperative to the understanding of
		the evolution of not only microbes, but of the entire biosphere. 
\end{itemize}

HGT is particularly studied by looking for the transfer of operational or
protein-coding sequences.Horizontal transfer of genetic information is not
limited to genes.  Regulatory sequences can be exchanged\cite{Oren:2014eea}
leading to new and interesting patterns of regulation which can have an affect
on the composition of the environment that are subtle but important.

Horizontal transfer has the potential to drastically change the physiology of
the organism not only through the transfer of informational and operational
genes, but can rewire extant genetic circuits.

\section*{Experimental Approach}
\subsection*{Watching DNA flow in real-time}

Measurements of transformation or transduction frequency has been performed
almost entirely at the bulk scale, leaving many questions about what happens at
the cellular level unanswered. In the case of bulk experiments, only the
"winners" are actually measured as unstable or fatal transfer events are
unobserved.Fluctuations at the scale that are relevant to environmental
communities  (10$^2$ - 10$^3$ cells) [I need to find a citation for this] are
drowned out when doing the experiments with saturated cultures (10$^9$ cells).
Getting a picture at the single-cell level will greatly advance our
understanding of how frequently genes are passed around and will let us quantify
the effect it has on the development of communities. The effects felt by HGT
begin with a single cell.  Being able to watch the first several cell divisions
will help us measure these effects.


\begin{itemize}
	\item With all of the evidence of HGT, it's often impossible to
		understand how it moved around relying on sequence alone. 
	\item Bulk-scale studies are almost certainly and underestimate of the
		frequency. 	
	\item Classical fluorescent reporters are useful but rely on a large
		series of complex events to be detected. They must be properly
		integrate, transcribed, translated, and then properly mature
		before they are detectable. This puts a limit on the temporal
		resolution although gives a readout of Whether the genes are
		expressed or not. It would be great to watch the DNA travel
		inside of cells and between cell in an orthogonal manner would
		be fucking awesome and abolish the problems of traditional reporters. 
	\item Labs in the past [CITE some of Pogliano's work] rely on large
		arrays of repressor binding sites. These often result in immense
		portions of the DNA being dedicated to the visualization. This
		is a large perturbation. To really understand the dynamics of
		horizontal gene transfer, you need a system that perturbs the
		DNA in the most minimal manner. Ideally, this would be something
		that is genetically tractable and orthogonal to the cell. This
		seems like it's asking a lot, but we already have that. 
	\item ParABS system is used by low copy-number plasmids to ensure
		faithful and equal inheritance in dividing cells. This is
		composed of three partners. ParA is the atpase which drives the
		active segregation, ParB is the protein which binds to the short
		($\sim$100 bp) \textit{parS} DNA sequence. This binding is
		cooperative. Once ParB is bound to the DNA, the conformation
		changes prompting cooperative association of other ParB
		proteins. This results in an effect called 'spreading' where the
		ParB proteins bind to the DNA nonspecifically but tightly because
		of the cooperative nature of ParB. This results in a high
		concentration of the ParB proteins in a very small volume. By
		fluorescently tagging ParB, removing ParA, and cloning
		\textit{parS} into regions of interest, we can fluorescently tag
		DNA specifically and orthogonally with only minor sequence
		perturbations. This system serves as an "instantaneous
		reporter" that does not rely on any of the issues described
		above with canonical reporter systems.
	\item Explain and use a figure to show that in the absence of DNA, the
		cell is uniformly bright. When \textit{parS} is present, a
		bright focus appears. 
	\item I need make it clear what I have done on this system so far. I
		should explain how I tried to make the two color system work but
		decided a one-color system will allow for more simple
		measurements. I can describe how I've integrated into different
		positions and have tried to optimize expression to allow for
		detection of low copy number elements. Make it clear that I've
		tried using several different fluorophores, promoters, RBSs, etc
		to get this working. This system has been used to observe the
		spatial organization of the chromosome, but has never been
		used to watch the flux of gene elements through the cell. This
		has required more work than initially thought. 
	\item  Need to show image of this working in \textit{E. coli} with
		plasmids and integrated sites as well as in \textit{B. subtilis}
		with working chromosomal sites. 
\end{itemize}
\subsection*{Measuring frequency of natural transformation in genetically
tractable systems}
\begin{itemize}
	\item The ParBS system allows for measurement of gene transfer without
		having to rely on the expression of the transferred genetic material.
	\item I've been trying to observe the uptake of plasmid DNA during the
		run-of-the-mill molecular biology techniques of electroporation
		and transformation, but have run into issues. 
	\item I should show a figure the results of electroporation and
		transformation and how the formation of aggregates and inclusion
		bodies has confounded the situation. 
	\item List the strategies I have tried and am currently trying to
		develop to watch DNA transfer into cells. I need to explain what
		we will learn from this type of measurement (focusing again on
		the inaccuracy of bulk measurement methods). I can emphasize
		that chemical transformation happens in nature (cite the E Coli
		natural transformation paper). And that being able to understand
		the frequency of transfer is important physiologically and
		environmentally. 
	\item Due to the problems with \textit{E. coli} and its enigmatic
		competence state, I decided to switch gears to an organism who's
		competence state is more understood and is genetically
		tractable. 
	\item In collaboration with the Elowitz lab, I have (or will be at
		least) translated this system into \textit{Bacillus subtilis},
		who's competence state is well understood and is genetically
		tractable. 
	\item We know that in cultures grown in physiological conditions that
		only about 10\% of the cells are competent.  It is unknown how
		many of those cells actually take up DNA and successfully
		integrate the material into their genome. Using the ParBS
		system,  I am in the unique position to make this measurement.
		In \textit{Bacillus subtilis}, the competence state can be
		turned on using common molecular biology tools. I will be able
		to induce competence and watch for the flux of \textit{parS} DNA
		into the cell by watching for the formation of puncta. 
	\item I plan to also translate this system into other organisms whose
		competence states are more understood (such as
		\textit{Streptococcus pyogenes} and \textit{pneumoniae}) and see
		what kind of differences there are between the frequency of
		uptake and the frequency of entry to competence. I would then be
		interesting to study how DNA is transferred between species.
	\item Does uptake of DNA signal exit from the competence state? One
		would imagine some kind of receptor for uptake to avoid
		perpetually taking up DNA.	
	\item How does frequency of uptake correlate to entry into the comptence
		state? Will all competent cells eventually take up DNA or is
		there some sweet spot? 
	\item Does duration of the competence state depend on DNA uptake? Do
		cells which took up DNA have a shorter or longer competence
		state than cells who do not? Need to do time lapse and record
		pComG activity. 
	\item Using what I have learned from the well-characterized systems, I
		can take this back to \textit{E. coli} and resume these studies.
		Expression of TFoX (Sxy) coupled with $\lambda$-red
		recombination allows for transformation of \textit{E. coli}
		cells \cite{Sinha:2012eh}. Only one report of this phenomenon,
		as TFoX has no known induction conditions. How frequently can
		this actually occur? Using what I have learned working with a
		naturally competent organism, perhaps I will be able to shed
		light on the enigmatic competence state of \textit{E. coli}.
 
\end{itemize}

\subsection*{Observing transduction in phage P1 and $\lambda$}
\begin{itemize}
	\item Both conjugation and natural transformation rely on being within close proximity of the DNA. DNA doesn't last very
		long in natural environments and conjugation relies on
		bacteria being very close to each other spatially.
		Transduction is a long-range form of gene transfer.
		Viral elements are very stable in natural environments
		(Find a citation for this).
	\item Because of the persistence of viral particles, they have
		the opportunity to stay in the environment through many
		generations of bacteria. This can include cycles of
		succession and changes in the members of the population. 
		This means that genes that were advantageous (or harmful
		) generations ago could come back into play through
		transduction.
	\item The transduction efficiency varies from phage to phage. 
	\item I should discuss the different ways of packaging host
		genome (headful vs. any DNA vs. other mechanisms).
		Phage genomes are often several 10's of kilobases.
		This means that many genes, even entire operons and
		metabolic circuits, could be transferred horizontally
		between organisms from a single transduction event. 
	\item Like transformation, transduction has really only been
		studied in the bulk often relying on a single infection
		of a phage, proper integration of the selectable
		element, and then expression of the selectable element
		to generate a transductant. 
	\item We would learn much from observing transduction at the
		single cell level. We could get a better handle on the
		frequency of transduction as well as how many trasducing
		particles can infect a cell at once. Using the ParBS
		system, we could be able to watch the DNA as it is
		integrated into the genome by measuring the difference
		in diffusion of the puncta.
	\item Transduction often occurs at a low frequency, however,
		meaning we either need some way to scan through many,
		many cells and infections at once or we need to boost up
		the frequency of transduction.
	\item Phage P1 is a commonly used tool of molecular biology
		because it is very lax with the DNA that it packages
		into its capsid. This raises the efficiency of
		transduction to a measureable level. 
	\item The experimental procedure involves integrating the
		\textit{parS} sequence into several different loci in a
		host. These could be surrounding different metabolic
		genes, accessory genes,  core genes etc. By integrating
		many at the same time, we increase the probability of
		having a detectable locus packaged into the phage.
	\item We can then infect cells that are expressing the YGFP-ParB
		construct and watch for the formation of puncta. With a
		system where we can detect individual pieces of DNA, we
		can ask all sorts of interesting questions. How viral
		particles can transfer DNA at once? How frequently are
		there infections where the DNA is degraded before
		integration occurs? How rapidly after infection are
		foreign genes integrated into the genome? How does this
		integration frequency compare with the phage genome?
	\item Previous studies in the Phillips lab have tracked the
		ejection of phage DNA material into the cell. We can
		expand on that by tracking \textit{specific} bits of
		genetic material.

\end{itemize}
\subsection*{Probing the positional and/or functional dependence of HGT}
\begin{itemize}
	\item Can be performed by studying either and both transformation and transduction.
	\item Previous bioinformatic studies have shown that operational rather
		than informational genes are more prone to transfer [CITE]. This
		However this does not take into consideration the neighboring
		genes that were transfered.
	\item All modern bioinformatic approaches to studying HGT look for genes
		or the regions of the chromosome that have non-optimal codon
		usage, aberrant GC content (compared to the rest of the genome),
		or novel functions that are not common to close ancestral
		neighbors. No studies have looked at aberrant positioning of
		genes when compared to ancestral neighbors. Can genes
		'hitch-hike' on transfer events of junk or non-coding DNA or
		vice versa? 
	\item Placement of various parS sites to measure the rate of transfer.
		This can be done at bulk and at single cell level.
	\item High-throughput methods of microfluidics can help get a better
		handle of the frequency of locus-specific transfer.
	\item One could imagine that this may drive the evolution of the genome
		architecture. Would there be a benefit to be selfish and put
		beneficial genes next to targets that are likely to be
		incompatible with horizontal transfer?
\end{itemize}

\section*{Concluding remarks}
All of evolution is driven by molecular behavior within the cell. While there has
been much advancement in both evolutionary and cellular biology, there has been
remarkably little overlap between them. A multidisciplinary approach using tools
from evolutionary biology and ecology, the mathematical logic from physics, and
modern biological techniques holds the potential to advance our understanding of
the biophysical details of biology's greatest idea
– evolution. This research will advance our understanding of how DNA migrates
between bacterial cells and contributes to evolution of their populations. This
work holds promise for a deeper understanding of the generation of multiple
antibiotic resistant bacteria. In addition, many important industrial processes
rely on the faithful propagation of large microbial communities (e.g. biofuel
production) or the growth of genetically modified organisms. Understanding how
genetic information flows within communities and between species is important
for development of novel industrial processes as well as for assessing the risk
posed by accidental release of genetically engineered organisms into the
environment. Blah blah other stuff


\section*{References}
\bibliographystyle{plainnat}
\bibliography{../../bib_files/library.bib}
\end{document}
