\documentclass[letterpaper, 12pt]{article}
\usepackage[margin=1in]{geometry}
\usepackage{microtype}
\usepackage[T1]{fontenc}
\usepackage{mathpazo}
\usepackage{graphicx}
\usepackage{setspace}
\onehalfspacing

\begin{document}
\centerline{\large \textbf{The Luria-Delbr\"{u}ck Fluctuation Test --
Experimental Description}}

\centerline{\small Griffin Chure -- \today}
\vspace*{2em}
When mutations occur in nature, they are often deleterious to the organism.
However, mutations are a critical part of the genetic heritage of living
organisms, arising in every type of organism and allowing life to evolve and
adapt to new environments. In 1943, the question of how microorganisms acquire
mutations was described in a famous article by Salvador Luria and Max Delbr\"{u}ck
(S. E. Luria and M. Delbr\"{u}ck, {\em Genetics}, {\bf 28}, 491 -- 511, 1943). At the time, there
were two prominent theories of genetic inheritance. Scientists did not know if
mutations arose randomly in the absence of an environmental cue, the ``random
mutation hypothesis'', or whether they occur as an adaptive response to an
environmental stimulus, the ``adaptive immunity hypothesis.'' See Fig. 1.


In this experiment, we tested these two hypothesis by observing and quantifying
the rate by which the budding yeast \textit{S. cerevisiae} acquired resistance
to canavanine, a potent antibiotic for yeasts and many other fungi. Resistance
to this drug often occurs through a mutation in the \texttt{CAN1} region of the
\textit{S. cerevisae} genome which encodes a transporter through which
canavanine enters the cell. In addition to replicating the Luria-Delbr\"{u}ck
experiment, we sequenced this region of the resistant colonies to learn some
basic yet valuable bioinformatic skills.  


The experimental procedure is as follows and is described in the bottom panel of
Figure 1. Briefly, two strains of \textit{S. cerevisiae} (WT and MSH2) with two
different mutation rates (10$^{-10}$ and 10$^{-8}$ mutations/bp/generation
respectively) were grown for 24 hours to satuation. These cells were then
strongly diluted (1:10$^{5}$) into 96 separate wells of fresh growth medium.
These cultures were then allowed to grow for many generations. During this
period, some cells would be able to experience mutation under the random
mutation hypothesis allowing for the formation of ``jackpots'' where many cells
would be resistant to canavanine. After 12 -- 13 hours of growth, the cell densities
were calculated and 100$\mu$L aliquots were placed onto canavanine-containing
growth plates. After another 24 hours of growth, the resistant colonies were
counted. Each group's colony counts were combined into a single document which
was then provided to the students. The students were then able to generate
distributions of mutations for the WT and MSH2 strains and estimate the mutation
rate. By calculating the mean and Fano factor, the students were able to
determine whether mutation was adaptive or random.

\begin{figure}
	\centerline{\includegraphics[width=\textwidth]{figs/LD_figure.png}}
	\caption{\small \textbf{The Luria-Delbr\"{u}ck fluctuation test using
	{\em S. cerevisiae}}. The top two panels describe the two hypothesis for
	mutation given by Luria and Delbr\"{u}ck. In the adaptive immunity
hypothesis, cells (in this case budding yeast) will not undergo mutation
(indicated in red) until they are presented with some selective pressure. In the
random mutation hypothesis, however, cells have an equal probability of mutating
whether they are in the presence or absence of selective pressure. If a mutation
occurs early in a lineage, all descendants of that lineage will carry this
mutation as well. The bottom panel illustrates the experimental procedure used
to generate the accompanying data files. Two strains (WT and MSH2) of budding yeast with
different mutation rates (10$^{-10}$ and 10$^{-8}$ respectively) were used to
emphasize the variance of the distribution.}
\end{figure}
\end{document}

