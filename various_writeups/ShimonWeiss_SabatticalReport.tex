\documentclass[letterpaper, 12pt]{article}
\usepackage{microtype}
\usepackage[margin=0.5in]{geometry}
\usepackage{graphicx}
\usepackage{wrapfig}
\usepackage{epstopdf}
\begin{document}
\centerline{\textbf{Probing the Mobile Genome At Single-Cell Resolution}}\\
\centerline{\footnotesize Griffin Chure -- \textit{Biochemistry and Molecular Biophysics, California
Institute of Technology}}
\vspace*{0.8em}
One of the most fundamental aspects of biological inquiry is the relationship
between the sequence of the genome and the physiology of the organism. While we
are still trying to understand the sequence-structure-function relationship of
the genome, there is still much to be understood about how genomes are built and
organized. Whether by viral infection, conjugative transfer, or direct uptake of DNA
through transformation, bacteria frequently share their genetic material with
their neighbors (including those of different species) influencing the
architecture of the recipients' genome. How this horizontal transfer
of genetic information has influenced their evolution, however, remains
enigmatic. 

Using some clever tricks of molecular biology, we have engineered a system that
allows for sequence specific visualization of DNA molecules \textit{in vivo}.
Our system exploits the ParBS plasmid partitioning system to allow for the
subcellular localization and detection of plasmids in living cells without
relying on a secondary reporter associated with the plasmid of interest.   
Fluorescently labeled ParB molecules are constitutively expressed in the
cells of interest. Upon the uptake of DNA containing a \textit{parS} sequence,
ParB specifically binds and quickly oligomerizes allowing for a large number of
ParB molecules to be concentrated over a short stretch of DNA forming a
fluorescent puncta within the cell (Fig. 1A). The association between ParB and
\textit{parS} is rapid  and allows for instantaneous measurement of the uptake
of DNA.

Using this system, we sought to develop a more thorough understanding of the dynamics
and frequency at which DNA molecules are taken into \textit{E. coli} through
electroporation, a procedure commonly used in molecular biology. In this
procedure, \textit{E. coli} cells are washed repeatedly with deionized water and
are mixed with high concentrations of DNA. Upon an incubation on ice, the cells
are shocked with a strong electric field (1800
$\mathrm{V}\cdot\mathrm{cm}^{-1}$) which transiently introduces pores within the
membrane, allowing DNA molecules to enter. If this DNA molecule is an antibiotic
resistance conferring plasmid, cells that survive the treatment can propagate in
the face of new environmental pressures, such as the presence of antibiotic. The
efficiency of transformation is typically determined in bulk by counting the
number of resistant colonies that form after plating on selection
(transformants) as a function of the total number of cells that were exposed to
the electric shock. This frequency, however, is almost certainly an
underestimate of the true number of cells that managed to uptake DNA.

We found that a remarkably high percentage ($\sim$ 90\%) of the population dies
during the process of electroporation (Fig. 1B). It is very possible that the
limiting factor for a high efficiency of transformation is not the entry of DNA
into the cell itself, rather it is the poor survivability of the cells after
electroporation. Unfortunately, we found that upon cell death, cells expressing
the fluorescently labeled ParB protein form brightly fluorescent aggregates that
are difficult to differentiate from the puncta that indicate uptake of a
\textit{parS} containing plasmid. We attempted to counteract this aggregation by
making changes to the preparation of the electrocompetent cells, by lowering the
expression of reporter system, and adding excess salt and protein denaturant
guanidinium hydrochloride (GnHCl). While neither changes to the protocol nor
lowering the expression ameliorated the aggregation, the addition of 1M
guanidinium chloride to the surrounding medium after elecroporation prevented
the formation of aggregates, allowing cells to remain uniformly fluorescent
(Fig. 1C).  This concentration, however, prevented cell growth and lead to
senescence of all cells over several hours.  The concentration of GnHCl
necessary for aggregation prevention is likely less than the 1M tested.
Titration of GnHCl to a concentration that prevents aggregation but is not
cytotoxic is currently being determined. The research described above helped
initiate a line of research into the dynamics and frequency of DNA uptake
through natural transformation in \textit{Bacillus subtilis}, an organism famous
for its ability to uptake free DNA from the environment. The experimental methods
and techniques learned in this research have proved invaluable for the
experimental design and data analysis of this new line of investigation.


\begin{figure}
	\centerline{
		\includegraphics{./figs/sabattical_figure.eps}
	}
	\caption{\small Single-cell studies of transformation in
		\textit{E. coli}. The mechanism of the ParBS plasmid
		visualization system is described in \textbf{A}. The left hand
		image demonstrates the labeling of low copy-number plasmids
		(1 -- 5 per cell) in \textit{E. coli}. Panel \textbf{B}
		illustrates electroporation dependent death of \textit{E. coli}
		competent cells. Dead cells are determined by their inability to
		grow as well as by the presence of large, stationary, aggregates
		of fluorescently-labled ParB. These aggregates can be prevented
		by the addition of guanidinium hydrocloride, but not by high
		concentrations of salt, as can be seen in \textbf{C}.}
	\end{figure}
\end{document}
