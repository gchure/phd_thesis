\section{Discussion}
Allosteric regulation is often couched as ``biological action at a distance".
Despite extensive knowledge of protein structure and function, it remains
difficult to translate the coordinates of the atomic constituents of a
protein to the precise parameter values which define the functional response,
making each mutant its own intellectual adventure.
Bioinformatic approaches to understanding the sequence-structure relationship 
have permitted us to examine how the residues of allosteric proteins
evolve, revealing conserved regions which hint to their function.
Co-evolving residues reveal sectors of conserved
interactions which traverse the protein that act as the allosteric
communication channel between domains \cite{Suel2002, McLaughlin2012,
Reynolds2011}. Elucidating these sectors has advanced our understanding of
how distinct domains "talk" to one another and has permitted direct
engineering of allosteric responses into non-allosteric enzymes
\cite{Raman2014, Raman2016, Poelwijk2016}. Even so, we are left without a
quantitative understanding of how these admittedly complex networks set the
energetic difference between active and inactive states or how a given mutation
influences binding affinity. In this context, a biophysical model
in which the various parameters are intimately connected to the molecular
details can be of use and can lead to quantitative predictions of
the interplay between amino-acid identity and system-level response.

By considering how each parameter contributes to the observed change in free
energy, we are able to tease out different classes of parameter perturbations
which result in stereotyped responses to changing inducer concentration.
These characteristic changes to the free energy can be used as a diagnostic
tool to classify mutational effects. For example, we show in Fig.
\ref{fig:deltaF_theory} that modulating the inducer binding constants $K_A$
and $K_I$ results in non-monotonic free energy changes that are dependent on
the inducer concentration, a feature observed in the inducer binding mutants
examined in this work. Simply looking at the inferred $\Delta F$ as a function
of inducer concentration, which
requires no fitting of the biophysical parameters, indicates that $K_A$ and $K_I$ must be
modified considering those are the only parameters which can generate such a response.

Another key observation is that a perturbation to only $K_A$ and $K_I$ requires
that the $\Delta F = 0$ at $c = 0$. Deviations from this condition imply that more than
the inducer binding constants must have changed. If this shift in $\Delta F$ off of $0$ at
$c = 0$ is not constant across all inducer concentrations, we can surmise
that the energy difference between the allosteric states
$\Delta\varepsilon_{AI}$ must also be modified. We again see this effect for all
of our inducer mutants. By examining the inferred $\Delta F$, we can
immediately say that in addition to $K_A$ and $K_I$, $\Delta\varepsilon_{AI}$
must decrease relative to the wild-type value as $\Delta F > 0$ at $c = 0$.
When the allosteric parameters are fit to the induction profiles, we indeed
see that this is the case, with all four mutations decreasing the energy gap
between the active and inactive states. Two of these mutations, Q291R and
Q291K, make the inactive state of the repressor \textit{more} stable than the
active state, which is not the case for the wild-type
repressor \cite{Razo-Mejia2018}.

Our formulation of $\Delta F$ indicates that shifts away from $0$ that are
independent of the inducer concentration can only arise from changes to the
repressor copy number and/or DNA binding specificity, indicating that the
allosteric parameters are untouched. We see that for three mutations in the
DNA binding domain, $\Delta F$ is the same irrespective of the inducer
concentration. Measurements of $\Delta F$ for these mutants with repressor
copy numbers across three orders of magnitude yield approximately the same
value, revealing that $\Delta\varepsilon_{RA}$ is the sole parameter altered
via the mutations.

We note that the conclusions stated above can be qualitatively drawn without
resorting to fitting various parameters and measuring the goodness-of-fit.
Rather, the distinct behavior of $\Delta F$ is sufficient to determine which
parameters are changing. Here, these conclusions are quantitatively confirmed by
fitting these parameters to the induction profile, which results in accurate
predictions of the fold-change and $\Delta F$ for nearly every strain across 
different mutations, repressor copy numbers, and operator sequence, all at
different inducer concentrations. With a collection of evidence as to what
parameters are changing for single mutations, we put our model to the test
and drew predictions of how double mutants would behave both in terms of the
titration curve and free energy profile.

A hypothesis that arises from our formulation of $\Delta F$ is that a
simple summation of the energetic contribution of each mutation should be
sufficient to predict the double mutants (so long as they are in separate
domains). We find that such a calculation permits precise and accurate
predictions of the double mutant phenotypes, indicating that there are no
epistatic interactions between the mutations examined in this work. With an
expectation of what the free energy differences should be, epistatic
interactions could be understood by looking at how the measurements deviate
from the prediction. For example, if epistatic interactions exist which
appear as a systematic shift from the predicted $\Delta F$ 
independent of inducer concentration, one could conclude that DNA binding
energy is not equal to that of the single mutation in the DNA binding domain alone. Similarly,
systematic shifts that are dependent on the inducer concentration (i.e. not
constant) indicate that the allosteric parameters must be influenced. If the
expected difference in free energy is equal to $0$ when $c=0$, one could
surmise that the modified parameter must not be $\Delta\varepsilon_{AI}$ nor
$\Delta\varepsilon_{RA}$ as these would both result in a shift in leakiness,
indicating that $K_A$ and $K_I$ are further modified.

Ultimately, we present this work as a proof-of-principle for using
biophysical models to investigate how mutations influence the response of
allosteric systems. We emphasize that such a treatment allows one to boil
down the complex phenotypic responses of these systems to a single-parameter
description which is easily interpretable as a free energy. The general
utility of this approach is illustrated in Fig. \ref{fig:all_data_collapse}
where gene expression data from previous work \cite{Garcia2011, Brewster2014,
Razo-Mejia2018} along with all of the measurements presented in this work
collapse onto the master curve defined by \eqref{eq:collapse}. While our
model coarse grains many of the intricate details of transcriptional
regulation into two states (one in which the repressor is bound to the
promoter and one where it is not), it is sufficient to describe a swath
of regulatory scenarios. As discussed in the SI text, any
architecture in which the transcription-factor bound and transcriptionally
active states of the promoter can be separated into two distinct coarse-grained
states can be subjected to such an analysis. 

Given enough parametric knowledge of the system, it
becomes possible to examine how modifications to the parameters move the
physiological response along this reduced one-dimensional parameter space.
This approach offers a glimpse at how mutational effects can be described in terms of
energy rather than Hill coefficients and arbitrary prefactors. While we have
explored a very small region of sequence space in this work, coupling of this
approach with high-throughput sequencing-based methods to query a library of
mutations within the protein will shed light on the phenotypic landscape
centered at the wild-type sequence. Furthermore, pairing libraries of protein
and operator sequence mutants will provide insight as to how the protein and
regulatory sequence coevolve, a topic rich with opportunity for a dialogue
between theory and experiment.

\begin{figure}
    \centering
    \includegraphics{chapter_03_evolution/figs/Fig6.pdf}
    \caption[Data collapse of the simple repression regulatory architecture.]{
    All data are means of biological replicates. Where present, error bars
    correspond to the standard error of the mean of five to fifteen
    biological replicates. Red triangles indicate data from Garcia and
    Phillips \cite{Garcia2011} obtained by colorimetric assays. Blue squares
    are data from Brewster et al.\cite{Brewster2014} acquired from video
    microscopy. Green circles are data from Razo-Mejia et al.
    \cite{Razo-Mejia2018} obtained via flow cytometry. All other symbols
    correspond to the work presented here. An interactive version of this
    figure can be found on the
    \href{https://www.rpgroup.caltech.edu/mwc_mutants}{paper website} where the different data sets can be viewed in more detail.}
    \label{fig:all_data_collapse}
\end{figure}  