\section{Introduction}
Thermodynamic treatments of transcriptional regulation have been
fruitful in their ability to generate quantitative predictions of
gene expression as a function of a minimal set of physically meaningful
parameters \cite{Ackers1982, Buchler2003, Vilar2003, Garcia2011,
Daber2011a,Brewster2014, Weinert2014, Rydenfelt2014, Razo-Mejia2014,
Razo-Mejia2018, Bintu2005, Bintu2005a, Kuhlman2007}. These models
quantitatively describe numerous properties of input-output functions, such
as the leakiness, saturation, dynamic range, steepness of response, and
the [$EC_{50}$] -- the concentration of inducer at which the response is half
maximal. The mathematical forms of these phenotypic properties are couched in
terms of a minimal set of experimentally accessible variables, such as the
inducer concentration, transcription factor copy number, and the DNA sequence
of the binding site \cite{Razo-Mejia2018}. While the amino acid sequence of
the transcription factor is another controllable variable, it is
seldom implemented in quantitative terms considering mutations with subtle changes
in chemistry frequently yield unpredictable physiological
consequences. In this work, we examine how a series of
mutations in either the DNA binding or inducer binding domains of a
transcriptional repressor influence the values of the biophysical parameters
which govern its regulatory behavior.

We first present a theoretical framework for understanding how mutations in
the repressor affect different parameters and alter the free energy of the
system. The multi-dimensional parameter space of the aforementioned
thermodynamic models is highly degenerate with multiple combinations of
parameter values yielding the same phenotypic response. This degeneracy can
be subsumed into the free energy of the system, transforming the input-output
function into a one-dimensional description with the form of a Fermi function \cite{Swem2008,Keymer2006}.
 We find that the parameters capturing the allosteric nature of
the repressor, the repressor copy number, and the DNA binding specificity contribute
independently to the free energy of the system with different degrees of
sensitivity. Furthermore, changes restricted to one of these
three groups of parameters result in characteristic changes in the free
energy relative to the wild-type repressor, providing falsifiable
predictions of how different classes of mutations should behave.

Next, we test these descriptions experimentally using the well-characterized
transcriptional repressor of the \textit{lac} operon LacI in \textit{E.
coli} regulating expression of a fluorescent reporter. We introduce a series
of point mutations in either the inducer binding or DNA binding domain. We
then measure the full induction profile of each mutant, determine the minimal
set of parameters that are affected by the mutation, and predict
how each mutation tunes the free energy at different inducer concentrations,
repressor copy numbers, and DNA binding strengths.
We find in general that mutations in the DNA binding domain only influence
DNA binding strength, and that mutations within the inducer binding domain
affect only the parameters which dictate the allosteric response. The
degree to which these parameters are insulated is notable, as the very nature
of allostery suggests that all parameters are intimately connected, thus enabling
binding events at one domain to be "sensed" by another. 

With knowledge of how a collection of DNA binding and inducer binding single mutants
behave, we predict the induction profiles and the free energy
changes of pairwise double mutants with quantitative accuracy. We find that
the energetic effects of each individual mutation are additive, indicating that
epistatic interactions are absent between the mutations examined here. Our model
provides a means for identifying and quantifying the extent of epistatic
interactions in a more complex set of mutations, and can shed light on how the
protein sequence and general regulatory architecture coevolve.