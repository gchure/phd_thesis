
\section{Introduction}
Understanding how organisms sense and respond to changes in their environment
has long been a central theme of biological inquiry. At the cellular level,
this interaction is mediated by a diverse collection of molecular signaling
pathways. A pervasive mechanism of signaling in these pathways is allosteric
regulation, in which the binding of a ligand induces a conformational change
in some target molecule, triggering a signaling cascade \cite{Lindsley2006}.
One of the most important examples of such signaling is offered by
transcriptional regulation, where a transcription factor's propensity to bind
to DNA will be altered upon binding to an allosteric effector.

Despite allostery’s ubiquity, we lack a formal, rigorous, and generalizable
framework for studying its effects across the broad variety of contexts in
which it appears. A key example of this is transcriptional regulation, in
which allosteric transcription factors can be induced or corepressed by
binding to a ligand. An allosteric transcription factor can adopt multiple
conformational states, each of which has its own affinity for the ligand and
for its DNA target site. \textit{In vitro} studies have rigorously quantified
the equilibria of different conformational states for allosteric
transcription factors and measured the affinities of these states to the
ligand \cite{Harman2001,Lanfranco2017}. In spite of these experimental
observations, the lack of a coherent quantitative model for allosteric
transcriptional regulation has made it impossible to predict the behavior of
even a simple genetic circuit across a range of regulatory parameters.

The ability to predict circuit behavior robustly— that is, across both broad
ranges of parameters and regulatory architectures —is important for multiple
reasons. First, in the context of a specific gene, accurate prediction
demonstrates that all components relevant to the gene’s behavior have been
identified and characterized to sufficient quantitative precision. Second, in
the context of genetic circuits in general, robust prediction validates the
model that generated the prediction. Possessing a validated model also has
implications for future work. For example, when we have sufficient confidence
in the model, a single data set can be used to accurately extrapolate a
system’s behavior in other conditions. Moreover, there is an essential
distinction between a predictive model, which is used to predict a system’s
behavior given a set of input variables, and a retroactive model, which is
used to describe the behavior of data that has already been obtained. We note
that even some of the most careful and rigorous analysis of transcriptional
regulation often entails only a retroactive reflection on a single
experiment. This raises the fear that each regulatory architecture may
require a unique analysis that cannot carry over to other systems, a worry
that is exacerbated by the prevalent use of phenomenological functions (e.g.
Hill functions or ratios of polynomials) that can analyze a single data set
but cannot be used to extrapolate a system’s behavior in other conditions
\cite{Setty2003, Poelwijk2011, Vilar2013, Rogers2015, Rohlhill2017}.

This work explores what happens when theory takes center stage, namely, we
first write down the equations governing a system and describe its expected
behavior across a wide array of experimental conditions, and only then do we
set out to experimentally confirm these results. Building upon previous work
\cite{Garcia2011,Brewster2014,Weinert2014} and the work of Monod, Wyman, and
Changeux \cite{MONOD1965}, we present a statistical mechanical rendering of
allostery in the context of induction and corepression (shown schematically
in Fig. \ref{figInductionCorepressionPhenotypicProperties}(A) and
henceforth referred to as the MWC model) and use it as the basis of
parameter-free predictions which we then test experimentally. More
specifically, we study the simple repression motif – a widespread bacterial
genetic regulatory architecture in which binding of a transcription factor
occludes binding of an RNA polymerase, thereby inhibiting transcription
initiation. The MWC model stipulates that an allosteric protein fluctuates
between two distinct conformations – an active and inactive state – in
thermodynamic equilibrium \cite{MONOD1965}. During induction, for example,
effector binding increases the probability that a repressor will be in the
inactive state, weakening its ability to bind to the promoter and resulting
in increased expression. To test the predictions of our model across a wide
range of operator binding strengths and repressor copy numbers, we design an
\textit{E. coli} genetic construct in which the binding probability of a
repressor regulates gene expression of a fluorescent reporter.

In total, the work presented here demonstrates that one extremely compact set
of parameters can be applied self-consistently and predictively to different
regulatory situations including simple repression on the chromosome, cases in
which decoy binding sites for repressor are put on plasmids, cases in which
multiple genes compete for the same regulatory machinery, cases involving
multiple binding sites for repressor leading to DNA looping, and induction by
signaling
\cite{Garcia2011,Garcia2011B,Brewster2012,Boedicker2013a,Boedicker2013b,Brewster2014}.
Thus, rather than viewing the behavior of each circuit as giving rise to its
own unique input-output response, the MWC model provides a means to
characterize these seemingly diverse behaviors using a single unified
framework governed by a small set of parameters.


\begin{figure}[h!]
	\centering \includegraphics[scale=0.8]{chapter_02_induction/figs/fig1.pdf}
	\caption[Transcription regulation architectures involving an allosteric
	repressor.]{(A) We consider a promoter regulated solely by an
	allosteric repressor. When bound, the repressor prevents RNAP from binding and
	initiating transcription. Induction is characterized by the addition of an
	effector which binds to the repressor and stabilizes the inactive state
	(defined as the state which has a low affinity for DNA), thereby increasing
	gene expression. In corepression, the effector stabilizes the repressor's
	active state and thus further reduces gene expression. We list several
	characterized examples of induction and corepression that support different
	physiological roles in \textit{E. coli} \cite{Huang2011,Li2014}. (B)
	A schematic regulatory response of the two architectures shown in Panel
	(A) plotting the fold-change in gene expression as a function of
	effector concentration, where fold-change is defined as the ratio of gene
	expression in the presence versus the absence of repressor. We consider the
	following key phenotypic properties that describe each response curve: the
	minimum response (leakiness), the maximum response (saturation), the difference
	between the maximum and minimum response (dynamic range), the concentration of
	ligand which generates a fold-change halfway between the minimal and maximal
	response ($[EC_{50}]$), and the log-log slope at the midpoint of the response
	(effective Hill coefficient). (C) Over time we have refined our understanding
	of simple repression architectures. A first round of experiments used
	colorimetric assays and quantitative Western blots to investigate how
	single-site repression is modified by the repressor copy number and
	repressor-DNA binding energy \cite{Garcia2011}. A second round of experiments
	used video microscopy to probe how the copy number of the promoter and presence
	of competing repressor binding sites affect gene expression, and we use this
	data set to determine the free energy difference between the repressor’s
	inactive and active conformations \cite{Weinert2014}. Here we used flow
	cytometry to determine the inducer-repressor dissociation constants and
	demonstrate that with these parameters we can predict \textit{a priori} the
	behavior of the system for any repressor copy number, DNA binding energy, gene
	copy number, and inducer concentration.}
	\index{figures}
\label{figInductionCorepressionPhenotypicProperties}
\end{figure}
