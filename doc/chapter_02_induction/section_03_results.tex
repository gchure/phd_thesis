\section{Results}
\subsection{Experimental Design}
We test our model by predicting the induction profiles for an array of
strains that could be made using previously characterized repressor copy
numbers and DNA binding energies. Our approach contrasts with previous
studies that have parameterized induction curves of simple repression motifs,
as these have relied on expression systems where proteins are expressed from
plasmids, resulting in highly variable and unconstrained copy numbers
\cite{Murphy2007, Daber2009, Murphy2010, Daber2011a, Sochor2014}. Instead,
our approach relies on a foundation of previous work as depicted in
Fig. \ref{figInductionCorepressionPhenotypicProperties}(C). This
includes work from our laboratory that used \textit{E. coli} constructs based
on components of the \textit{lac} system to demonstrate how the Lac repressor
(LacI) copy number $R$ and operator binding energy $\Delta\varepsilon_{RA}$
affect gene expression in the absence of inducer \cite{Garcia2011}.
Ref. \cite{Rydenfelt2014B} extended the theory used in that work to the case
of multiple promoters competing for a given transcription factor, which was
validated experimentally by Ref. \cite{Brewster2014}, who modified this system
to consider expression from multiple-copy plasmids as well as the presence of
competing repressor binding sites.

The present study extends this body of work by introducing three additional
biophysical parameters -- $\Delta\varepsilon_{AI}$, $K_A$, and $K_I$ -- which
capture the allosteric nature of the transcription factor and complement the
results shown by Ref. \cite{Garcia2011} and Ref. \cite{Brewster2014}. Although the
current work focuses on systems with a single site of repression, in the appendix, 
we utilize data from Ref. \cite{Brewster2014}, in which multiple sites of repression are explored, to
characterize the allosteric free energy difference $\Delta\varepsilon_{AI}$
between the repressor's active and inactive states. As explained in that
Section, this additional data set is critical because multiple degenerate sets
of parameters can characterize an induction curve equally well, with the
$\Delta\varepsilon_{AI}$ parameter compensated by the inducer dissociation
constants $K_A$ and $K_I$ (see Fig. \ref{fig_appendix_MWC_parameter_degeneracy}).
After fixing $\Delta\varepsilon_{AI}$ as described in the
Materials \& Methods, we can use data from single-site simple repression
systems to determine the values of $K_A$ and $K_I$.

We determine the values of $K_A$ and $K_I$ by fitting to a single induction
profile using Bayesian inferential methods \cite{Sivia2006}. We then use
Eq. \ref{eq_fold_change_full} to predict gene expression for any concentration of
inducer, repressor copy number, and DNA binding energy and compare these
predictions against experimental measurements. To obtain induction profiles for
a set of strains with varying repressor copy numbers, we used modified
\textit{lacI} ribosomal binding sites from Ref. \cite{Garcia2011} to generate
strains with mean repressor copy number per cell of $R = 22 \pm 4$, $60 \pm 20$,
$124 \pm 30$, $260 \pm 40$, $1220 \pm 160$, and $1740 \pm 340$, where the error
denotes standard deviation of at least three replicates as measured by
Ref. \cite{Garcia2011}. We note that $R$ refers to the number of repressor dimers
in the cell, which is twice the number of repressor tetramers reported by
Ref. \cite{Garcia2011}; since both heads of the repressor are assumed to always
be either specifically or non-specifically bound to the genome, the two
repressor dimers in each LacI tetramer can be considered independently. Gene
expression was measured using a Yellow Fluorescent Protein (YFP) gene, driven by
a \textit{lacUV5} promoter. Each of the six repressor copy number variants were
paired with the native O1, O2, or O3 \textit{lac} operator \cite{Oehler1994}
placed at the YFP transcription start site, thereby generating eighteen unique
strains. The repressor-operator binding energies (O1 $\Delta\varepsilon_{RA} =
-15.3 \pm 0.2~k_BT$, O2 $\Delta\varepsilon_{RA} = -13.9~k_BT \pm 0.2$, and O3
$\Delta\varepsilon_{RA} = -9.7 \pm 0.1~k_BT$) were previously inferred by
measuring the fold-change of the \textit{lac} system at different repressor copy
numbers, where the error arises from model fitting \cite{Garcia2011}.
Additionally, we were able to obtain the value $\Delta \varepsilon_{AI} = 4.5\
k_BT$ by fitting to previous data as discussed in the appendix. We measure fold-change over a range of
known IPTG concentrations $c$, using $n=2$ inducer binding sites per LacI dimer
and approximating the number of non-specific binding sites as the length in
base-pairs of the \textit{E. coli} genome, $N_{NS} = 4.6 \times 10^6$.

Our experimental pipeline for determining fold-change using flow cytometry is
shown in Fig. \ref{fig_experimental_flowchart}. Briefly, cells were grown to
exponential phase, in which gene expression reaches steady state
\cite{Scott2010}, under concentrations of the inducer IPTG ranging between 0
and $5$ mM. We measure YFP fluorescence using flow cytometry and
automatically gate the data to include only single-cell measurements (see
Materials \& Methods). To validate
the use of flow cytometry, we also measured the fold-change of a subset of
strains using the established method of single-cell microscopy (see Appendix). We found that the fold-change measurements obtained
from microscopy were indistinguishable from that of flow-cytometry and yielded
values for the inducer binding constants $K_A$ and $K_I$ that were within error.

\begin{figure}[h!]
	\centering \includegraphics[scale=0.8]{chapter_02_induction/figs/fig3.pdf}
	\caption[An experimental pipeline for high-throughput fold-change
		measurements.]{Cells are grown to exponential steady state and their
	fluorescence is measured using flow cytometry. Automatic gating methods using
	forward- and side-scattering are used to ensure that all measurements come from
	single cells (see Methods). Mean expression is then quantified at different
	IPTG concentrations (top, blue histograms) and for a strain without repressor
	(bottom, green histograms), which shows no response to IPTG as expected.
	Fold-change is computed by dividing the mean fluorescence in the presence of
	repressor by the mean fluorescence in the absence of repressor.}
\label{fig_experimental_flowchart}
\index{figures}
\end{figure}

%%%%%%%%%%%%%%%%%%%%%%%%%%%%%%%%%%%%%%%%%%%%%%%%%%%%%%%%%%%%%%%%%%%%%%%%%%%
\subsection{Determination of the \textit{in vivo} MWC Parameters}

The three parameters that we tune experimentally are shown in
Fig. \ref{fig_O2_R260_fit}(A), leaving the three allosteric parameters
($\Delta \varepsilon_{AI}$, $K_A$, and $K_I$) to be determined by fitting. We
used previous LacI fold-change data \cite{Brewster2014} to infer that
$\Delta\varepsilon_{AI} = 4.5~k_BT$ (see Materials \& Methods). Rather than fitting $K_A$ and $K_I$ to our entire
data set of eighteen unique constructs, we performed Bayesian parameter
estimation on data from a single strain with $R=260$ and an O2 operator
($\Delta\varepsilon_{RA}=-13.9~k_BT$ \cite{Garcia2011}) shown in
Fig. \ref{fig_O2_R260_fit} (D, white circles). Using Markov Chain Monte
Carlo, we determine the most likely parameter values to be $K_A=139^{+29}_{-22}
\times 10^{-6}$ M and $K_I=0.53^{+0.04}_{-0.04} \times 10^{-6}$ M
, which are the modes of their respective distributions, where the
superscripts and subscripts represent the upper and lower bounds of the
$95^\text{th}$ percentile of the parameter value distributions [see
Fig. \ref{fig_O2_R260_fit}(B)]. Unfortunately, we are not able to make a
meaningful value-for-value comparison of our parameters to those of earlier
studies \cite{Daber2009, Daber2011a} because of uncertainties in both gene copy
number and transcription factor copy numbers in these studies, as illustrated by
the plots in Appendix. We then predicted the fold-change
for the remaining seventeen strains with no further fitting (see
Fig. \ref{fig_O2_R260_fit}(C-E) together with the specific
phenotypic properties described in Fig. \ref{figInductionCorepressionPhenotypicProperties} and discussed in detail
below (see Fig. \ref{fig_O2_R260_fit}(F-J). The shaded regions in
Fig. \ref{fig_O2_R260_fit}(C-D) denote the 95\% credible regions.
Factors determining the width of the credible regions are explored in the appendix.


We stress that the entire suite of predictions in Fig. \ref{fig_O2_R260_fit} is
based upon the induction profile of a single strain. Our ability to make such a
broad range of predictions stems from the fact that our parameters of interest -
such as the repressor copy number and DNA binding energy - appear as distinct
physical parameters within our model. While the single data set in
Fig. \ref{fig_O2_R260_fit}(D) could also be fit using a Hill function, such
an analysis would be unable to predict any of the other curves in the figure
(see Materials \& Methods).
Phenomenological expressions such as the Hill function can describe data, but
lack predictive power and are thus unable to build our intuition, help us design
\textit{de novo} input-output functions, or guide future experiments
\cite{Kuhlman2007, Murphy2007}.

\begin{figure}[p]
	\centering \includegraphics[scale=0.7]{chapter_02_induction/figs/fig4.pdf}

	\caption[Predicting induction profiles for different biological control
		parameters.] {(A) We can quantitatively tune $R$ via ribosomal
	binding site (RBS) modifications, $\Delta\varepsilon_{RA}$ by mutating the
	operator sequence, and $c$ by adding different amounts of IPTG to the growth
	medium. (B) Previous experiments have characterized the $R$,
	$N_{NS}$, $\Delta\varepsilon_{RA}$, and $\Delta\varepsilon_{AI}$ parameters
	[see Fig. \ref{figInductionCorepressionPhenotypicProperties}(C)], leaving only the unknown dissociation constants
	$K_A$ and $K_I$ between the inducer and the repressor in the active and
	inactive states, respectively. These two parameters can be inferred using
	Bayesian parameter estimation from a single induction curve. (C - E)
	Predicted IPTG titration curves for different repressor copy numbers and
	operator strengths. Titration data for the O2 strain [white circles in Panel
	(D)] with $R=260$, $\Delta\varepsilon_{RA} = -13.9~k_BT$, $n=2$, and
	$\Delta\varepsilon_{AI}=4.5\,k_BT$ can be used to determine the thermodynamic
	parameters $K_A=139^{+29}_{-22} \times 10^{-6}$ M and
	$K_I=0.53^{+0.04}_{-0.04} \times 10^{-6}$ M (orange line). The
	remaining solid lines predict the fold-change Eq. \ref{eq_fold_change_full} for
	all other combinations of repressor copy numbers (shown in the legend) and
	repressor-DNA binding energies corresponding to the O1 operator ($-15.3~k_B
	T$), O2 operator ($-13.9~k_B T$), and O3 operator ($-9.7~k_B T$). Error bars of
	experimental data show the standard error of the mean (eight or more
	replicates) when this error is not smaller than the diameter of the data point.
	The shaded regions denote the 95\% credible region, although the credible
	region is obscured when it is thinner than the curve itself. To display the
	measured fold-change in the absence of inducer, we alter the scaling of the
	$x$-axis between $0$ and $10^{-7}$ M to linear rather than logarithmic, as
	indicated by a dashed line. Additionally, our model allows us to investigate
	key phenotypic properties of the induction profiles [see
	Fig. \ref{figInductionCorepressionPhenotypicProperties}(B)]. Specifically,
	we show predictions for the (F) leakiness, (G)
	saturation, (H) dynamic range, (I) $[EC_{50}]$, and
	(J) effective Hill coefficient of the induction profiles.}
\label{fig_O2_R260_fit}
\end{figure}

%%%%%%%%%%%%%%%%%%%%%%%%%%%%%%%%%%%%%%%%%%%%%%%%%%%%%%%%%%%%%%%%%%%%%%%%%%%
\subsection{Comparison of Experimental Measurements with Theoretical
Predictions}

We tested the predictions shown in Fig. \ref{fig_O2_R260_fit} by measuring
fold-change induction profiles in strains with a broad range of repressor copy
numbers and repressor binding energies as characterized in
Ref. \cite{Garcia2011}. With a few notable exceptions, the results shown in
Fig. \ref{fig_O2_R260_pred_data} demonstrate agreement between theory and
experiment. We note  that there was an apparently systematic shift in the O3
$\Delta\varepsilon_{RA} = -9.7\ k_BT$ strains
[Fig. \ref{fig_O2_R260_pred_data}(C)] and all of the $R=1220$ and $R =1740$
strains. This may be partially due to imprecise previous determinations of their
$\Delta\varepsilon_{RA}$ and $R$ values. By performing a global fit where we
infer all parameters including the repressor copy number $R$ and the binding
energy $\Delta\varepsilon_{RA}$, we found better agreement for these strains,
although a discrepancy in the steepness of the response for all O3 strains
remains (see Materials \& Methods).
We considered a number of hypotheses to explain these discrepancies such as
including other states (e.g. non-negligible binding of the inactive repressor),
relaxing the weak promoter approximation, and accounting for variations in gene
and repressor copy number throughout the cell cycle, but none explained the
observed discrepancies. As an additional test of our model, we considered
strains using the synthetic Oid operator which exhibits an especially strong
binding energy of $\Delta\varepsilon_{RA}=-17~k_B T$ \cite{Garcia2011}. The
global fit agrees well with the Oid microscopy data, though it asserts a
stronger Oid binding energy of $\Delta\varepsilon_{RA}=-17.7~k_B T$ (see appendix).

To ensure that the agreement between our predictions and data is not an accident
of the strain we used to perform our fitting, we also inferred $K_A$ and $K_I$
from each of the other strains. As shown in Materials \& Methods
and Fig. \ref{fig_O2_R260_pred_data}(D), the inferred values of $K_A$ and $K_I$
depend minimally upon which strain is chosen, indicating that these parameter
values are highly robust. We also performed a global fit using the data from all
eighteen strains in which we fitted for the inducer dissociation constants $K_A$
and $K_I$, the repressor copy number $R$, and the repressor DNA binding energy
$\Delta\varepsilon_{RA}$ (see Materials \& Methods). The resulting parameter values were nearly
identical to those fitted from any single strain. For the remainder of the text
we continue using parameters fitted from the strain with $R=260$ repressors and
an O2 operator.

\begin{figure}[h]
	\centering \includegraphics[scale=0.7]{chapter_02_induction/figs/fig5.pdf}
	\caption[Comparison of predictions against measured and inferred data.]{
	Flow cytometry measurements of fold-change over a range of IPTG concentrations
	for (A) O1, (B) O2, and (C) O3 strains at
	varying repressor copy numbers, overlaid on the predicted responses. Error bars
	for the experimental data show the standard error of the mean (eight or more
	replicates). As discussed in Fig. \ref{fig_O2_R260_fit}, all of the predicted
	induction curves were generated prior to measurement by inferring the MWC
	parameters using a single data set [O2 $R=260$, shown by white circles in Panel
	(B)]. The predictions may therefore depend upon which strain is used to
	infer the parameters. (D) Inferred parameter values of the
	dissociation constants $K_A$ and $K_I$ using any of the eighteen strains
	instead of the O2 $R=260$ strain. Nearly identical parameter values are
	inferred from each strain, demonstrating that the same set of induction
	profiles would have been predicted regardless of which strain was chosen. The
	points show the mode, and the error bars denote the $95\%$ credible region of
	the parameter value distribution. Error bars not visible are smaller than the
	size of the marker.} 
	\label{fig_O2_R260_pred_data}
	\index{figures}
\end{figure}

\subsection{Predicting the Phenotypic Traits of the Induction Response}
A subset of the
properties shown in Fig. \ref{figInductionCorepressionPhenotypicProperties} (i.e.
the leakiness, saturation, dynamic range, $[EC_{50}]$, and effective Hill
coefficient) are of significant interest to synthetic biology. For example, synthetic biology is often
focused on generating large responses (i.e. a large dynamic range) or finding a
strong binding partner (i.e. a small $[EC_{50}]$) \cite{Brophy2014, Shis2014}.
While these properties are all individually informative, when taken together
they capture the essential features of the induction response. We reiterate that
a Hill function approach cannot predict these features \textit{a priori} and
furthermore requires fitting each curve individually. The MWC model, on the
other hand, enables us to quantify how each trait depends upon a single set of
physical parameters as shown by Fig. \ref{fig_O2_R260_fit}(F-J).

We define these five phenotypic traits using expressions derived from the model,
Eq. \ref{eq_fold_change_full}. These results build upon extensive work by
Ref. \cite{Martins2011}, who computed many such properties for ligand-receptor
binding within the MWC model. We begin by analyzing the leakiness, which is the
minimum fold-change observed in the absence of ligand, given by

\begin{equation}
\label{eqLeakiness}
\text{leakiness} = \text{fold-change}(c=0) \nonumber\\
= \left(
	1+\frac{1}{1+e^{-\beta \Delta \varepsilon_{AI} }}\frac{R}{N_{NS}}e^{-\beta \Delta\varepsilon_{RA}} \right)^{-1},
\end{equation}

and the saturation, which is the maximum fold change observed in the presence of saturating ligand,

\begin{equation}
\label{eqSaturation}
\text{saturation} = \text{fold-change}(c \to \infty) \nonumber\\
= \left(
	1+\frac{1}{1+e^{-\beta \Delta \varepsilon_{AI} } \left(\frac{K_A}{K_I}\right)^n }\frac{R}{N_{NS}}e^{-\beta \Delta\varepsilon_{RA}} \right)^{-1}.
\end{equation}


Systems that minimize leakiness repress strongly in the absence of effector
while systems that maximize saturation have high expression in the presence of
effector. Together, these two properties determine the dynamic range of a
system's response, which is given by the difference
\begin{equation} \label{eqDynamicRangeDef}
	\text{dynamic\,range} = \text{saturation} - \text{leakiness}.
\end{equation}
These three properties are shown in Fig. \ref{fig_O2_R260_fit}(F - H).
We discuss these properties in greater detail in the Materials \& Methods.
Fig. \ref{fig_properties_data}(A - C) shows that the measurements of
these three properties, derived from the fold-change data in the absence of IPTG
and the presence of saturating IPTG, closely match the predictions for all three
operators.

\begin{figure}[h!]
	\centering \includegraphics[scale=0.7]{chapter_02_induction/figs/fig6.pdf}
	\caption[Predictions and experimental measurements of key properties of
		induction profiles.]{Data for the (A) leakiness, (B)
	saturation, and (C) dynamic range are obtained from fold-change
	measurements in Fig. \ref{fig_O2_R260_pred_data} in the absence of IPTG and at
	saturating concentrations of IPTG. The three repressor-operator binding
	energies in the legend correspond to the O1 operator ($-15.3~k_B T$), O2
	operator ($-13.9~k_B T$), and O3 operator ($-9.7~k_B T$). Both the
	(D) $[EC_{50}]$ and (E) effective Hill coefficient are
	inferred by individually fitting each operator-repressor pairing in
	Fig. \ref{fig_O2_R260_pred_data}(A - C) separately to
	Eq. \ref{eq_fold_change_full} in order to smoothly interpolate between the data
	points. Error bars for (A - C) represent the standard error of
	the mean for eight or more replicates; error bars for (D - E) 
	represent the 95\% credible region for the parameter found by propagating the
	credible region of our estimates of $K_A$ and $K_I$ into
	Eq. \ref{ec50} and Eq. \ref{effectiveHill}.} \label{fig_properties_data}
	\index{figures}
\end{figure}

Two additional properties of induction profiles are the $[EC_{50}]$ and
effective Hill coefficient, which determine the range of inducer concentration
in which the system's output goes from its minimum to maximum value. The
$[EC_{50}]$ denotes the inducer concentration required to generate a system
response Eq. \ref{eq_fold_change_full} halfway between its minimum and maximum
value,
\begin{equation} \label{ec50}
\text{fold-change}(c = [EC_{50}]) = \frac{\text{leakiness} + \text{saturation}}{2}.
\end{equation}
The effective Hill coefficient $h$, which quantifies the steepness of the
curve at the $[EC_{50}]$ \cite{Marzen2013}, is given by
\begin{equation} \label{effectiveHill}
h = \left( 2 \frac{d}{d \log c} \left[ \log \left( \frac{ \text{fold-change}(c) - \text{leakiness}}{\text{dynamic\,range}} \right) \right] \right)_{c = [EC_{50}]}.
\end{equation}
Fig. \ref{fig_O2_R260_fit}(I-J) shows how the $[EC_{50}]$ and
effective Hill coefficient depend on the repressor copy number. In
the Materials \& Methods, we discuss
the analytic forms of these two properties as well as their dependence on the
repressor-DNA binding energy.

Fig. \ref{fig_properties_data}(D-E) shows the estimated values of
the $[EC_{50}]$ and the effective Hill coefficient overlaid on the theoretical
predictions. Both properties were obtained by fitting Eq. \ref{eq_fold_change_full}
to each individual titration curve and computing the $[EC_{50}]$ and effective
Hill coefficient using Eq. \ref{ec50} and Eq. \ref{effectiveHill}, respectively. We
find that the predictions made with the single strain fit closely match those
made for each of the strains with O1 and O2 operators, but the predictions for
the O3 operator are markedly off. In the Materials \& Methods, we show that the large, asymmetric error bars
for the O3 $R=22$ strain arise from its nearly flat response, where the lack of
dynamic range makes it impossible to determine the value of the inducer
dissociation constants $K_A$ and $K_I$, as can be seen in the uncertainty of
both the $[EC_{50}]$ and effective Hill coefficient. Discrepancies between
theory and data for O3 are improved, but not fully resolved, by performing a
global fit or fitting the MWC model individually to each curve (see
Materials \& Methods). It remains an open question how to
account for discrepancies in O3, in particular regarding the significant
mismatch between the predicted and fitted effective Hill coefficients.


\subsection{Data Collapse of Induction Profiles}
Our primary interest heretofore was to determine the system response at a
specific inducer concentration, repressor copy number, and repressor-DNA binding
energy. However, the cell does not necessarily ``care about'' the precise number
of repressors in the system or the binding energy of an individual operator. The
relevant quantity for cellular function is the fold-change enacted by the
regulatory system. This raises the question: given a specific value of the
fold-change, what combination of parameters will give rise to this desired
response? In other words, what trade-offs between the parameters of the system
will give rise to the same mean cellular output? These are key questions both
for understanding how the system is governed and for engineering specific
responses in a synthetic biology context. To address these questions, we follow
the data collapse strategy used in a number of previous studies
\cite{Sourjik2002, Keymer2006, Swem2008}, and rewrite
Eq. \ref{eq_fold_change_full} as a Fermi function,
\begin{equation}
	\label{eq_free_energy_definition} 
	\text{fold-change} = \frac{1}{1+e^{-
F(c)}}, \end{equation} where $F(c)$ is the free energy of the repressor binding
to the operator of interest relative to the unbound operator state in $k_B T$
units  \cite{Keymer2006, Swem2008, Phillips2015a}, which is given by

\begin{equation}
	\label{eq_free_energy_MWC_parameters}
F(c) = \frac{\Delta\varepsilon_{RA}}{k_BT} - \log \frac{\left(1+\frac{c}{K_A}\right)^n}{\left(1+\frac{c}{K_A}\right)^n+e^{-\beta \Delta\varepsilon_{AI} }\left(1+\frac{c}{K_I}\right)^n} - \log \frac{R}{N_{NS}}.
\end{equation}

The first term in $F(c)$ denotes the repressor-operator binding energy, the second the contribution from the inducer
concentration, and the last
the effect of the repressor copy number. We note that elsewhere, this free energy
has been dubbed the Bohr parameter since such families of curves are analogous
to the shifts in hemoglobin binding curves at different pHs known as the Bohr
effect \cite{Mirny2010, Phillips2015a, Einav2016}.

Instead of analyzing each induction curve individually, the free energy provides
a natural means to simultaneously characterize the diversity in our eighteen
induction profiles. Fig. \ref{fig_datacollapse}(A) demonstrates how the
various induction curves from Fig. \ref{fig_O2_R260_fit}(C-E) all
collapse onto a single master curve, where points from every induction profile
that yield the same fold-change are mapped onto the same free energy.
Fig. \ref{fig_datacollapse}(B) shows this data collapse for the 216 data
points in Fig. \ref{fig_O2_R260_pred_data}(A - C), demonstrating the
close match between the theoretical predictions and experimental measurements
across all eighteen strains.

There are many different combinations of parameter values that can result in the
same free energy as defined in Eq. \ref{eq_free_energy_MWC_parameters}. For
example, suppose a system originally has a fold-change of 0.2 at a specific
inducer concentration, and then operator mutations increase the
$\Delta\varepsilon_{RA}$ binding energy \cite{Garcia2012}. While this serves to
initially increase both the free energy and the fold-change, a subsequent
increase in the repressor copy number could bring the cell back to the original
fold-change level. Such trade-offs hint that there need not be a single set of
parameters that evoke a specific cellular response, but rather that the cell
explores a large but degenerate space of parameters with multiple, equally valid
paths.

\begin{figure}[ht]
	\centering \includegraphics[scale=0.6]{chapter_02_induction/figs/fig7.pdf}
	\caption[Fold-change data from a broad collection of different strains
		collapse onto a single master curve.]{(A) Any combination of
	parameters can be mapped to a single physiological response (i.e. fold-change)
	via the free energy, which encompasses the parametric details of the model.
	(B) Experimental data from Fig. \ref{fig_O2_R260_pred_data} collapse
	onto a single master curve as a function of the free energy
	Eq. \ref{eq_free_energy_MWC_parameters}. The free energy for each strain was
	calculated from Eq. \ref{eq_free_energy_MWC_parameters} using $n=2$,
	$\Delta\varepsilon_{AI}=4.5~k_BT$, $K_A=139 \times 10^{-6} \, \text{M}$,
	$K_I=0.53 \times 10^{-6}\, \text{M}$, and the strain-specific $R$ and
	$\Delta\varepsilon_{RA}$. All data points represent the mean, and error bars are
	the standard error of the mean for eight or more replicates.}
\label{fig_datacollapse}
\end{figure}