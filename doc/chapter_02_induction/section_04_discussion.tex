%%%%%%%%%%%%%%%%%%%%%%%%%%%%%%%%%%%%%%%%%%%%%%%%%%%%%%%%%%%%%%%%%%%%%%%%%%%
\section*{Discussion}

Since the early work by Monod, Wyman, and Changeux \cite{Monod1963,
MONOD1965}, an array of biological phenomena has been tied to the existence
of macromolecules that switch between inactive and active states. Examples
can be found in a wide variety of cellular processes, including ligand-gated
ion channels \cite{Auerbach2012}, enzymatic reactions \cite{Velyvis2007,
Einav2016}, chemotaxis \cite{Keymer2006}, quorum sensing \cite{Swem2008},
G-protein coupled receptors \cite{Canals2012}, physiologically important
proteins \cite{Milo2007, Levantino2012a}, and beyond. One of the most
ubiquitous examples of allostery is in the context of gene expression, where
an array of molecular players bind to transcription factors to influence
their ability to regulate gene activity \cite{Huang2011, Li2014}. A number of
studies have focused on developing a quantitative understanding of allosteric
regulatory systems. Ref. \cite{Martins2011, Marzen2013} analytically derived
fundamental properties of the MWC model, including the leakiness and dynamic
range described in this work, noting the inherent trade-offs in these
properties when tuning the model's parameters. Work in the Church and Voigt
labs, among others, has expanded on the availability of allosteric circuits
for synthetic biology \cite{Lutz1997, Moon2012, Rogers2015, Rohlhill2017}.
Recently, Daber \textit{et al.} theoretically explored the induction of
simple repression within the MWC model \cite{Daber2009} and experimentally
measured how mutations alter the induction profiles of transcription factors
\cite{Daber2011a}. Vilar and Saiz analyzed a variety of interactions in
inducible \textit{lac}-based systems including the effects of oligomerization
and DNA folding on transcription factor induction \cite{Leonor2008,
Vilar2013}. Other work has attempted to use the \textit{lac} system to
reconcile \textit{in vitro} and \textit{in vivo} measurements
\cite{Tungtur2011, Sochor2014}.

Although this body of work has done much to improve our understanding of
allosteric transcription factors, there have been few attempts to explicitly
connect quantitative models to experiments. Here, we generate a predictive
model of allosteric transcriptional regulation and then test the model
against a thorough set of experiments using well-characterized regulatory
components. Specifically, we used the MWC model to build upon a
well-established thermodynamic model of transcriptional regulation
\cite{Bintu2005, Garcia2011}, allowing us to compose the model from a minimal
set of biologically meaningful parameters. This model combines both
theoretical and experimental insights; for example, rather than considering
gene expression directly we analyze the fold-change in expression, where the
weak promoter approximation [see Eq. \ref{eq_fold_change_approx}] circumvents
uncertainty in the RNAP copy number. The resulting model depended upon
experimentally accessible parameters, namely, the repressor copy number, the
repressor-DNA binding energy, and the concentration of inducer.
We tested these predictions on a range of strains whose repressor
copy number spanned two orders of magnitude and whose DNA binding affinity
spanned 6 $k_BT$. We argue that one would not be able to generate such a wide
array of predictions by using a Hill function, which abstracts away the
biophysical meaning of the parameters into phenomenological parameters
\cite{Forsen1995}.

More precisely, we tested our model in the context of a \textit{lac}-based simple repression
system by first determining the allosteric dissociation constants $K_A$ and
$K_I$ from a single induction data set (O2 operator with binding energy $\Delta
\varepsilon_{RA} = -13.9~k_BT$ and repressor copy number $R = 260$) and then
using these values to make parameter-free predictions of the induction profiles
for seventeen other strains where $\Delta \varepsilon_{RA}$ and $R$ were varied
significantly [see Fig. \ref{fig_O2_R260_fit}]. We next measured the induction
profiles of these seventeen strains using flow cytometry and found that our
predictions consistently and accurately captured the primary features for each
induction data set, as shown in
Fig. \ref{fig_O2_R260_pred_data}(A-C). Importantly, we find that
fitting $K_A$ and $K_I$ to data from any other strain would have resulted in
nearly identical predictions (see Fig. \ref{fig_O2_R260_pred_data}(D) and
the Materials \& Methods. This
suggests that a few carefully chosen measurements can lead to a deep
quantitative understanding of how simple regulatory systems work without
requiring an extensive sampling of strains that span the parameter space.
Moreover, the fact that we could consistently achieve reliable predictions after
fitting only two free parameters stands in contrast to the common practice of
fitting several free parameters simultaneously, which can nearly guarantee an
acceptable fit provided that the model roughly resembles the system response,
regardless of whether the details of the model are tied to any underlying
molecular mechanism.

Beyond observing changes in fold-change as a function of effector concentration,
our application of the MWC model allows us to explicitly predict the values of
the induction curves' key parameters, namely, the leakiness, saturation, dynamic
range, $[EC_{50}]$, and the effective Hill coefficient [see
Fig. \ref{fig_properties_data}]. We are consistently able to accurately predict the
leakiness, saturation, and dynamic range for each of the strains. For both the
O1 and O2 data sets, our model also accurately predicts the effective Hill
coefficient and $[EC_{50}]$, though these predictions for O3 are noticeably less
accurate. While performing a global fit for all model parameters marginally
improves the prediction for O3 (Materials \& Methods), we are still unable to accurately predict
the effective Hill coefficient or the $[EC_{50}]$. We further tried including
additional states (such as allowing the inactive repressor to bind to the
operator), relaxing the weak promoter approximation, accounting for changes in
gene and repressor copy number throughout the cell cycle \cite{Jones2014a}, and
refitting the original binding energies from Ref. \cite{Garcia2011B}, but we were
still unable to account for the O3 data. It remains an open question as to how
the discrepancy between the theory and measurements for O3 can be reconciled.

The dynamic range, which is of considerable interest when designing or
characterizing a genetic circuit, is revealed to have an interesting property:
although changing the value of $\Delta \varepsilon_{RA}$ causes the dynamic
range curves to shift to the right or left, each curve has the same shape and in
particular the same maximum value. This means that strains with strong or weak
binding energies can attain the same dynamic range when the value of $R$ is
tuned to compensate for the binding energy. This feature is not immediately apparent
from the IPTG induction curves, which show very low dynamic ranges for several
of the O1 and O3 strains. Without the benefit of models that can predict such
phenotypic traits, efforts to engineer genetic circuits with allosteric
transcription factors must rely on trial and error to achieve specific responses
\cite{Rogers2015,Rohlhill2017}.

Despite the diversity observed in the induction profiles of each of our
strains, our data are unified by their reliance on fundamental biophysical
parameters. In particular, we have shown that our model for fold-change can
be rewritten in terms of the free energy Eq.
\ref{eq_free_energy_MWC_parameters}, which encompasses all of the physical
parameters of the system. This has proven to be an illuminating technique in
a number of studies of allosteric proteins
\cite{Sourjik2002,Keymer2006,Swem2008}. Although it is experimentally
straightforward to observe system responses to changes in effector
concentration $c$, framing the input-output function in terms of $c$ can give
the misleading impression that changes in system parameters lead to
fundamentally altered system responses. Alternatively, if one can find the
``natural variable" that enables the output to collapse onto a single curve,
it becomes clear that the system's output is not governed by individual
system parameters, but rather the contributions of multiple parameters that
define the natural variable. When our fold-change data are plotted against
the respective free energies for each construct, they collapse cleanly onto a
single curve [see Fig. \ref{fig_datacollapse}]. This enables us to analyze
how parameters can compensate each other. For example, rather than viewing
strong repression as a consequence of low IPTG concentration $c$ or high
repressor copy number $R$, we can now observe that strong repression is
achieved when the free energy $F(c) \leq -5 k_BT$, a condition which can be
reached in a number of ways.

While our experiments validated the theoretical predictions in the case of
simple repression, we expect the framework presented here to apply much more
generally to different biological instances of allosteric regulation. For
example, we can use this model to study more complex systems such as
when transcription factors interact with multiple operators
\cite{Bintu2005}. We can further explore different regulatory configurations
such as corepression, activation, and coactivation, each of which are found in
\textit{E. coli} (see Appendix \ref{AppendixApplications}). This work can also
serve as a springboard to characterize not just the mean but the full gene
expression distribution and thus quantify the impact of noise on the system
\cite{eldar2010}. Another extension of this approach would be to theoretically
predict and experimentally verify whether the repressor-inducer dissociation
constants $K_A$ and $K_I$ or the energy difference $\Delta \varepsilon_{AI}$
between the allosteric states can be tuned by making single amino acid
substitutions in the transcription factor \cite{Daber2011a, Phillips2015a}.
Finally, we expect that the kind of rigorous quantitative description of the
allosteric phenomenon provided here will make it possible to construct
biophysical models of fitness for allosteric proteins similar to those already
invoked to explore the fitness effects of transcription factor binding site
strengths and protein stability \cite{Gerland2002, Berg2004, Zeldovich2008}.

To conclude, we find that our application of the MWC model provides an accurate,
predictive framework for understanding simple repression by allosteric
transcription factors. To reach this conclusion, we analyzed the model in the
context of a well-characterized system, in which each parameter had a clear
biophysical meaning. As many of these parameters had been measured or inferred
in previous studies, this gave us a minimal model with only two free parameters
which we inferred from a single data set. We then accurately predicted the
behavior of seventeen other data sets in which repressor copy number and
repressor-DNA binding energy were systematically varied. In addition, our model
allowed us to understand how key properties such as the leakiness, saturation,
dynamic range, $[EC_{50}]$, and effective Hill coefficient depended upon the
small set of parameters governing this system. Finally, we show that by framing
inducible simple repression in terms of free energy, the data from all of our
experimental strains collapse cleanly onto a single curve, illustrating the many
ways in which a particular output can be targeted. In total, these results show
that a thermodynamic formulation of the MWC model supersedes phenomenological
fitting functions for understanding transcriptional regulation by allosteric
proteins.
